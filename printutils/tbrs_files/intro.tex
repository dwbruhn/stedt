\renewcommand\thesection{\arabic{section}}
\renewcommand\thesubsection{\thesection.\arabic{subsection}}
\renewcommand\theenumi{\alph{enumi}} % use letters for top-level enums
\renewcommand\labelenumi{(\theenumi)} % use parens for top-level enum labels

\chapter{Introduction}


\section{The place of this volume in the STEDT project}

The Sino-Tibetan (ST) language family, comprising Chinese on the one hand,
and the hundreds of Tibeto-Burman (TB) languages on the other, is one of the
largest in the world, with well over a billion and a half
speakers.\footnote{Some scholars, especially in China,
consider Sino-Tibetan to include the
Tai-Kadai (TK) and Hmong-Mien (HM) (=Miáo-Yáo) language families as well.  While
there is definitely a striking typological similarity among Chinese, TK, and HM,
this is undoubtedly due to prolonged ancient contact rather than genetic
relationship.  See Benedict 1975a (\textit{Austro-Thai Language and Culture, with a
glossary of roots}).}  Yet the field of ST linguistics is only about 70 years
old, and many TB languages remain virtually unstudied.  The \textit{Sino-Tibetan
Etymological Dictionary and Thesaurus} project (STEDT) was begun in August 1987,
with the goal of reconstructing the lexicon of Proto-Sino-Tibetan and
Proto-Tibeto-Burman from both the phonological and the semantic point of view.

In a sense the present work is a companion volume to the \textit{Handbook of
Proto-Tibeto-Burman} (\textit{HPTB}; Matisoff 2003), where TB/ST roots were discussed,
sorted, and analyzed according to their \textit{phonological shapes}, regardless of their
meanings.  In the present volume, a group of phonologically disparate TB/ST
etyma have been assembled according to their \textit{meanings}, all of which have to do
with the body’s reproductive system.\footnote{My ultimate inspiration for a
thesaurus-like approach to the proto-lexicon was Buck 1949 \textit{(A Dictionary of
Selected Synonyms in the Principal Indo-European Languages: a contribution to
the history of ideas}), a copy of which I purchased as a graduate student in the
early 1960’s, at the then astronomical price of \$40.  In each section of this
great work, arranged Roget-like into semantic categories and subcategories, Buck
first lists the forms for each concept in 30-plus modern and ancient IE
languages; then he assembles these synonymous forms into etymological groups. 
Each of these etyma is briefly discussed in terms of breadth of attestation,
solidity of the reconstruction, and semantic connections with other areas of the
lexicon.}

Even though the number of etyma reconstructed in this volume (nearly 200) is
not inconsiderable, they represent only a small fraction of the thousands of
roots in the proto-lexicon.  Experience has taught us that STEDT’s original goal
of simultaneously etymologizing the entire vocabulary of the proto-language was
unrealistic.  As originally conceived, STEDT was to produce a series of large
print volumes, each devoted to the exhaustive presentation of the reconstructed
roots in a specific semantic area, covering the entire lexicon, approximately as
follows:

\begin{quote}
{\footnotesize
Volume I: \textit{Body Parts}\nopagebreak[4]\\
Volume II: \textit{Animals}\\
Volume III: \textit{Natural Objects, Plants, Foods }\\
Volume IV: \textit{Kinship Terms, Ethnonyms, Social Roles}\\
Volume V: \textit{Culture, Artifacts, Religion}\\
Volume VI: \textit{Verbs of Motion, of Manipulation, and of Production}\\
Volume VII: \textit{Adjectival Verbs}\\
Volume VIII: \textit{Abstract Nouns and Verbs, Psychological Verbs, Verbs of Utterance }\\
Volume IX: \textit{Shape, Size, Color, Measure, Number, Time, Space}\\
Volume X: \textit{Grammatical words}\\
}
\end{quote}

Each volume was in turn to be divided into a number of smaller units called
“fascicles”. Thus Vol.~I \textit{Body Parts} was to comprise the following nine
fascicles:

\begin{quote}
{\footnotesize
1. \textit{Body (general)}\\
2. \textit{Head and Face}\\
3. \textit{Mouth and Throat}\\
4. \textit{Torso}\\
5. \textit{Limbs, Joints, and Body Measures}\\
6. \textit{Diffuse Organs}\\
7. \textit{Internal Organs}\\
8. \textit{Secretions and Somatophonics}\footnote{By “somatophonics” I mean sneezes,
belches, farts, and the like.}\\
9. \textit{Reproductive System}\\
}
\end{quote}

Despite the limitations of computer technology in the 1980s and 1990s, the
STEDT staff and I managed to build up a database of nearly 300,000 forms from
some 250 TB languages and dialects, using a wide variety of heterogeneous
sources.\footnote{See the section on \textit{Source Abbreviations} below.
}
I spent
several years laboriously “tagging” tens of thousands of individual words and
syllables in the database with numerals, each of which corresponded to a
reconstructed etymon in an ever-growing list of “official STEDT
roots”.\footnote{For example, the Lahu word \textbf{ɣû-tu-šī} ‘navel’ was tagged with
the numerals “137, 520, 1019”, indicating that the first syllable descends from PTB
\textbf{*ram} ‘belly’, the second syllable from PTB \textbf{*du} ‘navel’ (see \#44 in this volume),
and the third syllable from PTB \textbf{*sey} ‘fruit; small round object’.}  All forms
tagged with a certain number could then be assembled into an etymological set
supporting the reconstruction.  Some 2000 etyma were eventually set up in this
manner.  As the work proceeded, however, every subpart and sub-subpart of the
lexicon expanded and bloomed into a major project, forcing me to gradually lower
my sights: first from dealing with the whole lexicon to confining myself to
bodyparts; then from dealing with the whole body to confining myself to one of
the nine “fascicles” of the bodypart volume as originally planned.  I decided to
select the Reproductive System as a pilot project, not merely for its prurient
interest, but also because this semantic field tends to be neglected in
historical linguistic studies, despite the fact that it is particularly rich in
metaphorical associations with other areas of the lexicon.


Clearly it would be impractical to continue with print publications in this
fashion for a century or so until the entire lexicon is covered.  Our approach
in the future must be electronic and interactive.  Over the next several years,
it is planned gradually to release groups of STEDT etymologies on the web,
perhaps 25 or so at a time, in one semantic field after another.  This
electronic conduit I would like to call the \textit{STEDT Root Canal}.
Colleagues will
be invited to comment on roots already reconstructed and to establish new ones.

\section{Structure of the chapters and sections}


The material in each of the nine chapters of this book is presented in a
certain order, as outlined in 2.1-2.9.

\subsection{Semantic diagrams}


Each chapter begins with a semantic diagram.  These diagrams, called
“metastatic flowcharts” in STEDT parlance, were first introduced in Matisoff 1978a (\textit{VSTB}),
and have been used subsequently in many of my articles.\footnote{See, e.g.,
Matisoff 1980 (“Stars, moon, and spirits”); 1985a (“God and the ST copula”);
1985b (“Arm, hand, wing”); 1988b (“Property, livestock, talent”); 1991a
(“Grammatization in Lahu”); 1991b (“Mother of all morphemes”); 1994b (“Buttock
and dull”); 2000b (“Three TB word-families”); 2004 (“Areal semantics”).}
They
are intended to represent the paths of semantic association undergone by etyma,
as established by comparative/historical and/or internal synchronic evidence. An
association between two points (X,Y) in semantic space may be established either
synchronically or diachronically, either on the basis of a single language or
comparatively.\footnote{As a desideratum for the future, one can envision
three-dimensional semantic diagrams like those used to model molecules in
organic chemistry!}  I rely on three basic types of evidence:\footnote{See the
discussion in \textit{VSTB}: 194-200.}
\begin{enumerate}
\item
\textit{Synchronic intra-lingual vagueness.} A given daughter language has a single
form that means X or Y according to context, e.g.\ 
Mikir \textbf{artho} means either ‘blood vessel’ or ‘tendon’ or ‘muscle’ or ‘nerve’.
In many Chin languages reflexes of \textbf{*m‑luŋ} can mean either HEART or LIVER.
\item
\textit{Inter-lingual semantic shift of phonological cognates,} i.e.\ reflexes of the
same etymon mean X in Lg.~A but Y in Lg.~B, e.g.:\\
\hspace*{2ex}PTB \textbf{*r-kliŋ} ‘marrow/brain’ > Mikir \textbf{arkleŋ} ‘marrow’, Dimasa \textbf{buthluŋ} ‘brain’;\\
\hspace*{2ex}PTB \textbf{*s-pʷik} ‘bowels/stomach’ > Mikir \textbf{phek} ‘bowels’, Lahu \textbf{ɔ̀-\textit{fɨ́}-qō} ‘stomach’.
\item
\textit{Association via compounding.}  Three points (A,B,C) in semantic space are
related, such that in some language a compound of two morphemes, A + B, has the
meaning C.  In other words, an etymon appears as a constituent in compounds,
such that part of the meaning of the compound derives from it, e.g.:

FOOT + EYE $\to$ ANKLE (Lahu \textbf{khɨ-mɛ̂ʔ-šī} < \textbf{khɨ} ‘foot’ + \textbf{mɛ̂ʔ-šī} ‘eye’);
similarly Indonesian \textbf{mata-kaki} ‘ankle’ (< \textbf{mata} ‘eye’ + \textbf{kaki} ‘foot’), which
establishes the association EYE $\longleftrightarrow$ ANKLE\footnote{The same formation is
found in many other TB languages, e.g.:

\begin{tabular}{llll}
				&\textit{FOOT}	&\textit{EYE}	&\textit{ANKLE}\\
Lalung			&\textbf{ia-thong}	&\textbf{mi}	&\textbf{ia-thong-mi}\\
Limbu			&\textbf{lāŋ}	&\textbf{mik}	&\textbf{lāŋ-mik}\\
Lushai			&\textbf{ke}		&\textbf{mit}	&\textbf{ke-mit}\\
Meithei			&\textbf{khu}	&\textbf{mit}	&\textbf{khu-mit}\\
Tangkhul		&\textbf{phei}	&\textbf{mik-ra}	&\textbf{phei-mik-ra}\\
Written Burmese	&\textbf{khre}	&\textbf{myak-ci’}	&\textbf{khre-myak-ci’}\\
\end{tabular}}
\end{enumerate}

Certain conventions are observed in the metastatic flowcharts of this volume:

\begin{enumerate}
\item Points in semantic space between which an association has been established
are connected by solid lines.  If a point is a bodypart, it is labelled in
capital letters.  An association between two points that are both bodyparts is
an “intra-field association”, e.g.:

\hspace*{0.7in} \XeTeXpdffile "intro21a.pdf"

\item If the association crosses into another semantic field (i.e., with respect
to this volume, if it is between a bodypart and a non-bodypart), the
non-bodypart point is labeled with small letters, e.g.:

\hspace*{0.7in} \XeTeXpdffile "intro21b.pdf"

\item Antonymic associations (cases where the etymon has acquired opposite
meanings) are diagrammed by a curved \textit{yin-yang} type of line, e.g.:

\hspace*{0.7in} \XeTeXpdffile "intro21c.pdf"

\item Compounds are diagrammed by a pitchfork-like symbol, with the two
constituents appearing at the points of the fork, and the overall meaning of the
compound indicated at the tip of the handle, e.g.:

\hspace*{0.7in} \XeTeXpdffile "intro21d.pdf"

\end{enumerate}

The same convention with respect to capital vs.\ small letters applies to
compounds. In cases where several different combinations of morphemes are attested in compounds with the same meaning, graphic constraints sometimes require geometric reorientations of the pitchfork, e.g.

\hspace*{1in} \XeTeXpdffile "intro21d2.pdf"

The category of “reproductive bodyparts” is construed broadly to include
related verbs (e.g., KISS, SUCK, LOVE, SQUIRT). This volume also includes some
non-bodypart terms which frequently appear in compounds with etyma referring to
the reproductive system.  See especially Ch.~IX, “Body fluids”.


Deciding how much semantic latitude to allow among putative cognates is
definitely an art rather than a science. Here as elsewhere a middle-of-the-road
approach is necessary, neither overly conservative nor too wildly speculative.
As a positive example of a promising new etymology involving a semantic leap, we
may offer \textbf{*m-t(s)i} ‘salt~/ yeast’ [\textit{HPTB} 3.3.1].  Although forms in the daughter
languages sometimes mean ‘salt’ and sometimes ‘yeast’, the phonological
correspondences between both semantic groups of forms are good.  On the other
hand, the semantic association between ‘salt’ and ‘yeast’ has yet to be attested
in other language families, even though it has great initial plausibility. Both
are efficacious substances that have dramatic effects on the taste of food or
drink; their lack renders the food or drink insipid.\footnote{Yeast is used for
brewing liquor rather than for baking bread in East and SE Asia.}

\subsection{Reconstructed PTB etyma}


After the semantic chart which begins each chapter, the reconstructed PTB
roots of the chapter are presented one after the other, roughly in the order of
the strength of their attestation.  After preliminary remarks about the
distribution of the etymon, the “supporting forms” for the reconstruction are
listed, subgroup by subgroup.


The reconstructions all conform to the syllable canon posited for the
proto-language,\footnote{See \textit{HPTB}, pp.~11-13.}

\begin{quote}
(P²)\hspace{2em}(P¹)\hspace{2em}$\textrm{C}_i$\hspace{2em}(G)\hspace{2em}V\hspace{2em}(ː)\hspace{2em}($\textrm{C}_f$)\hspace{2em}(s),
\end{quote}


where the initial consonant ($\textrm{C}_i$) may be preceded by up to two prefixes (with the
inner prefix P¹ assumed to be historically prior to the outer one (P²); the $\textrm{C}_i$
may optionally be followed by one of four glides (G), */-y-, -r-, -w-, -l-/, and
the vowel, which may be long (ː), may be followed by a final consonant ($\textrm{C}_f$); if
the syllable does contain a $\textrm{C}_f$, it may also (although quite rarely) end with
suffixal -s.  It should be noted that many daughter TB languages have much
simpler canons, e.g.\ Lahu, where native syllables consist maximally of an
initial consonant, a vowel, and a tone:


\begin{quote}
\begin{tabular}{ll}
	&T\\
($\textrm{C}_i$)	&V\\
\end{tabular}
\end{quote}

No attempt is made to reconstruct tones beyond the subgroup level, since it is
far from proven that a single system of tonal contrasts can be set up for PTB.  


Reconstructions at the subgroup level (i.e.\ “meso-reconstructions” like
Proto-Lolo-Burmese (PLB), Proto-Northern-Naga (PNN), Proto-Tani) are listed as
individual records along with their supporting forms.


A few notational conventions with respect to my PTB reconstructions should
be mentioned:
\begin{itemize}
\item Variant reconstructed forms are indicated in several ways.  They are usually
\mbox{written} with the “allofam symbol” \STEDTU{⪤} between them, e.g.: \textbf{*glim} \STEDTU{⪤} \textbf{*glip}
BROOD~/ INCUBATE; \textbf{*s-riŋ} \STEDTU{⪤} \textbf{*s-r(y)aŋ} LIVE / ALIVE / GREEN / RAW / GIVE BIRTH.
Sometimes, however, I use an alternative notation with parentheses, e.g.: \textbf{*(t)si}
COPULATE/LOVE; this is equivalent to \textbf{*si} \STEDTU{⪤} \textbf{*tsi}.  Slashes may also be used,
e.g.\ \textbf{*p/buk} \STEDTU{⪤} \textbf{*p/bik} BORN/GIVE BIRTH; this is equivalent to
\textbf{*puk} \STEDTU{⪤} \textbf{*buk} \STEDTU{⪤} \textbf{*pik} \STEDTU{⪤} \textbf{*bik}.
Finally, still another way of indicating proto-variation is by means of a
“vertical reconstruction”, e.g.:

\begin{quote}
*\textbf{\begin{tabular}[c]{c}t\\d\\\end{tabular}uŋ}  NAVEL.
This means the same as \textbf{*tuŋ} \STEDTU{⪤} \textbf{*duŋ.}
\end{quote}


\item Parentheses are especially appropriate for those frequent cases where there is
variation or indeterminacy between dental and palatal fricates; in fact that is
one of my principal motivations for writing the palatal series as sequences of
dental plus \textbf{-y-},
rather than writing them with \textit{hačeks} or grave accents, e.g.:

\begin{quote}
\begin{tabular}{lll}
\textbf{*ts(y)uːŋ}	&NAVEL / CENTER		&(= \textbf{*tsuːŋ} \STEDTU{⪤} \textbf{*tšuːŋ)}\\
\textbf{*s(y)ok}	&DRINK / SUCK / SMOKE	&(= \textbf{*sok} \STEDTU{⪤} \textbf{*šok)}\\
\end{tabular}
\end{quote}

\item Etyma which show variation between initial \textbf{*p-} and \textbf{*w-} are reconstructed with
the morphophonemic symbol \textbf{*pʷ-,} which is roughly equivalent to treating the stop
element as a prefix (\textbf{*p-w-}).\footnote{For extended discussion of this issue,
see Matisoff 2000a.}  Thus, a reconstruction like \textbf{*pʷu} EGG~/ BIRD~/ ROUND OBJECT
implies the existence of two sub-roots, \textbf{*pu} and \textbf{*wu,} whatever the ultimate
explanation for this variation might prove to be.

\item In the original version of Benedict 1972 (henceforth \textit{STC}, ca.~1943), Benedict reconstructed two PTB high
long vowels \textbf{*-iy} and \textbf{*-uw,} contrasting with the much less frequent short high
vowels \textbf{*-i} and \textbf{*-u.}  In the published version (1972) he modified the
reconstruction of these long vowels to \textbf{*-əy} and \textbf{*-əw,} a practice which I follow
myself.  Occasionally, however, when the evidence does not permit us to decide
between a long and a short high proto-vowel, it is convenient to revert to the
earlier notation, with parentheses, e.g.\ \textbf{*b-ni(y)} ‘petticoat’ (\textit{STC} \#476);
\textbf{*sru(w)} ‘relative’ (\textit{STC} p.~108).  There are no such cases among the etyma in
this volume, however.
\end{itemize}


For more discussion of variational patterns in PTB, see “Regularity and
variation”, section 3.1 below.


Many of the PTB etyma in this volume are here reconstructed for the first
time in print, and a good number of the TB/Chinese comparanda are likewise here
proposed for the first time.  If references are not explicitly given to \textit{STC}
and/or \textit{HPTB} in the introductory note for an etymon it may be assumed that the
reconstruction is new.\footnote{References to HPTB as labelled with “\textbf{H:}” followed by a page number, e.g.\ \textbf{(H:165)} \textbf{*wa} \STEDTU{⪤} \textbf{*wu} BIRD / FOWL means that the root is discussed chiefly on page 165 of \textit{HPTB}.}


\subsection{Subgroup names}


Tibeto-Burman is an extremely complex language family, with great internal
typological diversity, comparable to that of modern Indo-European.  This
diversity is due largely to millennia of language contact, especially with the
prestigious cultures of India and China,\footnote{I have called the Indian and
Chinese areas of linguistic and cultural influence the \textit{Indosphere} and the
\textit{Sinosphere}. See Matisoff 1973.} but also with the other
great language families of Southeast Asia (Austroasiatic, Tai-Kadai,
Hmong-Mien), as well as with other TB groups.  We are thus faced with what I
have described as “an interlocking network of fuzzy-edged clots of languages,
emitting waves of mutual influence from their various nuclear ganglia.  A mess,
in other words.”\footnote{Matisoff 1978 (\textit{VSTB}), p.~2.}  While subgrouping such
a recalcitrant family is difficult, there is certainly no need to go so far as
van Driem by denying that TB subgroups exist at all, or by claiming that even if
they do exist, there are so many of them that there is no point in talking about
them!\footnote{See his review (2003) of G. Thurgood \& R.J. LaPolla, eds. (2003),
\textit{The Sino-Tibetan Languages}.}


In the published version of \textit{STC} (1972),
P.\ K.\ Benedict wisely refrained from offering a pseudo-precise family-tree model of
the higher-order taxonomic relationships in TB, presenting instead a schematic
chart where Kachin (=~Jingpho) was conceived as the center of geographical and
linguistic diversity in the family.  See Fig.~1.

\begin{figure}[ht]
\XeTeXpdffile "intro231.pdf"  width \textwidth
\begin{center}
\textit{Figure 1. Schematic Chart of Sino-Tibetan Languages}\footnotemark
\end{center}
\end{figure}
\footnotetext{Reproduced from \textit{STC}, p.~6;
\textit{VSTB}, p.~3; \textit{HPTB}, p.~4.}

A simpler scheme represents the heuristic model now used at STEDT.  See Fig.~2.

\begin{figure}[ht]
\XeTeXpdffile "intro232.pdf"  width \textwidth
\begin{center}
\textit{Figure 2.  Simplified STEDT Family Tree of ST Languages}
\end{center}
\end{figure}

This diagram differs from \textit{STC} in several respects:\footnote{See \textit{HPTB}, pp.~5-6.}
\begin{itemize}
\item Karenic is no longer regarded as having a special status, but is now
considered to be a subgroup of TB proper.
\item Baic, hardly mentioned (under the name “Minchia”) in \textit{STC}, but later
hypothesized by Benedict to belong with Chinese in the “Sinitic” branch of 
Sino-Tibetan, is now also treated as just another subgroup of TB, though one
under particularly heavy Chinese contact influence. Both Karenic and Baic have
SVO word order, unlike the rest of the TB family.
\item The highly ramified Kuki-Chin and Naga groups have provisionally been
amalgamated with Bodo-Garo (=Barish) and Abor-Miri-Dafla (=Mirish) into a
supergroup called by the purely geographical name of \textit{Kamarupan}, from the old
Sanskrit name for Assam.
\item The important Tangut-Qiang languages (deemed to include rGyalrong
[=Gyarung =Jiarong] and the extinct Xixia [=Tangut]) were hardly known to Western
scholars at the time \textit{STC} was written (ca.~1942-3) or published (1972). It seems
doubtful that a special relationship exists between Qiangic and Jingpho, or
between Qiangic and Lolo-Burmese, as some Chinese scholars maintain.\footnote{A
supergroup called “Rung” was proposed by Thurgood (1984), into which he placed,
among others, some Qiangic languages, Nungish, and Lepcha.  This grouping was
based partly on shared “proto-morphosyntax”, and partly on nomenclature,
including the \textit{-rong} of \textit{rGyalrong},
the Nungish language \textit{Rawang}, and the Lepcha autonym \textit{Rong}.}
\item The Nungish and Luish languages are grouped with Jingpho (=Kachin).  Jingpho
is also recognized to have a special contact relationship with the Northern Naga
(=Konyak) group.\footnote{The \textit{Linguistic Survey of India}
(Grierson and Konow, 1903-28) recognized a “Bodo-Naga-Kachin” group,
an idea revived by Burling
(1983), whose “Sal” supergroup comprises Bodo-Garo (Barish), Northern Naga
(Konyak), and Jingpho (=Kachin).  Burling’s name for this grouping is derived
from the etymon \textbf{*sal} ‘sun’ (ult.\ < PTB \textbf{*tsyar} ‘sunshine’), one of a number of
roots which is attested chiefly in these languages.  See \textit{HPTB}:393-4.}
\item The somewhat idiosyncratic Mikir, Meithei (=Manipuri), and Mru languages are
included under Kamarupan.
\item The Himalayish (=Himalayan) group is considered to include Bodic (i.e.\
\mbox{Tibetanoid}) languages, as well as Kanauri-Manchad, Tamang-Gurung-Thakali,
Kiranti (=Rai), Lepcha, and Newar.
\item The relatively well-studied Lolo-Burmese group (= \textit{STC}’s “Burmese-Lolo”) is
deemed to include the aberrant Jinuo language of Xishuangbanna,
Yunnan.\footnote{Chinese scholars have further divided the Loloish languages of
China into six nuclei, although no attempt is made in this volume to distinguish
them.  In a recent talk (Matisoff 2007b) I examined Loloish tonal developments
and the fate of the PLB rhyme \textbf{*-a} in terms of this six-way grouping, with
inconclusive results.}  The Naxi/Moso language is quite close to LB, but stands
somewhat outside the core of the family.\footnote{I have grouped Naxi with
Lolo-Burmese proper in a supergroup called “Burmo-Naxi-Lolo” (Matisoff 1991c). 
On the basis of some shared tonal developments, I have also entertained the idea
of a special relationship between Lolo-Burmese and Jingpho, to which I assigned
the jocular designation \textit{Jiburish} (<~\textbf{Ji-}(ngpho)
+ \textbf{-bur}(mish) + (Lolo)\textbf{ish}).  See
Matisoff 1974, 1991c.}
\item The mysterious Tujia language of Hunan and Hubei (not mentioned in \textit{STC}) has so
far not been assigned to a subgroup.
\end{itemize}


Still, a schema like Fig.~2 hardly does justice to the complexity
of the problem of subgrouping the TB languages.  In particular, the “Kamarupan”
and “Himalayish” groupings are based more on geographical convenience than on
strong constellations of similar characteristics.\footnote{Several scholars have objected to the term Kamarupan,
largely on the grounds that it has distinctly Indo-Aryan connotations, which
might irritate TB groups.  See, e.g.\ R.~Burling, “On \textit{Kamarupan}” (1999; \textit{LTBA}
22.2:169-71), and the reply by Matisoff, “In defense of \textit{Kamarupan}”
(1999; \textit{LTBA} 22.2:173-82).  The only alternative term
suggested so far to refer to these
geographically contiguous languages collectively is the verbose “TB languages of
Northeast India and adjacent areas”.}
More detailed subgroupings are certainly possible, as in
STEDT Monograph \#2,\footnote{J.~Namkung, ed. (1996),
\textit{Phonological Inventories of Tibeto-Burman Languages},
pp.~455-457.}
which makes distinctions like the following:


\begin{quote}
\textit{Kamarupan}\\
- Abor-Miri-Dafla (=Mirish)\footnote{A well-defined subgroup of AMD has been
dubbed \textit{Tani} by J. Sun (1993).}\\
- Kuki-Chin\\
- Naga\\
\hspace*{3ex}· Konyak (=Northern Naga)\\
\hspace*{3ex}· Angamoid\\
\hspace*{3ex}· Central\\
\hspace*{3ex}· Eastern\\
\hspace*{3ex}· Southern\\
\hspace*{3ex}· Southwestern\\
- Meithei\\
- Mikir\\
- Mru\\
- Bodo-Garo (=Barish)\\
- Chairel

\textit{Himalayish}\\
- Western (Bunan, Kanauri, Manchad/Pattani)\\
- Bodic (Tibetanoid)\\
- Lepcha\\
- Tamangic (incl. Chantyal, Gurung, Tamang, Thakali, Manang, Narphu)\\
- Dhimalish\\
- Newar\\
- Central Nepal Group (Kham, Magar, Chepang, Sunwar)\\
- Kiranti (=Rai), including Bahing and Hayu\\
\end{quote}


\subsection{Language names}


Tibeto-Burman languages are notorious for the multiplicity of names by which
they are referred to. These may include the name they use for themselves
(autonym), as opposed to the name(s) other groups use for them (exonyms). 
Languages are frequently referred to by the principal town in which they are
spoken (loconyms).  Some exonyms are now felt to be pejorative, and have been
abandoned, thus acquiring the status of “paleonyms” for which “neonyms” have
been substituted.\footnote{The terminology for the various types of TB language
names was developed in Matisoff 1986a: “The languages and dialects of
Tibeto-Burman: an alphabetic/genetic listing, with some prefatory remarks on
ethnonymic and glossonymic complications.”  In John McCoy and Timothy Light,
eds., \textit{Contributions to Sino-Tibetan Studies},  pp.~1-75.  This article was later
(1996) expanded into a STEDT Monograph, with the assistance of J.B. Lowe and
S.\ P.\ Baron.}  A certain Angamoid Naga group call themselves and their language
\textit{Memi} (autonym), and their chief village they call \textit{Sopvoma};
but other groups use
\textit{Mao} for this village or its people (exonym), and either \textit{Mao}
or \textit{Sopvoma} (exonymic
loconym) for their language.  There is an older term \textit{Imemai} (probably an
autonymic paleonym) which refers to the same language and people.


Some names are used in both a broader and a narrower sense, both for a
specific language and for a group of languages that share a close contact
relationship.  The Maru, Atsi, and Lashi\footnote{Referred to as Langsu, Zaiwa,
and Leqi in Chinese sources.} (who speak Burmish languages) consider themselves
to be “Kachin” in the broad sense, and in this the Jingpho themselves seem to
agree, even though the Jingpho language belongs to a different TB subgroup.


In recent years cultural sensitivities have forced the abandonment of many
language names that had been well established in the academic literature.  The
important Central Chin language that used to be called \textit{Lushai} (a name which is
said to mean “long-headed”) should now properly be called \textit{Mizo}.
A Karenic group
that used to be known by the Burmese exonym \textit{Taungthu} (lit.\ “mountain folk”) now
prefers to be referred to by their autonym \textit{Pa-o}.  The Southern Loloish people
formerly known by the Tai exonym \textit{Phunoi} (lit.\ “little people”) should now be
called by their autonym \textit{Coong}.  Speakers of several TB languages of Nepal now
object to the Indianized versions of their names with the Indo-Aryan \textbf{-i} suffix
(e.g.\ \textit{Newari, Magari, Sunwari}), and prefer to omit the suffix, even though this
can lead to ambiguity between the names of the people and their languages
(\textit{Newar, Magar, Sunwar}).  The psychological dimensions of these issues are often
as fascinating as they are paradoxical.  Chinese linguists now feel that the
term \textit{Lolo(ish)}, widely used outside of China,
is offensive, and insist that the
proper respectful term is \textit{Yi}, written with the character \TC{彝} ‘type of
sacrificial wine vessel’.  Yet this is only a recent substitution for the
homophonous character \TC{夷} ‘barbarian; savage group on the fringes of the Chinese
empire’.


Naturally enough, what is true for the names of individual languages is also
true for the names of subgroups.  Some of this nomenclatural variation goes back
to differences between Benedict and his former collaborator and supervisor
Robert Shafer,\footnote{Shafer and Benedict collaborated on the Depression-era
\textit{Sino-Tibetan Linguistics} project at Berkeley (1939-40), which aimed to assemble
all data then available on TB languages.  The direct fruits of this project were
Shafer’s \textit{Introduction to Sino-Tibetan} (1967-73), 5 vols. (Wiesbaden: Otto
Harrassowitz) and the MS of Benedict’s \textit{STC}.  Benedict produced (1975) an
entertaining account of this seminal project in LTBA 2.1:81-92: “Where it all
began: memories of Robert Shafer and the \textit{Sino-Tibetan Linguistics} project,
Berkeley (1939-40).”} e.g.\ Shafer’s \textit{Barish} and \textit{Mirish} are the same as Benedict’s
\textit{Bodo-Garo} and \textit{Abor-Miri-Dafla}, respectively.  An important group of at least a
dozen TB languages spoken in East Nepal is known either as \textit{Kiranti} or
\textit{Rai}.\footnote{According to K.~P.~Malla (p.c.~2008), “\textit{Kirãt} is a loose label in Old Indo-Aryan for the cave-dweller, attested in late Vedic texts as well as in the \textit{Mahābhārata}.” Rai is “a Nepali word, linked to IA \textit{raaya} ‘lord’, given to the Khambu chiefs by the Gorkhali rulers in the late 18th century.”}
  An extreme example of proliferation is furnished by the well-established and
non-controversial group I call Lolo-Burmese, which has also been referred to as
Burmese-Lolo, Yi-Burmese, Burmese-Yi, Burmese-Yipho, Yipho-Burmese, Yi-Myanmar,
Myanmar-Yipho, etc.—and even Myanmar-Ngwi!


Bearing all these complicating factors in mind, an attempt has been made in
this volume to use maximally clear and consistent designations for the TB
languages and subgroups.

\subsection{Supporting forms in the individual languages}


The forms which support the reconstructions are cited according to the
notation of the particular source.  Although this policy of “following copy”
often leads to redundancy (see 2.7 below), since one and the same form in a
given language may be transcribed in a variety of different ways,\footnote{Cf.\ the
multiple transcriptions of the Written Burmese form for BREAST/MILK under
\textbf{*s-nəw}, \#53 below:  \textbf{no¹}
(ZMYYC:281.39); \textbf{nuí} (AW-TBT:327; \textit{STC}:419); \textbf{núi}
(WSC-SH:48); \textbf{nuiʼ} (JAM-Ety; GEM-CNL; PKB-WBRD);
\textbf{nui.} (GEM-CNL); \textbf{nuiwʼ} (GHL-PPB).
 For these source abbreviations, see the \textit{Appendix}. Similarly, cf.\ the many slightly different forms from the Bodic and Tamangic groups that reflect the etymon \textbf{*tsaŋ} NEST/WOMB/PLACENTA (\#103 below).} it seems preferable to a
policy of “normalization”, which might have the effect of losing some phonetic
detail that is captured in one source but not in another.

\subsection{Glosses of the supporting forms}


In almost all cases, the gloss given in each particular source is preserved,
unless it is so awkward or misleading as to require emendation.  Even if the
glosses in consecutive records are identical, the gloss is repeated for each
individual record, instead of using a symbol like the “ditto-mark”.  


If a gloss is too long to fit onto a single line, it is “wrapped” so that
the additional lines are indented under the first one.

\subsection{Source abbreviations}


Each supporting form is ascribed to a particular source.  Many forms are
cited in more than one source in our database.  If the form is not identical in
different sources, we include them all.  This is especially useful in cases
where one or more of the sources might not be totally accurate phonemically, or
where subphonemic phonetic detail is provided. When the forms in different
sources are identical, the form only appears once, but there are multiple source
abbreviations, separated by commas.  Forms from well-studied languages
(e.g.\ Written Tibetan, Written Burmese, Jingpho) are likely to appear in several
sources used by STEDT.


The STEDT database contains forms from sources of many different kinds,
including:
\begin{itemize}
\item printed books, monographs, articles, especially dictionaries and grammars of
individual languages;
\item synonym lists (i.e.\ groups of forms from different languages with the same
meaning, but with no reconstructions provided), e.g.\ Luce 1986 (PPPB); Sun
Hongkai et al.\ 1991 (ZMYYC); Dai Qingxia, Huang Bufan et al.\ 1992 (TBL);
\item semantically based questionnaires solicited by STEDT from fieldworkers working
on particular languages;
\item monographs and treatises of an etymological nature, including works which
provide reconstructions at the subgroup level, e.g.:

\begin{verse}
Proto-Bodo: Burling 1959\\
Proto-Karenic: Haudricourt 1942-45/1975, Jones 1961, Burling 1969, Benedict~1972 (\textit{STC}), Solnit, in prep.\\
Proto-Kiranti: Michailovsky 1991\\
Proto-Kuki-Chin: VanBik 2003\\
Proto-Lolo-Burmese: Burling 1968, Matisoff 1969/1972, Bradley 1979\\
Proto-Northern-Naga: French 1983\\
Proto-Tamangic: Mazaudon 1978\\
Proto-Tani: Sun Tianshin 1993\\
\end{verse}
\end{itemize}


The abbreviations used in these source attributions are in general quite
transparent,\footnote{For a complete list of the source abbreviations that
appear in this volume, see the \textit{Appendix}.} e.g.:

\begin{quote}
\begin{tabular}{ll}
CK-YiQ	&Chen Kang, “Yi Questionnaire”\\
JZ-Zaiwa	&Xu Xijian, \textit{Outline Grammar (Jiǎnzhì) of Zaiwa}\\
AW-TBT	&A.~Weidert, \textit{Tibeto-Burman Tonology}\\
GHL-PPB	&G.~H.~Luce, \textit{Phases of Pre-Pagán Burma}\\
JAM:MLBM	&J.~A.~Matisoff, “Mpi and Lolo-Burmese microlinguistics”\\
EJAH:BKD	&E.~J.~A.~Henderson, \textit{Bwo Karen Dictionary}\\
\end{tabular}
\end{quote}

The abbreviation “JAM-Ety” refers to my own etymological notes compiled in the
pre-STEDT era, derived especially from older, classic sources.  These specific
sources can easily be tracked down from the \textit{Bibliography}.

\subsection{Chinese comparanda}

After the evidence for a TB etymon is presented, one or more Chinese
comparanda are often suggested in the interests of pushing the reconstruction
further back to the Proto-Sino-Tibetan stage.  For all of these comparanda
Zev J.\ Handel has kindly provided comparisons of the Old Chinese reconstructions cited
in Karlgren’s (1957) system with those of Li Fang-kuei (1971, 1976, 1980) and
William Baxter (1992),\footnote{Handel also contributed a detailed comparison
of these systems in his \textit{A Concise Introduction to Old Chinese Phonology},
which appeared as Appendix A to \textit{HPTB}, pp.~543-74.}
evaluating the plausibility of the
putative TB/Chinese comparison according to each of these systems.\footnote{Handel  also frequently refers to several other reconstructive systems for OC that are to be found
in the literature, e.g.\ those of W. South Coblin (1986), Axel Schuessler (1987),
Laurent Sagart (1999), Gong Hwang-cherng (1990, 1994, 1995, 1997, 2000), and Pan
Wuyun (2000).}   Handel’s invaluable contributions are marked with his initials “ZJH”. 
Comparisons between TB and OC etyma that are not explicitly ascribed to a
particular scholar are original with me, as far as I know.

\subsection{Notes}

Footnotes may appear at virtually any point in the text. They may refer to an entire chapter, to a semantic diagram, to an etymon as a whole, to a specific supporting form, or to a Chinese comparandum.

\section{Theoretical issues}


Implicit in the reconstructions of this volume are my positions on certain
theoretical issues.

\subsection{Regularity and variation}


It must be admitted that a lot of guesswork is involved in etymologizing
material from hundreds of languages and dialects at once, without having
established the “sound laws” in advance. The problems are especially acute when
comparing phonologically depleted languages with those having richer syllable
canons. When there is a partial phonological similarity between distinct etyma
with the same meaning (e.g.\ \textbf{*sem} and \textbf{*sak}  ‘mind / breath’;
\textbf{*muːr} and \textbf{*muk}
‘mouth’; \textbf{*s-maːy} and \textbf{*s-mel} ‘face’;
\textbf{*s-r(y)ik} and \textbf{*s(y)ar} ‘louse’), it is not
easy to decide by simple inspection to which etymon we should assign a
phonologically slight form in a daughter language (e.g.\ \textbf{sɒ} ‘mind’,
\textbf{mɔ} ‘mouth’,
\textbf{hmɛ} ‘face’).


There is a dialectical relationship between synchronic data and sound laws.
The “laws” are derived by inference from the data in the first place, but once
proto-forms are reconstructed, they can be used to guide us in our hunt for
cognates in languages not yet examined (even if they have undergone semantic
change). Almost every TB/ST etymology so far proposed presents problems and
complications—irregularities—in some language or other, which is par for
the course even in the much better known Indo-European family. Part of our task
is to indicate where the exceptions, problems, and irregularities lie, in the
hope that they can ultimately be explained.\footnote{The computer can be very
useful in deciding between alternative etymologies. Once “sound-laws” have been
formulated, computer checking can test whether a particular reconstruction
follows the laws, identifying inconsistencies in the reflexes of the same
proto-element in a given language. Such a methodology has been applied to the
Tamangic languages, using the “reconstruction engine” developed by J.B. Lowe at
STEDT in collaboration with Martine Mazaudon and Boyd Michailovsky during their
sojourns at Berkeley as visiting scholars (1987-89, 1990-91).} The concept of
“regularity” itself is by no means simple, nor does it mean the same thing to
different scholars.\footnote{See Matisoff 1992 (“Following the marrow”) and
1994a (“Regularity and variation”).}


Those who lack what I have called “Proto-Sprachgefühl”\footnote{See
Matisoff 1982.} can produce abstract, formulaic reconstructions bristling with
strange symbols but devoid of any phonetic or typological
plausibility.\footnote{Recent examples of this genre include Sedláček
1970; Weidert 1975, 1979, 1981, 1987; Peiros \& Starostin 1996; Sagart 2007.} 
Given sufficient semantic latitude and proto-forms that are complex enough, one
can formulate “sound laws” in such a way that they appear completely regular and
exceptionless.  At an extreme level we find “megalocomparative” proposals of
genetic relationship that turn received notions upside down (e.g.\ Sino-Mayan,
Sino-Caucasian, Sino-Austronesian, Japanese-Dravidian), and which can lead the
unwary down fruitless paths, obscuring the differences among cognates,
borrowings, and chance resemblances.\footnote{See Matisoff 1990a (“On
megalocomparison”).  Megalocomparison has the apparent advantage of
non-falsifiability, since, as Haudricourt has observed, one can never prove that
any two languages are not related. But non-falsifiable hypotheses are not
scientific. When presented with alternative non-falsifiable proposals it is
impossible to choose among them.}  Various tricks of analysis that I have lumped
under the rubric of “proto-form stuffing” can help the Nostraticist or
Sino-Mayanist convince himself that his fantastical comparisons are “perfectly
regular”. Paradigmatically, one can multiply the number of proto-phonemes. If
you reconstruct 35 proto-vowels, any anomalous vowel correspondence can be
regarded as “regularly reflecting” a separate proto-vowel.  Syntagmatically, if
you reconstruct etyma like \textbf{*mrgsla,} and the monstrous proto-cluster
\textbf{*mrgsl-}
occurs only in a single etymon, any set of reflexes in the daughter languages
can be said to be “regular”.\footnote{This is actually the proto-form offered
in Weidert 1981:25 for an etymon meaning ‘spirit, ghost, shadow’ (reconstructed
as \textbf{*m-hla} in \textit{STC} \#475). As I have observed (Matisoff 1982:22), “It is always
possible and sometimes necessary to invent an \textit{ad hoc} explanation for an
anomalous case. It is even true that some such \textit{ad hoc} ‘solutions’ are more
plausible than others. The only harm is in deluding oneself that an explanation
which covers only a single case establishes a ‘regularity’.”}


The time-depth of PST is perhaps 6000 years B.P., about at the limits of the
comparative method. We can hardly afford to insist on “perfect regularity” of
correspondence among our putative cognates.  But instead of resorting to
“proto-form stuffing” to try to explain away problems, what is needed is an
explicit theory of variational phenomena.  TB and ST etyma, like those of other
language families, are not independent isolated entities, but stand in complex
phonosemantic relationships with each other.  It has been recognized for a long
time that words in Chinese and TB languages participate in morphophonemic groups
of partially resemblant forms that have been called “word families”.\footnote{See the pioneering study of Karlgren (1933), “Word families in Chinese”.}  In
Matisoff 1978 (\textit{VSTB}) I developed the notion of the \textit{allofam},
or individual member
of a word-family, and advocated the formulation of “allofamic reconstructions”
that accommodated all the well-attested variants deemed to descend from the same
proto-word-family.  The symbol \STEDTU{⪤} was introduced to symbolize an allofamic
relationship between variant forms, i.e., “A \STEDTU{⪤} B” means that “A and B are
synchronic allofams of each other”, while “*A \STEDTU{⪤} *B” means that there is a
word-family relationship between A and B at the proto-level.\footnote{This
symbol \STEDTU{⪤}, a combination of  >  ‘goes to’ and < ‘comes from’, is meant to suggest
that neither variant is necessarily deemed to have temporal priority, but that
both must be set up to account for attested forms.}


Needless to say, extreme care must be used in claiming that different forms
are variants of the same etymon.  Allofamic theory must be applied in a
controlled and constrained way.\footnote{See the extended discussion in Ch.~XII
of \textit{HPTB} (pp.~491-534), “Allofamic variation in  rhymes”.} Not everything may be
said to vary with everything else! It is sometimes quite difficult to decide
whether partially resemblant forms represent separate etyma or whether they are
merely allofams of the same word-family.  Not only must each proto-allofam fit
our canonic template (above 2.2), but the type of variation posited must be
abundantly replicated in other examples.  This volume does not attempt to
conceal such uncertainties, but frequently entertains the possibility that etyma
set up as independent might actually be co-allofams, or \textit{vice versa}. 

The best attested patterns of variation in ST/TB are all exemplified in the
etymologies of this volume.  They include the following:

\begin{enumerate}
\item %(a)
Voicing vs.\ voicelessness of the initial consonant:\footnote{Nothing is
more common in TB word families than variation of voicing in initial consonants,
largely due to the pervasive influence of prefixes on the manner of the initial.
 This is in sharp contrast to the situation in Indo-European, where such
variation in manner is quite rare, and is usually not tolerated in PIE
reconstructions.}\nopagebreak[4]

\begin{tabular}{lll}
\textbf{*gop} \STEDTU{⪤} \textbf{*kop}	&(11a)	&HATCH/INCUBATE/COVER\\
\textbf{*prat} \STEDTU{⪤} \textbf{*brat}	&(75)	&BREAK/WEAN\\
\textbf{*tuŋ} \STEDTU{⪤} \textbf{*duŋ}	&(44a)	&NAVEL\\
\end{tabular}

\item %(b)
Variation between fricative and affricate:

\begin{tabular}{lll}
\textbf{*(t)sum}	&(45)		&NAVEL\\
\textbf{*(t)sip} \STEDTU{⪤} \textbf{*(t)sup} &(107)	&NEST/WOMB/SCROTUM\footnotemark\\
\end{tabular}
\footnotetext{This etymon also illustrates (f), below.}

\item %(c)
Presence vs.\ absence of medial \textbf{-y-}

\begin{tabular}{lll}
\textbf{*b(y)at}	&(81)		&VAGINA\\
\textbf{*l(y)ap}	&(151)		&COPULATE\\
\end{tabular}

A special case of (c) is the alternation between dental and palatal fricatives
and affricates:

\begin{tabular}{lll}
\textbf{*s(y)ok}	&(61)		&DRINK/SUCK/SMOKE\\
\textbf{*dz(y)əw}	&(56)		&BREAST/MILK\\
\textbf{*ts(y)uːŋ}	&(44b)		&NAVEL\\
\end{tabular}

\item %(d)
Variation between labial stop and labial semivowel:

\begin{tabular}{lll}
\textbf{*pu} \STEDTU{⪤} \textbf{*wu} 	&(1a, 1b)		&EGG\\
\textbf{*pam} \STEDTU{⪤} \textbf{*wam}	&(98a, 98b)	&WOMB/PLACENTA/NEST\\
\end{tabular}

\item %(e)
Variation between different prefixes:

\begin{tabular}{lll}
\textbf{*r-ga} \STEDTU{⪤} \textbf{*N-ga} \STEDTU{⪤} \textbf{*d-ga} \STEDTU{⪤} \textbf{*s-ga} &(141)	&COPULATE/LOVE/WANT\\
\textbf{*n-tow} \STEDTU{⪤} \textbf{*s-tow} &(3)		&EGG\\
\textbf{*m-ŋal} \STEDTU{⪤} \textbf{*l-ŋal} &(100)		&WOMB/PLACENTA\\
\end{tabular}

\item %(f)
Variation between \textbf{-u-} and \textbf{-i-} in closed syllables:

\begin{tabular}{lll}
\textbf{*dul} \STEDTU{⪤} \textbf{*dil} 		&(2b)	&EGG/TESTICLE\\
\textbf{*m-dzup} \STEDTU{⪤} \textbf{*m-dzip}	&(55)	&SUCK/SUCKLE/MILK/KISS\\
\textbf{*tsyur} \STEDTU{⪤} \textbf{*tsyir}	&(66)	&MILK/SQUEEZE/WRING\\
\end{tabular}

\item %(g)
Variation between medial \textbf{-ya-} and \textbf{-i-}:

\begin{tabular}{lll}
\textbf{*s-riŋ} \STEDTU{⪤} \textbf{*s-ryaŋ} &(39)	&LIVE/ALIVE/GREEN/RAW/GIVE BIRTH\\
\textbf{*s-nik} \STEDTU{⪤} \textbf{*s-nyak} &(124)	&PENIS/COPULATE\\
\textbf{*b-rim} \STEDTU{⪤} \textbf{*b-ryam} &(46)	&NAVEL/UMBILICAL CORD\\
\end{tabular}

\item %(h)
Alternation between medial \textbf{-wa-} and \textbf{-u-}:

\begin{tabular}{lll}
\textbf{*tsyul} \STEDTU{⪤} \textbf{*tsywal} &(105)	&WOMB/PLACENTA\\
\end{tabular}

\item %(i)
Alternation between final homorganic stops and nasals:\nopagebreak[4]

\begin{tabular}{lll}
\textbf{*glim} \STEDTU{⪤} \textbf{*glip} 	&(15)	&BROOD/INCUBATE EGGS\\
\textbf{*s-nəwn} \STEDTU{⪤} \textbf{*s-nəwt} &(53c)	&BREAST/MILK/SUCK\\
\textbf{*tsiŋ} \STEDTU{⪤} \textbf{*tsik} &(78)	 	&VAGINA\\
\end{tabular}
\end{enumerate}


As some of the above examples illustrate, some roots show more than one type
of variation.  When a posited allofamic reconstruction (e.g.\ \textbf{*sir} \STEDTU{⪤} \textbf{*sit} (6)
EGG) does not fall into a well-attested variational category, I comment on it. 
Handel makes similar remarks with respect to some of my TB comparisons with OC.


Occasionally, when the phonosemantic variation among the allofams is
considerable, and when each variant is amply attested, I split up the
presentation of the data into subroots that are designated by the same number
but with different lower case letters, e.g.: \textbf{*p-wu} (1) EGG is split into \textbf{*wu}
(1a) and \textbf{*pu} (1b);
\textbf{*m/s-la(ː)y} \STEDTU{⪤} \textbf{*s-tay} (40) NAVEL/CENTER/SELF is split into
\textbf{*m/s-la(ː)y} (40a) and \textbf{*s-tay} (40b);
\textbf{*m-ley} \STEDTU{⪤} \textbf{*m-li} \STEDTU{⪤} \textbf{*m-ney} (114) PENIS is
broken down into \textbf{*m-ley} \STEDTU{⪤} \textbf{*m-li} (114a)
and \textbf{*m-ney} (114b).


As I put it 35 years ago, “We must steer an Aristotelian middle path between
a dangerous speculativism and a stodgy insensitivity to the workings of
variational phenomena in language history.”\footnote{Matisoff 1972b (“Tangkhul
Naga and comparative TB”), p.~282.}




\subsection[Etymological accuracy and rectification of possible errors]{Etymological accuracy and rectification of possible errors}

There are all too many ways in which one can make etymological mistakes, and
I have been guilty of all of them at one time or another.\footnotemark A rough taxonomy of
errors would have to include the following:

\footnotetext{The discussion in this section is adapted from \textit{HPTB}, pp.~538-40.}

\begin{itemize}
\item Treating a loanword as native

I was at first delighted when I ran across the Jingpho form \textbf{wéʔ-wū} ‘screw’,
since its first syllable looked like an excellent match with Lahu \textbf{ɔ̀-vɛ̀ʔ} ‘id.’,
for which I then had no etymology. Could this be a precious example of the rare
PTB rhyme \textbf{*-ek}? But the screw is hardly an artifact of any great antiquity, and
it would be \textit{prima facie} implausible that a root with such a meaning would have
existed in PTB. The truth quickly became apparent. The modern Burmese form for
‘screw’, \textbf{wɛ́ʔ-ʔu} (WB \textbf{wak-ʔu}), the obvious source from which both Jingpho and
Lahu borrowed these words, means literally “pig-intestine”.  The semantic
association is the corkscrew-like appearance of a pig’s small intestine. This
etymology is also interesting from the viewpoint of distinguishing native vs.\
borrowed co-allofams. The usual, native words for ‘pig’ in Jingpho and Lahu are
\textbf{wàʔ} and \textbf{vàʔ}, respectively; but the doublets borrowed from Burmese have front
vowels, as in spoken Burmese. Unless a native speaker of Jingpho knows Burmese,
s/he is unlikely to realize that the first syllable of \textbf{wéʔ-wū} means ‘pig’,
especially since this syllable is in the high-stopped tone, while ‘pig’ is
low-stopped. The native Lahu speaker is even less likely to recognize the source
of \textbf{ɔ̀-vɛ̀ʔ}, since the morpheme for ‘intestine’ has been completely dropped from
the original Burmese compound, rather like the way our word \textit{camera} (< Lat.\
‘room; chamber; vaulted enclosure’) is a shortening of the old compound \textit{camera
obscura} (“dark chamber”).\footnote{There is a difference in detail between the
two cases, however: the deleted ‘intestine’ is the head of the compound
“pig-intestine”, but the deleted \textit{obscura} is the modifier in the collocation
“dark-chamber”.}

\item Combining reflexes of unrelated roots

When two forms bearing a semantic resemblance in a phonologically depleted
language differ only in tone, it is tempting to try to relate them. I once
entertained the possibility that such pairs of Lahu forms as \textbf{phu} ‘silver, money’
/ \textbf{phû} ‘price, cost’ and \textbf{mu} ‘high, tall’ / \textbf{mû} ‘sky’ were co-allofams, though they
can easily be shown to descend from quite separate etyma: \textbf{phu} < PTB \textbf{*plu} (\textit{STC} p.
89) / \textbf{phû} < PTB \textbf{*pəw} (\textit{STC} \#41);
\textbf{mu} < PTB \textbf{*mraŋ} (\textit{STC} p.~43) / \textbf{mû} < PTB \textbf{*r-məw}
(\textit{STC} \#488).\footnote{See Matisoff 1973b (\textit{GL}:29); such speculations were debunked in
the 2nd Printing (1982) of \textit{GL}, p.~675.}

\item Failure to recognize that separately reconstructed etyma are really co-allofams

An opposite type of error is to overlook the etymological identity between sets
of forms, assigning them to separate etyma when they are really co-allofams.
Thus \textit{STC} sets up two independent PTB roots, both with the shape \textbf{*dyam}, one
meaning ‘full; fill’ (\textit{STC} \#226) and the other glossed as ‘straight’ (\textit{STC} \#227).
Yet it can be shown that the latter root also means ‘flat’, and that all
reflexes of \#226 and \#227 may be subsumed under a single etymon, with the
underlying idea being “perfection in a certain dimension”.\footnote{See Matisoff
1988b:4-9.}


Similarly, I was slow to recognize that two roots I had set up separately,
PLB \textbf{*dzay²} ‘cattle; domestic animal’ (Matisoff 1985a \#129)
and Kamarupan \textbf{*tsaːy} ‘elephant;
cattle’ (\#143) are really one and the same.\footnote{I have argued that a
third root set up in Matisoff 1985a (\textit{GSTC} \#106), \textbf{*(t)saːy} \STEDTU{⪤} \textbf{*(d)zaːy} ‘temperament / aptitude
/talent’, is also related, the common notion being ‘property (either material or
intellectual)’. See Matisoff 1985a:44-45; 1988b:10-13.}

\item Double-dipping

This embarrassing situation occurs when an author inadvertently assigns the same
form in a daughter language to two different etyma, perhaps within the pages of
the same book, but more likely in separate articles. At different times I have
compared Chinese \textbf{chún} \TC{唇} ‘lip’ (OC \textbf{d̑i̯wən}) to both PTB \textbf{*dyal}
and \textbf{*m-ts(y)ul}, finally deciding in favor of the latter.\footnote{See \textit{HPTB}
9.2.1, 9.22(4), 9.2.4.}  It
is of course perfectly legitimate to change one’s mind, as long as one explains
why. The best course is to present the alternative etymologies together,
inviting the reader to choose between them.

\item Misanalyses of compounds

A vast number of words in TB languages are di- or tri-syllabic compounds, a fact
which greatly complicates the task of etymologization.  Many traps lie in wait
for the analyst, leading to potential errors of several kinds.

\begin{enumerate}
\item Wrong segmentation

This can happen when a form in an inadequately transcribed source is not
syllabified.  The Pochury and Sangtam forms for ‘star’, transcribed as \textbf{awutsi}
and \textbf{chinghi}, respectively, in the little glossaries compiled by the
\textit{Nagaland
Bhasha Parishad},\footnote{Kumar et al., \textit{Hindi Pochury English Dictionary} (1972);
\textit{Hindi
Sangtam English Dictionary} (1973). Kohima: Linguistic Circle of Nagaland.}
 should
be segmented as \textbf{a-wu-tsi} and \textbf{ching-hi},
and not as \textbf{a-wut-si} and \textbf{chi-nghi}, as I
imprudently did in Matisoff 1980:21.


\item Misunderstanding the meaning of a constituent

A special case of this problem is mistaking an affix for a root, especially
likely to occur when no grammatical description exists for a language. Several
Naga languages have dissyllabic forms for ‘moon’ with similar final syllables,
e.g.\ Chang \textbf{litnyu}, Konyak \textbf{linnyu},
Phom \textbf{linnyü}, Sangtam \textbf{chonu}, Liangmai \textbf{chahiu}.
Yet these final elements do not constitute a new root meaning ‘moon’, as I had
originally guessed; rather they represent an abstract formative, ultimately
grammaticalized from a root \textbf{*n(y)u} ‘mother’, that occurs in nouns from all sorts
of semantic fields (e.g.\ Chang \textbf{chinyu} ‘center’,
\textbf{henyu} ‘ladder’, \textbf{lamnyu} ‘road’,
\textbf{pinyu} ‘snake’).\footnote{See Matisoff 1980 (“Stars, moon, spirits”), p.\ 35;
for the
suffixal use of morphemes meaning ‘mother’, see Matisoff 1991b (“The mother of all
morphemes”).}


\item Choosing the wrong syllable of a compound for an etymology

This can happen when two different syllables of a compound are phonologically
similar, especially if one is dealing with a poorly known language with depleted
final consonants, e.g.\ forms like Guiqiong Ganzi \textbf{tʃhə⁵⁵sɑ̃⁵⁵}
and Ersu \textbf{ʂɿ⁵⁵ji⁵⁵} ‘otter’.
Which syllables are to be ascribed to PTB \textbf{*sram}? 
\end{enumerate}

\end{itemize}




\subsection{Looking toward the future of ST/TB studies}


Although I feel that we are entering a new era of etymological
responsibility in TB/ST studies—the bar has been raised, as it were—I am
not suggesting that we turn our fieldinto a “tough neighborhood” like that of
the Indo-Europeanists. In particular I hope we can avoid the \textit{“Gotcha!”}
attitude,\footnote{Non-American readers might need a word of explanation here. 
“Gotcha!” is an attempt to render the colloquial pronunciation of “(I’ve) got
you (now)!”, a triumphant phrase used by someone who feels he has won an
argument.} whereby if a single error, real or fancied, is found in an article or
book, the whole work is impugned. This attitude is encapsulated in the dreadful
maxim \textit{Falsum in uno, falsum in omnibus.}\footnote{“If one thing is wrong, it’s
all wrong.”} Historical linguists cannot afford to be too thin-skinned, as long
as criticism is fair, constructive, and proportionate. As I have said in print,
“I ask nothing better than to be corrected.”\footnote{Matisoff 1985b:422 (“Out
on a limb”).}  Or again, “We can take comfort from our mistakes. Reconstruction
of a proto-lexicon is a piecemeal process. It is hardly surprising that we
stumble along from one half-truth to another, as we try to trace the
[phonological and] semantic interconnections among our reconstructed etyma. We
should not be discouraged if we barge off down blind alleys occasionally, or if
the solution to one problem raises as many questions as it answers.”\footnote{Matisoff 1988a:13.} After all, a computerized etymological enterprise by its
very nature is eminently revisable.  The reconstructive process by its very
nature is provisional and open-ended.  Our STEDT etymologies undergo a constant
process of “rectification”, and may be roughly divided into three types: (a)
those to be accepted as is; (b) those to be accepted with modifications; (c)
those to be rejected.  As with all scientific hypotheses, our reconstructions
are falsifiable in the light of new data or better analyses.


We still have a long way to go before comparative/historical TB studies are
as advanced as they deserve to be.  Despite the quickening pace of research, our
knowledge of the various branches of this multifarious family remains highly
uneven. With a few important exceptions mentioned above, reliable
reconstructions at the subgroup level are not yet available.  Many more roots
remain to be reconstructed at all taxonomic levels of the family. Much remains
to be done on the Chinese side as well, and we seem destined for a period of
flux until the dust settles and competing reconstructions of OC have sorted
themselves out.


Nevertheless, it is hard not to be optimistic about the future of TB/ST
linguistics, as fieldwork opportunities increase and new generations of talented
researchers enter the discipline. Eventually it seems inevitable that scholars
throughout the world will share their information more and more, granting mutual
access to their databases for the common good. On the other hand, too many TB
languages are endangered, and may well disappear before they have been
adequately recorded. In any case, “the reconstruction of PTB is a noble
enterprise, where a spirit of competitive territoriality is out of place. We
should pool our knowledge and encourage each other to venture outside of our
specialized niches, so that we begin to appreciate the full range of TB
languages....”\footnote{Matisoff 1982:41.  There is nothing more satisfying
than to have inadequate data on a language of which one has no firsthand
knowledge corrected by a specialist in that language.  The STEDT project
has recently (summer of 2007) benefited tremendously from the kindness of K.P.
Malla, who edited all the Newar(i) forms in our database, identifying loanwords,
putting verbs into their proper citation forms, and correcting the transcription
of vowels and consonants used in our previous sources.}

\cleartooddpage[\thispagestyle{empty}]
