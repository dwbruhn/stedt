\renewcommand\thefootnote{}
\chapter[References]{References\footnotemark}
\footnotetext{Sources for particular records not mentioned specifically in the text are listed in the Appendix of Source Abbreviations.}
\newcommand{\refitem}{\hangindent=0.25in}
\newcommand{\rrefline}{\rule[3pt]{3em}{0.5pt}. } % horizontal line for repeat author
\label{pg:start-refs}

\refitem
%: 
\textit{American Heritage Dictionary of the English Language.}
2000.
% empty author --> put in beginning
J. P. Pickett, et al., eds. Boston: Houghton Mifflin Company.

\refitem
%: MB-Lal
Balawan, M.
1965.
\textit{A First Lalung Dictionary, with the corresponding words in English and Khasi.}
Shillong.

\refitem
%: RSB-STV
Bauer, Robert S.
1991.
“Sino-Tibetan *vulva.”
\textit{LTBA} 14.1:147-72.

\refitem
%: WHB-OC
Baxter, William.
1992.
\textit{A Handbook of Old Chinese Phonology.}
Berlin, New York: Mouton de Gruyter.

\refitem
%: 
Baxter, William H. and Laurent Sagart.
1998.
“Word formation in Old Chinese.”
In \textit{New approaches to Chinese word formation: morphology, phonology and the lexicon in modern and ancient Chinese.} Jerome L.\ Packard, ed., 35-76. Berlin and New York: Mouton de Gruyter.

\refitem
%: 
Benedict, Paul K.
1939.
“Semantic differentation in Indo-Chinese.”
HJAS 4:213-29.

\refitem
%: PKB-KSEA
\rrefline % Benedict, Paul K.
1941/2008.
\textit{Kinship in Southeastern Asia.}
Ph.D. dissertation, Department of Anthropology, Harvard University (1941). To be published as STEDT Monograph \#6, 2008.

\refitem
%: STC
\rrefline % Benedict, Paul K.
1972.
\textit{Sino-Tibetan: a Conspectus.}
James A.\ Matisoff, contributing editor. Princeton-Cambridge Series in Chinese Linguistics, \#2.  New York: Cambridge University Press.

\refitem
%: 
\rrefline % Benedict, Paul K.
1975a.
\textit{Austro-Thai Language and Culture, with a glossary of roots.}
New Haven: HRAF Press.

\refitem
%: 
\rrefline % Benedict, Paul K.
1975b.
“Where it all began: memories of Robert Shafer and the \textit{Sino-Tibetan Linguistics Project}, Berkeley 1939-40.”
\textit{LTBA} 2.1:81-92.

\refitem
%: 
\rrefline % Benedict, Paul K.
1976.
“Sino-Tibetan: another look.”
\textit{JAOS} 96.2:167-97.

\refitem
%: PKB-KLH
\rrefline % Benedict, Paul K.
1979.
“Four forays into Karen linguistic history.”
Edited and expurgated by James A. Matisoff. \textit{LTBA}. 5.1:1-35. [“A note on the loss of final stop in Karen”, pp.\ 4-7; “A note on the reconstruction of Karen preglottalized surd stops”, pp.\ 8-12; “A note on the reconstruction of Karen final \textbf{*-s}”, pp.\ 13-20; “A note on Karen genital flipflop”, pp.\ 21-24.]

\refitem
%: 
\rrefline % Benedict, Paul K.
1981.
“A further (unexpurgated) note on Karen genital flipflop.”
\textit{LTBA} 6.1:103.

\refitem
%: 
\rrefline % Benedict, Paul K.
1983.
“Qiang monosyllables: a third phase in the cycle.”
\textit{LTBA} 7.2:113-14.

\refitem
%: 
\rrefline % Benedict, Paul K.
1988.
Untitled ms. circulated at ICSTLL \#21, Lund, Sweden.


\refitem
%: 
\rrefline % Benedict, Paul K.
1990.
\textit{Austro-Tai/Japanese.}
Ann Arbor: Karoma Press.

\refitem
%: 
\rrefline % Benedict, Paul K.
1991.
“Genital flipflop: a Chinese note.”
\textit{LTBA} 14.1:143-6.

\refitem
%: 
Bernot, Denise.
1978-92.
\textit{Dictionnaire birman-français.}
15 fascicules. Paris: SELAF.

\refitem
%: Bhat-Boro
Bhat, D.~N.~Shankara.
1968.
\textit{Boro Vocabulary (with a grammatical sketch).}
Poona: Deccan College Postgraduate and Research Institute.

\refitem
%: 
Bodman, Nicholas C.
1980.
“Proto-Chinese and Sino-Tibetan: data towards establishing the nature of the relationship.”
In Frans van Coetsem and Linda R. Waugh, eds., \textit{Contributions to Historical Linguistics: Issues and Materials}, pp.\ 34-199. Leiden: E. J. Brill.

\refitem
%: 
\rrefline % Bodman, Nicholas C.
1969.
“Tibetan \textit{sdud} ‘folds of a garment’, the character \TC{卒}, and the \textbf{*st-} hypothesis.”
\textit{AS/BIHP} 39:327-45.

\refitem
%: DB-PLolo
Bradley, David.
1979.
\textit{Proto-Loloish.}
Scandinavian Institute of Asian Studies Monograph Series, no.~39.  London and Malmö: Curzon Press.

\refitem
%: 
Buck, Carl Darling.
1949.
\textit{A Dictionary of Selected Synonyms in the Principal Indo-European Languages: a contribution to the history of ideas.}
Chicago: University of Chicago Press.

\refitem
%: RB-PB
Burling, Robbins.
1959.
“Proto-Bodo.”
\textit{Language} 35:433-53.

\refitem
%: 
\rrefline % Burling, Robbins.
1966.
“The addition of final stops in the history of Maru.”
\textit{Language} \mbox{42.3:581-86}.

\refitem
%: RB-PLB
\rrefline % Burling, Robbins.
1967/1968.
\textit{Proto-Lolo-Burmese.}
Indiana University Research Center in Anthropology, Folkore, and Linguistics, publication 43.  The Hague: Mouton. Issued simultaneously as a Special Publication, \textit{IJAL} 33.2, Part II.

\refitem
%: RB-PKR
\rrefline % Burling, Robbins.
1969.
“Proto-Karen:  a reanalysis.”
\textit{OPWSTBL} vol.\ I, Alton L.\ Becker, ed., pp.\ 1-116.  Ann Arbor: University of Michigan.

\refitem
%: 
\rrefline % Burling, Robbins.
1983.
“The \textit{sal} languages.”
\textit{LTBA} 7.2:1-32.

\refitem
%: RB-GB
\rrefline % Burling, Robbins.
1992.
\textit{Garo (Bangladesh dialect) Semantic Dictionary.}
Unpublished.

\refitem
%: 
\rrefline % Burling, Robbins.
1999.
“On Kamarupan.”
\textit{LTBA} 22.2:169-71.

\refitem
%: RB-LMMG
\rrefline % Burling, Robbins.
2003.
\textit{The Language of the Modhupur Mandi (Garo). Vol.~III: Glossary.}
Ann Arbor, Michigan.

\refitem
%: WSC-SH
Coblin, Weldon South.
1986.
\textit{A Sinologist’s Handlist of Sino-Tibetan Lexical Comparisons.}
Monumenta Serica Monograph Series, Vol.\ 18.  Nettetal: Steyler Verlag.

\refitem
%: 
Cook, Richard S.
1996.
\textit{The Etymology of Chinese \TC{辰} \textnormal{Chén}.}
\textit{LTBA} 18.2:1-238.

\refitem
%: 
Cushing, Josiah N.
1881/1914.
\textit{A Shan and English Dictionary.}
Second edition (1914). Rangoon: American Baptist Mission Press.

\refitem
%: TBL
Dai Qingxia \SC{戴庆厦}, et al., eds.
1992.
\SC{藏缅语族语言词汇} \textit{A Tibeto-Burman Lexicon.}
Beijing: Central Institute of Minorities. (“TBL”)

\refitem
%: JCD
Dai Qingxia \SC{戴庆厦}, Xu Xijian \SC{徐悉艰} , et al.
1983.
\SC{景汉辞典} \textit{Jing-Han cidian – Jinghpo Miwa ga ginsi chyum – Jinghpo-Chinese dictionary.}
Kunming: Yunnan Nationalities Press.

\refitem
%: SVD-Lim
Driem, George van.
1987.
\textit{A Grammar of Limbu.}
Mouton Grammar Library \#4.  Berlin, New York, Amsterdam: Mouton de Gruyter.

\refitem
%: SVD-Dum
\rrefline % Driem, George van.
1993.
\textit{A Grammar of Dumi.}
Mouton Grammar Library \#10.  Berlin, New York: Mouton de Gruyter.

\refitem
%: 
\rrefline % Driem, George van.
2003.
Review of Graham Thurgood and Randy J.\ LaPolla (eds.): \textit{The Sino-Tibetan Languages.} London and New York: Routledge. 2003.
\textit{BSOAS}, 66.2:282-84.

\refitem
%: 
Duàn Yùcái \TC{段玉裁}.
1815.
\TC{說文解字注} \textit{Shuōwén Jiězì Zhù [Commentary on the Shuowen Jiezi].}
Reprinted 1989 by Shànghǎi Gǔjí Chūbǎnshè.

\refitem
%: WTF-PNN
French, Walter T.
1983.
\textit{Northern Naga: a Tibeto-Burman Mesolanguage.}
Ph.D. dissertation, City University of New York.

\refitem
%: 
Gong Hwang-cherng \TC{龔煌城}.
1989/2002.
“The phonological reconstruction of Tangut through examination of phonological alternations.”
Reprinted in Gong 2002, pp.\ 75-110.  Originally published in: \textit{AS/BIHP} 60.1:1-45.

\refitem
%: 
\rrefline % Gong Hwang-cherng \TC{龔煌城}.
1990.
\TC{從漢藏語的比較看上古漢語若干的擬測} “Cóng Hàn-Zàngyǔ de bǐjiào kàn Shànggǔ Hànyǔ ruògān shēngmǔ de nǐcè [Reconstruction of some initials in Archaic Chinese from the viewpoint of comparative Sino-Tibetan].”
In \textit{A Collection of Essays in Tibetan Studies}, Vol.~3, pp.\ 1-18. Taipei: Committee on Tibetan Studies.

\refitem
%:
\rrefline % Gong Hwang-cherng \TC{龔煌城}.
1994.
“The first palatalization of velars in Late Old Chinese.”
In Matthew Y. Chen and Ovid J. L. Tzeng, eds., \textit{Linguistics Essays in Honor of William S.-Y.~Wang: inter-disciplinary studies on language and language change}, pp.\ 131-142. Taipei: Pyramid Press.

\refitem
%:
\rrefline % Gong Hwang-cherng \TC{龔煌城}.
1995.
“The system of finals in Proto-Sino-Tibetan.” 
In William S.-Y. Wang, ed., \textit{The Ancestry of the Chinese Language}, pp.\ 41-92. Berkeley: POLA.

\refitem
\rrefline % Gong Hwang-cherng \TC{龔煌城}.
1997.
\TC{從漢藏語的比較看重紐問題(兼論上古介音對中古韻母演變的影像)} “Cóng Hàn-Zàng yǔ de bǐjiào kàn chóngniǔ wèntí (jiān lùn Shànggǔ \textbf{*-rj-} jièyīn duì Zhōnggǔ yùnmǔ yǎnbiàn de yǐngxiǎng) [The \textit{chongniu} problem from the viewpoint of comparative Sino-Tibetan (with discussion of the effect of the Old Chinese medial \textbf{*-rj-} on the development of Middle Chinese rhymes)].”
In Republic of China Phonology Conference, Taiwan Normal University Chinese Department, and Academia Sinica Institute of History and Philology, eds., \TC{聲韻論叢} \textit{Shēngyùn lùn cóng [Collected essays in Chinese phonology]}, Vol.~VI, pp.\ 195-243. Taipei: \TC{台灣學生書局} Táiwān Xuéshēng Shūjú.

\refitem
%: 
\rrefline % Gong Hwang-cherng \TC{龔煌城}.
2000.
\TC{從漢藏語的比教看上古漢語的詞頭問題} “Cóng Hàn-Zàng yǔ de bǐjiào kàn Shànggǔ Hànyǔ de cítóu wèntí [The problem of Old Chinese prefixes from the perspective of comparative Sino-Tibetan studies].” \textit{Languages and Linguistics} (Taipei) 1.2:39-62.

\refitem
%: 
\rrefline % Gong Hwang-cherng \TC{龔煌城}.
2002.
\TC{漢藏語硏究論文集} \textit{Hàn Zàng yǔ yánjiù lùnwén jí [Collected papers on Sino-Tibetan linguistics].}
Language and Linguistics Monograph Series, C2-2. Taipei: Institute of Linguistics (Preparatory Office), Academia Sinica.


\refitem
%: LSI
Grierson, Sir George Abraham and Sten Konow (eds.).
1903-28.
\textit{Linguistic Survey of India.}
13 vols. Calcutta: Office of the Superintendent of Government Printing. Reprinted (1967, 1973), Delhi: Motilal Banarsidass.

\refitem
%: EG-Tangut
Grinstead, Eric.
1972.
\textit{Analysis of the Tangut Script.}
Scandinavian Institute of Asian Studies Monograph Series, \#10.  Lund: Studentlitteratur.

\refitem
%: AH-CSDPN
Hale, Austin.
1973.
\textit{Clause, Sentence, and Discourse Patterns in Selected Languages of Nepal IV: Word Lists.}
Summer Institute of Linguistics Publications in Linguistics and Related Fields \#40. Kathmandu: SIL and Tribhuvan University Press.

\refitem
%: 
Handel, Zev J.
1998.
\textit{The Medial Systems of Old Chinese and Proto-Sino-Tibetan.}
Ph.D.\ dissertation, University of California, Berkeley.

\refitem
%: OH-DKL
Hanson, Ola.
1906.
\textit{A Dictionary of the Kachin Language.}
Rangoon. Reprinted (1954, 1966), Rangoon: Baptist Board of Publications.

\refitem
%: 
Haudricourt, André-Georges.
1942-5.
“Restitution du karen commun.”
\textit{BSLP} 42.1:103-11.

\refitem
%: 
\rrefline % Haudricourt, André-Georges.
1975.
“Le système de tons du karen commun.”
\textit{BSLP} 70:339-43.

\refitem
%: EJAH-BKD
Henderson, Eugénie J.~A.
1997.
\textit{Bwe Karen Dictionary.}
School of Oriental and African Studies, University of London.

\refitem
%: HOD1857
Hodgson, Brian Houghton.
1857.
“Comparative vocabulary of the several languages (dialects) of the celebrated people called Kirântis.”
\textit{JASB} 26.5:333-71.

\refitem
%:
Hyman, Larry M., ed.
1973.
\textit{Consonant Types and Tone.}
Southern California Occasional Papers in Linguistics \#1. Los Angeles: University of California, Los Angeles.

\refitem
%: 
Imoba, S.
2004.
\textit{Manipuri to English Dictionary.}
Imphal: S.~Ibetombi Devi.

\refitem
%: HAJ-TED
Jäschke, Heinrich August.
1881/1958.
\textit{A Tibetan-English Dictionary, with special reference to the prevailing dialects.}
London. Reprinted (1958) by Routledge and Kegan Paul.

\refitem
%: RBJ-KLS
Jones, Robert B., Jr.
1961.
\textit{Karen Linguistic Studies:  description, comparison, and texts.}
UCPL \#25.  Berkeley and Los Angeles: University of California Press.

\refitem
%: AJ-BED
Judson, Adoniram.
1893.
\textit{Burmese-English Dictionary.}
Rangoon. Revised and enlarged (1953) by Robert C.\ Stevenson and F.\ H.\ Eveleth. Reprinted (1966), Rangoon: Baptist Board of Publications.

\refitem
%: BK-AD
Karlgren, Bernhard.
1923.
\textit{Analytic Dictionary of Chinese and Sino-Japanese.}
Paris: P.~Geuthner.

\refitem
%: 
\rrefline % Karlgren, Bernhard.
1933.
“Word families in Chinese.”
\textit{BMFEA} 5:5-120.

\refitem
%: GSR
\rrefline % Karlgren, Bernhard.
1957.
\textit{Grammata Serica Recensa.}
Stockholm: Museum of Far Eastern Antiquities, Publication 29. (“GSR”)

\refitem
%: 
Kitamura Hajime \TC{北村甫}, Nishida Tatsuo \TC{西田龍雄}, and Nagano Yasuhiko \TC{長野泰彥}, eds.
1994.
\textit{Current Issues in Sino-Tibetan Linguistics.}
Osaka: Organizing Committee of 26th ICSTLL. (“CISTL”)

\refitem
%: 
Kumar, Braj Bihari and Thimase Pocuri.
1972.
\textit{Hindi-Pochury-English Dictionary.}
Kohima, Nagaland: Nagaland Bhasha Parishad (Linguistic Circle of Nagaland).

\refitem
%: 
Kumar, Braj Bihari, et al.
1973.
\textit{Hindi-Sangtam-English Dictionary.}
Kohima, Nagaland: Nagaland Bhasha Parishad (Linguistic Circle of Nagaland).

\refitem
%: LI1980
Li Fang-Kuei \TC{李方桂}.
1971/1980.
\TC{上古音研究} “Shànggǔyīn Yánjiū [Studies on Old Chinese phonology].”
\textit{Tsing Hua Journal of Chinese Studies}, n.s.~9:1-61. Reprinted (1980), Beijing: \SC{商务印书馆} Shāngwù Yìnshūguǎn, pp.\ 1-83.

\refitem
%: 
\rrefline % Li Fang-Kuei \TC{李方桂}.
1976.
\TC{幾個上古聲母問題} “Jǐge Shànggǔ shēngmǔ wèntí [Some problems concerning Old Chinese initials].”
In \TC{總統蔣公逝世週年論文集} \textit{Zǒngtǒng Jiǎng gōng shìshì zhōunián lùnwén jí [Collected papers in commemoration of the anniversary of the death of President Chiang]}, 1143-50. Taipei: Academia Sinica. Reprinted in Li 1980:85-94.

\refitem
%:
\rrefline % Li Fang-Kuei \TC{李方桂}.
1977.
\textit{A Handbook of Comparative Tai.}
Oceanic Linguistics Special Publication \#15. Honolulu: University Press of Hawaiʻi. (“HCT”)

%ignore
\rrefline
1980.
See Li 1971.

\refitem
%: LL-CMST
Löffler, Lorenz G.
1966.
“The contribution of Mru to Sino-Tibetan linguistics.”
\textit{ZDMG} 116.1:118-59.

\refitem
%: JHL-AM
Lorrain, J.~Herbert.
1907.
\textit{A Dictionary of the Abor-Miri Language, with illustrative sentences and notes.}
Shillong: Eastern Bengal and Assam Secretariat Printing Office.

\refitem
%: GHL-CWL
Luce, G.~H.
1981.
\textit{A Comparative Word-List of Old Burmese, Chinese, and Tibetan.}
London: School of Oriental and African Studies, University of London.

\refitem
%: GHL-PPB
\rrefline % Luce, G.~H.
1986.
\textit{Phases of Pre-Pagán Burma: languages and history.}
Vol.~2. Oxford: Oxford University Press.

\refitem
%: GBM-Lepcha
Mainwaring, G.B.
1898.
\textit{Dictionary of the Lepcha Language.}
Revised and completed by Albert Grünwedel. Berlin: Unger Brothers.

\refitem
%: GEM-CNL
Marrison, G.E.
1967.
\textit{The Classification of the Naga Languages of Northeast India.}
Ph.D. dissertation, School of Oriental and African Studies, University of London.  2 vols.

\refitem
%: JAM-LPLB
Matisoff, James A.
1969.
“Lahu and Proto-Lolo-Burmese.”
\textit{OPWSTBL} vol.\ I, Alton L.\ Becker, ed., pp.\ 117-221.  Ann Arbor: University of Michigan.

\refitem
%: 
\rrefline % Matisoff, James A.
1970.
“Glottal dissimilation and the Lahu high-rising tone: a tonogenetic case-study.”
\textit{JAOS} 90.1:13-44.

\refitem
%: JAM-TSR
\rrefline % Matisoff, James A.
1972a.
\textit{The Loloish Tonal Split Revisited.}
Research Monograph \#7.  Berkeley: Center for South and Southeast Asian Studies, University of California, Berkeley.

\refitem
%: 
\rrefline % Matisoff, James A.
1972b.
“Tangkhul Naga and comparative Tibeto-Burman.”
\textit{TAK} 10.2:1-13.

\refitem
%: 
\rrefline % Matisoff, James A.
1973a.
“Tonogenesis in Southeast Asia.”
In L.~M.~Hyman, ed., pp.\ 71-96.

\refitem
%: 
\rrefline % Matisoff, James A.
1973b/1982.
\textit{The Grammar of Lahu.}
UCPL \#75. Berkeley and Los Angeles: University of California Press. Reprinted 1982.

\refitem
%: 
\rrefline % Matisoff, James A.
1974.
“The tones of Jinghpaw and Lolo-Burmese: common origin vs. independent development.”
\textit{Acta Linguistica Hafniensia} (Copenhagen) 15.2, 153-212.

\refitem
%: JAM-Rhino
\rrefline % Matisoff, James A.
1975.
“Rhinoglottophilia: the mysterious connection between nasality and glottality.”
In C.\ A.\ Ferguson, L.\ M.\ Hyman, and J.\ J.\ Ohala, eds., \textit{Nasálfest} pp.\ 267-287.  Stanford, CA.

\refitem
%: JAM-VSTB
\rrefline % Matisoff, James A.
1978a.
\textit{Variational Semantics in Tibeto-Burman: the ‘organic’ approach to linguistic comparison.}
\textit{OPWSTBL} \#6.  Philadelphia: Institute for the Study of Human Issues.

\refitem
%: JAM-MLBM
\rrefline % Matisoff, James A.
1978b.
“Mpi and Lolo-Burmese microlinguistics.”
\textit{Monumenta Serindica} (ILCAA, Tokyo) 4:1-36.

\refitem
%: 
\rrefline % Matisoff, James A.
1980.
“Stars, moon, and spirits: bright beings of the night in Sino-Tibetan.”
\textit{GK} 77:1-45.

\refitem
%: 
\rrefline % Matisoff, James A.
1982.
“Proto-languages and proto-Sprachgefühl.”
\textit{LTBA} 6.2:1-64.

\refitem
%: JAM-TIL
\rrefline % Matisoff, James A.
1983.
“Translucent insights:  a look at Proto-Sino-Tibetan through Gordon H. Luce’s comparative word-list.”
\textit{BSOAS} 46.3:462-76.

\refitem
%: JAM-GSTC
\rrefline % Matisoff, James A.
1985a.
“God and the Sino-Tibetan copula, with some good news concerning selected Tibeto-Burman rhymes.”
\textit{Journal of Asian and African Studies} (Tokyo) 29:1-81.

\refitem
%: JAM-AHWST
\rrefline % Matisoff, James A.
1985b.
“Out on a limb: \textit{arm}, \textit{hand}, and \textit{wing} in Sino-Tibetan.”
In Graham Thurgood, et al., eds., \textit{Linguistics of the Sino-Tibetan area: the state of the art}, pp.\ 421-425.  (Pacific Linguistics Series C, No.~87).  Canberra: Australian National University.

\refitem
%: JAM-LDTB
\rrefline % Matisoff, James A.
1986.
“The languages and dialects of Tibeto-Burman: an alphabetic/genetic listing, with some prefatory remarks on ethnonymic and glossonymic complications.”
In John McCoy and Timothy Light, eds., \textit{Contributions to Sino-Tibetan Studies}, pp.\ 1-75.  Leiden: E.J.\ Brill. Revised and reprinted (1996) as STEDT Monograph \#2, with Stephen P.\ Baron and John B.\ Lowe.

\refitem
%: JAM-DL
\rrefline % Matisoff, James A.
1988a.
\textit{The Dictionary of Lahu.}
UCPL \#111.  Berkeley, Los Angeles, London: University of California Press.

\refitem
%: 
\rrefline % Matisoff, James A.
1988b.
“Universal semantics and allofamic identification: two Sino-Tibetan case-studies: \textit{straight/flat/full} and \textit{property/livestock/talent}.”
In Akihiro Sato, ed., \textit{Languages and History in East Asia: Festschrift for Tatsuo Nishida on the Occasion of his 60th Birthday}, pp.\ 3-14. Kyoto: Shokado.

\refitem
%: 
\rrefline % Matisoff, James A.
1990a.
“On megalocomparison.”
\textit{Language} 66.1:106-20.

\refitem
%: 
\rrefline % Matisoff, James A.
1990b.
“The dinguist’s dilemma: \textbf{l/d} interchange in Sino-Tibetan.”
Paper presented at ICSTLL \#23, University of Texas, Arlington.

\refitem
%: 
\rrefline % Matisoff, James A.
1991a.
“Areal and universal dimensions of grammatization in Lahu.”
In Elizabeth C.\ Traugott and Bernd Heine, eds., \textit{Approaches to Grammaticalization}, Vol.~II, pp.\ 383-453. Amsterdam: Benjamins.

\refitem
%: 
\rrefline % Matisoff, James A.
1991b.
“The mother of all morphemes.”
In Martha Ratliff and Eric Schiller, eds. \textit{Papers from the First Annual Meeting of the Southeast Asian Linguistics Society} (SEALS), pp.\ 293-349. Tempe: Arizona State University.

\refitem
%: 
\rrefline % Matisoff, James A.
1991c.
“Jiburish revisited: tonal splits and heterogenesis in Burmo-Naxi-Lolo checked syllables.”
\textit{AO} 52:91-114.

\refitem
%: 
\rrefline % Matisoff, James A.
1992.
“Following the marrow: two parallel Sino-Tibetan etymologies.”
\textit{LTBA} 15.1:159-177.

\refitem
%: 
\rrefline % Matisoff, James A.
1994a.
“Regularity and variation in Sino-Tibetan.”
In \textit{CISTL}, pp.\ 36-58.

\refitem
%:
\rrefline % Matisoff, James A.
1994b.
“How dull can you get?: \textit{buttock} and \textit{heel} in Sino-Tibetan.”
\textit{LTBA} 17.2:137-51. Reprinted in Pierre Pichard and François Rabine, eds., \textit{Études birmanes en hommage à Denise Bernot}, pp.\ 373-83. Paris: EFEO.

\refitem
%: 
\rrefline % Matisoff, James A.
1995.
“Sino-Tibetan palatal suffixes revisited.”
In \textit{NHTBM}, pp.\ 35-91. Osaka: National Museum of Ethnology.

\refitem
%: JAM-LITB
\rrefline % Matisoff, James A.
1997.
“Primary and secondary laryngeal initials in Tibeto-Burman.”
In Anne O.\ Yue and Mitsuaki Endo, eds., \textit{In Memory of Mantaro J. Hashimoto}, pp.\ 29-50. Tokyo: Uchiyama Books Co.

\refitem
%: 
\rrefline % Matisoff, James A.
1999.
“In defense of Kamarupan.”
\textit{LTBA} 22.2:173-182.

\refitem
%: 
\rrefline % Matisoff, James A.
2000a.
“An extrusional approach to \textbf{*p-/w-} variation in Sino-Tibetan.”
\textit{Language and Linguistics} (Taipei) 1.2:135-86.

\refitem
%: 
\rrefline % Matisoff, James A.
2000b.
“Three Tibeto-Burman/Sino-Tibetan word families: \textit{set (of the sun); pheasant/peacock; scatter/pour}.”
In Marlys Macken, ed., \textit{Papers from the Tenth Annual Meeting of the Southeast Asian Linguistics Society} (SEALS), pp.\ 215-32. Tempe: Arizona State University.

\refitem
%: 
\rrefline % Matisoff, James A.
2003.
\textit{Handbook of Proto-Tibeto-Burman: system and philosophy of Sino-Tibetan reconstruction.}
UCPL \#135.  Berkeley and Los Angeles: University of California Press. (“HPTB”)

\refitem
%: 
\rrefline % Matisoff, James A.
2004.
“Areal semantics: is there such a thing?”
In Anju Saxena, ed., \textit{Himalayan Languages, Past and Present}, pp.\ 347-393.  Berlin and New York: Mouton de Gruyter.


\refitem
%:
\rrefline % Matisoff, James A.
2007a.
“Response to Laurent Sagart’s review of \textit{Handbook of Proto-Tibeto-Burman}.”
\textit{Diachronica} 24.2:435-44.

\refitem
%: 
\rrefline % Matisoff, James A.
2007b.
“The fate of the Proto-Lolo-Burmese rhyme \textbf{*-a}: regularity and exceptions.”
Paper presented at ICSTLL \#40, Heilongjiang University, Harbin, China.

\refitem
%: MM-K78
Mazaudon, Martine.
1978.
“Consonantal mutation and tonal split in the Tamang sub-family of Tibeto-Burman.”
\textit{Kailash} 6.3:157-79.

\refitem
%: MM-Thesis
\rrefline % Mazaudon, Martine.
1994.
\textit{Problèmes de comparatisme et de reconstruction dans quelques langues de la famille tibéto-birmane.}
Thèse d’État, Université de la Sorbonne Nouvelle, Paris.

\refitem
%: BM-PK7
Michailovsky, Boyd.
1991.
\textit{Proto-Kiranti.}
Unpublished ms.

\refitem
%: 
Miller, Roy Andrew.
1968.
“Once again, the Maru final stops.”
Paper presented at ICSTLL \#1, Yale University.

\refitem
%: 
Mills, James Philip.
1926/1973.
\textit{The Ao Nagas.}
London. Reprinted (1973), Delhi: Oxford University Press.

\refitem
%: 
Momin, K.~W\@.
n.d.
\textit{English-Achikku Dictionary.}
Printed by V.\ N.\ Bhattacharya at the Inland Printing Works, 60-3, Dharamtala Street, Calcutta-13.

\refitem
%: 
Monier-Williams, Sir Monier.
1899/1970.
\textit{A Sanskrit-English Dictionary.}
Delhi, Varanasi, Patna: Motilal Banarsidass.

\refitem
%: 
Namkung, Ju, ed.
1996.
\textit{Phonological Inventories of Tibeto-Burman Languages.}
STEDT Monograph \#3. Berkeley: University of California.

\refitem
Nishi Yoshio \TC{西義郎}, James A.\ Matisoff, and Nagano Yasuhiko \TC{長野泰彥}, eds.
1995.
\textit{New Horizons in Tibeto-Burman Morphosyntax.}
Senri Ethnological Studies \#41. Osaka: National Museum of Ethnology. (“NHTBM”)

\refitem
%: NT-SGK
Nishida Tatsuo \TC{西田龍雄}.
1964, 1966.
\TC{西夏語の研究} \textit{Seikago no kenkyū [A Study of the Hsi-Hsia Language: reconstruction of the Hsi-Hsia language and decipherment of the Hsi-Hsia script].}
Tokyo: \TC{座右宝刊行会} Zauhō Kankōkai.  2 vols. Vol.~I (1964), Vol.~II (1966).

\refitem
%: 
Noonan, Michael, et al.
1999.
\textit{Chantyal Dictionary and Texts.}
Berlin and New York: Mouton de Gruyter.

\refitem
%: 
\textit{Oxford English Dictionary.}
1971.
% blank line
Compact Edition, 2 vols. reproduced micrographically.  3rd U.S.\ Printing, 1973.  Oxford University Press.

\refitem
%: 
Pan Wuyun \SC{潘悟云}.
2000.
\SC{汉语历史音韵学} \textit{Hànyǔ lìshǐ yīnyùnxué [Chinese Historical Phonology].}
Shanghai: \SC{教育出版社} Jiàoyù Chūbǎnshè.

\refitem
%: 
Peiros, Ilia and S.A.~Starostin.
1996.
\textit{A Comparative Vocabulary of Five Sino-Tibetan Languages.}
5 fascicles. Melbourne: University of Melbourne.


\refitem
%: DAP-Chm
Peterson, David A.
2008.
“Bangladesh Khumi verbal classifiers and Kuki-Chin ‘chiming’.”
\textit{LTBA}, to appear.

\refitem
%: 
Pulleyblank, Edwin G.
1962.
“The consonantal system of Old Chinese.”
\textit{AM} 9:58-144, 206-265.

\refitem
%: 
Qu Wanli \TC{屈萬里}.
1983.
\TC{詩經詮釋} \textit{Shījīng Quánshì [Complete text of  the Book of Odes].}
Taipei: \TC{聯經出版公司} Liánjīng Chūbǎn Gōngsī.

\refitem
%: 
Sagart, Laurent.
1999.
\textit{The Roots of Old Chinese.}
Amsterdam Studies in the Theory and History of Linguistic Science \#184. Amsterdam: John Benjamins.

\refitem
%: 
Sagart, Laurent.
2007.
“Reconstructing Old Chinese uvulars in the Baxter-Sagart system (ver. 0.97).”
Paper presented at ICSTLL \#40, Heilongjiang University, Harbin.

\refitem
%: 
\rrefline % Sagart, Laurent.
2006.
Review of Matisoff 2003.
\textit{Diachronica} 23.1:206-223

\refitem
%: 
Schuessler, Axel.
1987.
\textit{A Dictionary of Early Zhou Chinese.}
Honolulu: University of Hawaiʻi Press.

\refitem
%: 
\rrefline % Schuessler, Axel.
2007.
\textit{ABC Etymological Dictionary of Old Chinese.}
Honololu: University of Hawaiʻi Press.

\refitem
%: 
Sedláček, Kamil.
1970.
\textit{Das Gemein-Sino-Tibetische.}
Wiesbaden: Franz Steiner Verlag.

\refitem
%: SHA1966-73
Shafer, Robert.
1966-73.
\textit{Introduction to Sino-Tibetan.}
5 parts.  Wiesbaden: Otto Harrassowitz.

\refitem
%: 
Simon, Walter.
1929.
“Tibetisch-chinesische Wortgleichungen: Ein Versuch.”
\textit{MSOS} 32.1:157-228.

\refitem
%: 
\rrefline % Simon, Walter.
1975.
“Tibetan initial clusters of nasal and R.”
\textit{AM} 19.2:246-51.

\refitem
%: MVS-Grin
Sofronov, M.V.
ca.~1978.
“Annotations to Grinstead 1972.”
Reconstructions of Tangut body part terms, personally entered into the glossary of Grinstead 1972. % EG-Tangut

\refitem
%: 
Stimson, Hugh.
1966.
“A taboo word in the Peking dialect.”
\textit{Language} 42.2:285-294.

\refitem
%: ZMYYC
Sun Hongkai \SC{孙宏开}, et al., eds.
1991.
\SC{藏缅语语音和词汇} \textit{Zàngmiǎnyǔ yǔyīn hé cíhuì [Tibeto-Burman Phonology and Lexicon].}
Beijing: Chinese Social Sciences Press.

\refitem
%: JS-HCST
Sun, Jackson Tianshin \TC{孫天心}.
1993.
\textit{A Historical-Comparative Study of the Tani (Mirish) Branch in Tibeto-Burman.}
Ph.D. dissertation, University of California, Berkeley.

\refitem
%: 
Thurgood, Graham.
1984.
“The \textit{Rung} languages: a major new TB subgroup.”
In \textit{Proceedings of the Tenth Annual Meeting of the Berkeley Linguistics Society}, pp.\ 338-349. University of California, Berkeley.

\refitem
%: 
Thurgood, Graham and Randy J.~LaPolla, eds.
2003.
\textit{The Sino-Tibetan Languages.}
London: Routledge.

\refitem
%: RLT-IAD
Turner, R.L.
1966.
\textit{A Comparative Dictionary of the Indo-Aryan Languages.}
London: Oxford University Press.

\refitem
%: KVB-PKC
VanBik, Kenneth.
2003.
\textit{Proto-Kuki-Chin.}
Ph.D. dissertation, University of California, Berkeley.

\refitem
%: GDW-DML
Walker, George David.
1925.
\textit{A Dictionary of the Mikir Language, Mikir-English and English-Mikir.}
Shillong: Assam Government Press.

\refitem
%: 
Weidert, Alfons K.
1975.
\textit{Componential Analysis of Lushai Phonology.}
Amsterdam: J.~Benjamins B.~V.

\refitem
%: 
\rrefline % Weidert, Alfons K.
1979.
“The Sino-Tibetan tonogenetic laryngeal reconstruction theory.”
\textit{LTBA} 5.1:49-127.

\refitem
%: 
\rrefline % Weidert, Alfons K.
1981.
“Star, moon, spirits, and the affricates of Angami Naga: a reply to James A.\ Matisoff.”
\textit{LTBA} 6.1:1-38.

\refitem
%: 
\rrefline % Weidert, Alfons K.
1987.
\textit{Tibeto-Burman Tonology: a comparative account.}
\textit{Current Issues in Linguistic Theory} \#54. Amsterdam and Philadelphia: John Benjamins.

\refitem
%:
Wolfenden, Stuart N.
1929.
\textit{Outlines of Tibeto-Burman Linguistic Morphology}.
With special reference to the prefixes, infixes, and suffixes of Classical Tibetan, and the languages of the Kachin, Bodo, Naga, Kuki-Chin, and Burma groups. Prize Publication \#12. London: Royal Asiatic Society.

\refitem
%: 
Yu Nae-wing \TC{余迺永}.
2000.
\TC{新校互註‧宋本廣韻} \textit{Xīn Jiào Hù Zhù - Sòng Běn Guǎngyùn [A New Revision of the Sung Edition of the Kuang-yun Rhyming Dictionary].}
\TC{上海辭書 出版社} Shànghǎi Císhū Chūbǎnshè.

\label{pg:end-refs}

\cleartooddpage[\thispagestyle{empty}]
