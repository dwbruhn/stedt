{\large \parindent=-1em \textbf{Introduction}}
\addcontentsline{toc}{chapter}{Introduction}
\vspace{0.25em}

\section{Introduction}
This Front Matter includes material that has been reworked from a number of previous publications and oral presentations: HPTB, TBRS, VSTB, ICSTLL 46.

The Sino-Tibetan (ST) language family, comprising Chinese on the one hand,
and the hundreds of Tibeto-Burman (TB) languages on the other, is one of the
largest in the world, with well over a billion and a half
speakers.\footnote{Some scholars, especially in China,
consider Sino-Tibetan to include the
Tai-Kadai (TK) and Hmong-Mien (HM) (=Miáo-Yáo) language families as well.  While
there is definitely a striking typological similarity among Chinese, TK, and HM,
this is undoubtedly due to prolonged ancient contact rather than genetic
relationship.  See Benedict 1975a (\textit{Austro-Thai Language and Culture, with a
glossary of roots}).}  Yet the field of ST linguistics is only about 70 years
old, and many TB languages remain virtually unstudied.  The \textit{Sino-Tibetan
Etymological Dictionary and Thesaurus} project (STEDT) was begun in August 1987,
with the goal of reconstructing the lexicon of Proto-Sino-Tibetan and
Proto-Tibeto-Burman from both the phonological and the semantic point of view.

The great Sino-Tibetan language family, comprising Chinese on the one hand and Tibeto-Burman (TB) on the other,\footnote{Many scholars, especially in China, interpret ``Sino-Tibetan'' to include the Tai and Hmong-Mien families as well, though a consensus is developing that these latter two families, while possibly related to each other, have only an ancient contact relationship with Chinese (Benedict 1975a; JAM 1991a:486-90).} is comparable in time-depth and internal diversity to Indo-European, and equally important in the context of world civilization. The overwhelming cultural and numerical predominance of Chinese is counterbalanced by the sheer number of languages (some 250-300) in the TB branch.

After the existence of this vast and ramified family of languages was posited in the mid-19th century, British scholars and colonial administrators in India and Burma began to study some of the dozens of little-known ``tribal'' languages of the region that seemed to be genetically related to the two major literary languages, Tibetan and Burmese. This early work was collected in the monumental Linguistic Survey of India (Grierson and Konow 1903-28), three sections of which (Vol. III, Parts 1,2,3) are devoted to wordlists and brief texts from TB languages.

Further significant progress in TB studies had to wait until the late 1930's, when the eccentric amateur comparativist Robert Shafer headed a Depression-era project called ``Sino-Tibetan Linguistics'', sponsored by the eminent anthropologist A.L. Kroeber of U.C. Berkeley.\footnote{For a readable and humorous account of this project, see Benedict 1975b (LTBA 2.1:81-92).} With admirable thoroughness, the project staff assembled all the lexical material then available on TB languages, enabling Shafer to venture a detailed subgrouping of the family at different taxonomic levels, called (from higher to lower) divisions, sections, branches, units, languages, and dialects. This work was finally published piecemeal in a two-volume, five-part opus called Introduction to Sino-Tibetan (1966-67; 1974).

Shafer's junior collaborator Paul K. Benedict based his own work on the same body of material as Shafer, but achieved much more usable results. In an unpublished manuscript entitled Sino-Tibetan: a Conspectus (ca. 1942-43; henceforth STC), Benedict adopted a more modest approach to supergrouping and subgrouping than Shafer, stressing that many TB languages had so far resisted precise classification. While Shafer had included Tai in Sino-Tibetan, Benedict (1942) banished it from the family altogether, relating Tai instead to Austronesian.\footnote{To this putative megalolinguistic grouping, later to include Hmong-Mien and Japanese as well as Tai-Kadai and Austronesian, Benedict gave the name ``Austro-T(h)ai'' (see Benedict 1975a, 1990).} Shafer's pioneering work, valuable as it was, suffered from his mistrust of phonemics, with a consequent proliferation of pseudo-precise and arcane phonetic symbols. Benedict's structural insight  his flair for isolating that which is crucial from masses of data  enabled him to formulate sound correspondences with greater precision, and to distinguish between regular and exceptional phonological developments.

The publication of a revised and heavily annotated version of STC in 1972, with J. Matisoff as contributing editor, laid the foundations for modern Sino-Tibetan historical/ comparative linguistics. In this recension, nearly 700 Proto-Tibeto-Burman (PTB) roots were reconstructed (491 of them in numbered cognate sets, with about 200 more scattered throughout the text and footnotes), as well as some 325 comparisons of PTB roots with Old Chinese etyma, largely as reconstructed by Karlgren (1957). While Benedict focussed principally on five key, phonologically conservative TB languages (Tibetan, Burmese, Lushai [=Mizo], Kachin [=Jingpho], Garo), he also used data from more than 100 others, judiciously making allowances for inadequacies of transcription where necessary.\footnote{In a recently published work, Peiros and Starostin (1996) follow Benedict' s example in their choice of key TB languages, basing their Sino-Tibetan reconstructions on Written Tibetan, Written Burmese, Lushai, Jingpho, and Chinese, all of which are treated as if they belonged on the same taxonomic level. See the discussion in Handel (1998, Ch. 3).} the moment of writing (September, 1997) marks the 30th anniversary of the publication of STC in 1972. The recent tragic death of Benedict in a car accident (July 21, 1997) makes this a particularly appropriate time to take stock. How well has STC stood the test of time? The short answer is: remarkably well. The work has been reviewed about 15 times, almost always in a highly favorable tone,\footnote{A notable exception is the intemperate review by Miller (1974), which bitterly criticizes the fact that the notes added in 1972 sometimes modify points made in the original text (ca. 1942). See the defense of STC against Miller's attack by JAM (1975a).} and has been translated into Chinese.\footnote{See Le Saiyue and Luo Meizhen 1984.}In fact nearly all 700 of the TB cognate sets in STC have been shown to be perfectly valid, though many of the reconstructions have had to be changed slightly in the light of new data, and in a couple of cases etyma which had been reconstructed separately have been shown to be variant forms (``allofams'') of the same word-family.\footnote{E.g. *dyam \& *tyam [STC $\#$226] `full; fill' and *dyam [STC $\#$227] `straight' ; see JAM 1988a.}

\section{Original plan of STEDT}
As originally conceived, STEDT was to produce a series of large
print volumes, each devoted to the exhaustive presentation of the reconstructed
roots in a specific semantic area, covering the entire lexicon, approximately as
follows:

\begin{quote}
{\footnotesize
Volume I: \textit{Body Parts}\nopagebreak[4]\\
Volume II: \textit{Animals}\\
Volume III: \textit{Natural Objects, Plants, Foods }\\
Volume IV: \textit{Kinship Terms, Ethnonyms, Social Roles}\\
Volume V: \textit{Culture, Artifacts, Religion}\\
Volume VI: \textit{Verbs of Motion, of Manipulation, and of Production}\\
Volume VII: \textit{Adjectival Verbs}\\
Volume VIII: \textit{Abstract Nouns and Verbs, Psychological Verbs, Verbs of Utterance }\\
Volume IX: \textit{Shape, Size, Color, Measure, Number, Time, Space}\\
Volume X: \textit{Grammatical words}\\
}
\end{quote}

Each volume was in turn to be divided into a number of smaller units called
“fascicles”. Thus Vol.~I \textit{Body Parts} was to comprise the following nine
fascicles:

\begin{quote}
{\footnotesize
1. \textit{Body (general)}\\
2. \textit{Head and Face}\\
3. \textit{Mouth and Throat}\\
4. \textit{Torso}\\
5. \textit{Limbs, Joints, and Body Measures}\\
6. \textit{Diffuse Organs}\\
7. \textit{Internal Organs}\\
8. \textit{Secretions and Somatophonics}\footnote{By “somatophonics” I mean sneezes,
belches, farts, and the like.}\\
9. \textit{Reproductive System}\\
}
\end{quote}

Every subpart and sub-subpart of the lexicon expanded and bloomed into a major project. Concentrated on NBP's --- TBRS was fascicle 9 of Vol. I.

%\section{preface}
%[[INSERT HPTB PREFACE, pp. vii-xiv]]

\begin{itemize}
\item The end/culmination of the Sino-Tibetan Etymological Dictionary and Thesaurus project (STEDT), 1987-2014.
\item Lexical database of nearly 500,000 forms from ca.\ 300 TB languages.
\item Thousands of reconstructed roots, both at Proto-Tibeto-Burman and subgroup levels, along with Chinese comparanda.
\item A monograph series. HPTB, TBRS, ELL.
\item Final product: a printed volume of several hundred pages (only a few copies for libraries!), to be made generally available electronically.

\section{Phonological vs.\ semantic criteria}

Buck and Roget

Comparison with HPTB: In HPTB the etyma were discussed, sorted and analyzed according to their phonological shapes, regardless of their meanings.

Comparison with VSTB and TBRS: phonologically disparate etyma assembled according to their meanings. This is also the strategy of our Root Canal extractions:
HAND/ARM/WING; SKIN; ZODIAC.

This is also the strategy of this Final Product

\section{Methodology}
The ``tagging'' process. The Lahu trisyllable for NAVEL.

\section{Terminology}

[[InSERT JAM PROSE]]

\section{subgrouping}

[[INSERT JIM'S PROSE]]

\subsection{Subgroup names}


Tibeto-Burman is an extremely complex language family, with great internal
typological diversity, comparable to that of modern Indo-European.  This
diversity is due largely to millennia of language contact, especially with the
prestigious cultures of India and China,\footnote{I have called the Indian and
Chinese areas of linguistic and cultural influence the \textit{Indosphere} and the
\textit{Sinosphere}. See Matisoff 1973.} but also with the other
great language families of Southeast Asia (Austroasiatic, Tai-Kadai,
Hmong-Mien), as well as with other TB groups.  We are thus faced with what I
have described as “an interlocking network of fuzzy-edged clots of languages,
emitting waves of mutual influence from their various nuclear ganglia.  A mess,
in other words.”\footnote{Matisoff 1978 (\textit{VSTB}), p.~2.}  While subgrouping such
a recalcitrant family is difficult, there is certainly no need to go so far as
van Driem by denying that TB subgroups exist at all, or by claiming that even if
they do exist, there are so many of them that there is no point in talking about
them!\footnote{See his review (2003) of G. Thurgood \& R.J. LaPolla, eds. (2003),
\textit{The Sino-Tibetan Languages}.}


In the published version of \textit{STC} (1972),
P.\ K.\ Benedict wisely refrained from offering a pseudo-precise family-tree model of
the higher-order taxonomic relationships in TB, presenting instead a schematic
chart where Kachin (=~Jingpho) was conceived as the center of geographical and
linguistic diversity in the family.  See Fig.~1.

\begin{figure}[ht]
\XeTeXpdffile "intro231.pdf"  width \textwidth
\begin{center}
\textit{Figure 1. Schematic Chart of Sino-Tibetan Languages}\footnotemark
\end{center}
\end{figure}
\footnotetext{Reproduced from \textit{STC}, p.~6;
\textit{VSTB}, p.~3; \textit{HPTB}, p.~4.}

A simpler scheme represents the heuristic model now used at STEDT.  See Fig.~2.

\begin{figure}[ht]
\XeTeXpdffile "intro232.pdf"  width \textwidth
\begin{center}
\textit{Figure 2.  Simplified STEDT Family Tree of ST Languages}
\end{center}
\end{figure}

This diagram differs from \textit{STC} in several respects:\footnote{See \textit{HPTB}, pp.~5-6.}
\begin{itemize}
\item Karenic is no longer regarded as having a special status, but is now
considered to be a subgroup of TB proper.
\item Baic, hardly mentioned (under the name “Minchia”) in \textit{STC}, but later
hypothesized by Benedict to belong with Chinese in the “Sinitic” branch of 
Sino-Tibetan, is now also treated as just another subgroup of TB, though one
under particularly heavy Chinese contact influence. Both Karenic and Baic have
SVO word order, unlike the rest of the TB family.
\item The highly ramified Kuki-Chin and Naga groups have provisionally been
amalgamated with Bodo-Garo (=Barish) and Abor-Miri-Dafla (=Mirish) into a
supergroup called by the purely geographical name of \textit{Kamarupan}, from the old
Sanskrit name for Assam.
\item The important Tangut-Qiang languages (deemed to include rGyalrong
[=Gyarung =Jiarong] and the extinct Xixia [=Tangut]) were hardly known to Western
scholars at the time \textit{STC} was written (ca.~1942-3) or published (1972). It seems
doubtful that a special relationship exists between Qiangic and Jingpho, or
between Qiangic and Lolo-Burmese, as some Chinese scholars maintain.\footnote{A
supergroup called “Rung” was proposed by Thurgood (1984), into which he placed,
among others, some Qiangic languages, Nungish, and Lepcha.  This grouping was
based partly on shared “proto-morphosyntax”, and partly on nomenclature,
including the \textit{-rong} of \textit{rGyalrong},
the Nungish language \textit{Rawang}, and the Lepcha autonym \textit{Rong}.}
\item The Nungish and Luish languages are grouped with Jingpho (=Kachin).  Jingpho
is also recognized to have a special contact relationship with the Northern Naga
(=Konyak) group.\footnote{The \textit{Linguistic Survey of India}
(Grierson and Konow, 1903-28) recognized a “Bodo-Naga-Kachin” group,
an idea revived by Burling
(1983), whose “Sal” supergroup comprises Bodo-Garo (Barish), Northern Naga
(Konyak), and Jingpho (=Kachin).  Burling’s name for this grouping is derived
from the etymon \textbf{*sal} ‘sun’ (ult.\ < PTB \textbf{*tsyar} ‘sunshine’), one of a number of
roots which is attested chiefly in these languages.  See \textit{HPTB}:393-4.}
\item The somewhat idiosyncratic Mikir, Meithei (=Manipuri), and Mru languages are
included under Kamarupan.
\item The Himalayish (=Himalayan) group is considered to include Bodic (i.e.\
\mbox{Tibetanoid}) languages, as well as Kanauri-Manchad, Tamang-Gurung-Thakali,
Kiranti (=Rai), Lepcha, and Newar.
\item The relatively well-studied Lolo-Burmese group (= \textit{STC}’s “Burmese-Lolo”) is
deemed to include the aberrant Jinuo language of Xishuangbanna,
Yunnan.\footnote{Chinese scholars have further divided the Loloish languages of
China into six nuclei, although no attempt is made in this volume to distinguish
them.  In a recent talk (Matisoff 2007b) I examined Loloish tonal developments
and the fate of the PLB rhyme \textbf{*-a} in terms of this six-way grouping, with
inconclusive results.}  The Naxi/Moso language is quite close to LB, but stands
somewhat outside the core of the family.\footnote{I have grouped Naxi with
Lolo-Burmese proper in a supergroup called “Burmo-Naxi-Lolo” (Matisoff 1991c). 
On the basis of some shared tonal developments, I have also entertained the idea
of a special relationship between Lolo-Burmese and Jingpho, to which I assigned
the jocular designation \textit{Jiburish} (<~\textbf{Ji-}(ngpho)
+ \textbf{-bur}(mish) + (Lolo)\textbf{ish}).  See
Matisoff 1974, 1991c.}
\item The mysterious Tujia language of Hunan and Hubei (not mentioned in \textit{STC}) has so
far not been assigned to a subgroup.
\end{itemize}


Still, a schema like Fig.~2 hardly does justice to the complexity
of the problem of subgrouping the TB languages.  In particular, the “Kamarupan”
and “Himalayish” groupings are based more on geographical convenience than on
strong constellations of similar characteristics.\footnote{Several scholars have objected to the term Kamarupan,
largely on the grounds that it has distinctly Indo-Aryan connotations, which
might irritate TB groups.  See, e.g.\ R.~Burling, “On \textit{Kamarupan}” (1999; \textit{LTBA}
22.2:169-71), and the reply by Matisoff, “In defense of \textit{Kamarupan}”
(1999; \textit{LTBA} 22.2:173-82).  The only alternative term
suggested so far to refer to these
geographically contiguous languages collectively is the verbose “TB languages of
Northeast India and adjacent areas”.}
More detailed subgroupings are certainly possible, as in
STEDT Monograph \#2,\footnote{J.~Namkung, ed. (1996),
\textit{Phonological Inventories of Tibeto-Burman Languages},
pp.~455-457.}
which makes distinctions like the following:


\begin{quote}
\textit{Kamarupan}\\
- Abor-Miri-Dafla (=Mirish)\footnote{A well-defined subgroup of AMD has been
dubbed \textit{Tani} by J. Sun (1993).}\\
- Kuki-Chin\\
- Naga\\
\hspace*{3ex}· Konyak (=Northern Naga)\\
\hspace*{3ex}· Angamoid\\
\hspace*{3ex}· Central\\
\hspace*{3ex}· Eastern\\
\hspace*{3ex}· Southern\\
\hspace*{3ex}· Southwestern\\
- Meithei\\
- Mikir\\
- Mru\\
- Bodo-Garo (=Barish)\\
- Chairel

\textit{Himalayish}\\
- Western (Bunan, Kanauri, Manchad/Pattani)\\
- Bodic (Tibetanoid)\\
- Lepcha\\
- Tamangic (incl. Chantyal, Gurung, Tamang, Thakali, Manang, Narphu)\\
- Dhimalish\\
- Newar\\
- Central Nepal Group (Kham, Magar, Chepang, Sunwar)\\
- Kiranti (=Rai), including Bahing and Hayu\\
\end{quote}

\subsection{Scope and subgrouping of the TB family}

The exact number of TB languages is impossible to determine, not only because of the elusiveness of the distinction between ``languages'' and ``dialects'' , and the fact that a number of languages remain to be discovered and/or described, but especially because of the profusion and confusion of different names for the same language.\footnote{See JAM 1986a, and STEDT Monograph II (JAM 1996a).} At the present state of our knowledge we can estimate that the Tibeto-Burman family contains approximately 250 languages, which may be broken down into population categories as indicated in Table 1: there are 9 TB languages with over a million speakers (Burmese, Tibetan, Bai, Yi (=Lolo), Karen, Meithei, Tujia, Hani, Jingpho), and altogether about 50 with more than 100,000 speakers; at the other end of the scale are some 125 languages with less than 10,000 speakers, many of which are now endangered (JAM 1991b). Though much of the geographical area covered by TB languages has been chronically inaccessible to fieldwork by scholars from outside,\footnote{Very approximately, the distribution of TB languages by country is as follows: India 107, Burma 75, Nepal 69, China 50, Thailand 16, Bangladesh 16, Bhutan 9, Laos 8, Vietnam 8, Pakistan 1.} there has been a recent explosion of new data, especially from China\footnote{Among the most valuable of these new sources are Sun Hongkai, Xu Jufang et al. (ZMYYC; 1991), containing 1004 synonym sets in 52 languages and dialects; and Dai Qingxia and Huang Bufan (TBL; 1992), with 1822 synonym sets in 50 languages and dialects.} and Nepal.

As far as subgrouping this unruly conglomerate of languages goes, Benedict wisely refrained from constructing a family tree of the conventional type, presenting instead a schematic chart where Kachin (=Jingpho) was conceived as the center of geographical and linguistic diversity in the family. See Figure 1:Schematic Stammbaum of Sino-Tibetan Languages [STC, p. 6] The genetic schema now being used heuristically at the STEDT project differs from this in several respects.\footnote{The STEDT project' s working hypotheses regarding the subgrouping of individual languages may be  found in the indices to STEDT Monograph III (J. Namkung, ed. 1996:455-7). } See Figure 2:Provisional STEDT Family Tree
\begin{itemize}
\item Karenic is no longer regarded as having a special status, but is now considered to be a subgroup of TB proper.\item Baic, hardly mentioned (under the name ``Minchia'' ) in STC , but later hypothesized by Benedict to belong with Chinese in the ``Sinitic'' branch of Sino-Tibetan, is now also treated as just another subgroup of TB, though one under particularly heavy Chinese contact influence. Both Karenic and Baic have SVO word order, unlike the rest of the TB family. \item The highly ramified Kuki-Chin-Naga group has provisionally been amalgamated with Bodo-Garo (=Barish) and Abor-Miri-Dafla (=Mirish) into a supergroup called by the purely geographical name of Kamarupan, from the old Sanskrit name for Assam.\footnote{Issue has been taken with this term by Burling (1999), but see the reply by JAM (1999c).} \item The important Qiangic languages (deemed to include rGyalrong [=Gyarung=Jiarong] and the extinct Xixia [=Tangut]) were hardly known to non-Western scholars at the time STC was written (ca. 1942-3) or published (1972). It seems doubtful that a special relationship exists between Qiangic and Jingpho, or between Qiangic and Lolo-Burmese, as many Chinese scholars maintain. \item The Nungish and Luish languages are grouped with Jingpho (=Kachin).\footnote{The obscure Luish group, also known as Kadu-Andro-Sengmai, includes a few languages spoken by groups that were once exiled to a remote corner of NE India by the Rajah of Manipur. See Grierson 1921.} Jingpho is also recognized to have a special contact relationship with the Northern Naga (=Konyak) group. \item The somewhat idiosyncratic Mikir, Meithei (=Manipuri), and Mru languages are included under Kamarupan. \item The Himalayish (=Himalayan) group is considered to include Bodic (i.e. Tibetanoid) languages, as well as Kanauri-Manchad, Kiranti (=Rai), Lepcha, and Newar. \footnote{As part of a recent trend to purge TB language names of Indo-Aryan suffixes, specialists in Himalayish languages are no longer using the name ``Newari'' for this language, but rather ``Nepal Bhasha'' or simply ``Newar'' . Similarly, the language known formally as Magari is now preferably referred to as ``Magar.''}\footnote{Various other subgroupings have been proposed, e.g. ``Rungic'' (Thurgood 1984) and ``Sino-Bodic'' (van Driem 1997). See a critique of the latter by JAM (2000b).}
\end{itemize}

\subsection{Typological diversity of TB: Indosphere and Sinosphere}

The TB family, which extends over a huge geographic range, is characterized by great typological diversity, comprising languages that range from the highly tonal, monosyllabic, analytic type with practically no affixational morphology (e.g. Loloish), to marginally tonal or atonal languages with complex systems of verbal agreement morphology (e.g. the Kiranti group of E. Nepal). While most TB languages are verb-final, the Karenic and Baic branches are SVO, like Chinese.
 
This diversity is partly to be explained in terms of areal influence from Chinese on the one hand, and Indo-Aryan languages on the other. It is convenient to refer to the Chinese and Indian spheres of cultural influence as the ``Sinosphere'' and the ``Indosphere''.\footnote{See JAM 1990a (``On megalocomparison.'')} Some languages and cultures are firmly in one or the other: e.g. the Munda and Khasi branches of Austroasiatic, the TB languages of Nepal, and much of the Kamarupan branch of TB (notably Meithei = Manipuri) are Indospheric; while the Hmong-Mien family, the Kam-Sui branch of Kadai, the Loloish branch of TB, and Vietnamese (Mon-Khmer) are Sinospheric. Others (e.g. Thai and Tibetan) have been influenced by both Chinese and Indian culture at different historical periods. Still other linguistic communities are so remote geographically that they have escaped significant influence from either cultural tradition (e.g. the Aslian branch of Mon-Khmer in Malaya, or the Nicobarese branch of Mon-Khmer in the Nicobar Islands of the Indian Ocean).

Elements of Indian culture, especially ideas of kingship, religions (Hinduism/Brahminism, Buddhism), and {\it devan\=agar\={\i}} writing systems, began to penetrate both insular and peninsular Southeast Asia about 2000 years ago. Indic writing systems were adopted first by Austronesians (Javanese and Cham) and Austroasiatics (Khmer and Mon), then by Tai (Siamese and Lao) and Tibeto-Burmans (Pyu, Burmese, and Karen). the learned components of the vocabularies of Khmer, Mon, Burmese, and Thai/Lao consist of words of Pali/Sanskrit origin. Indian influence also spread north to the Himalayan region. Tibetan has used {\it devan\=agar\={\i}} writing since A.D. 600, but has preferred to calque new religious and technical vocabulary from native morphemes rather than borrowing Indic ones.

What is now China south of the Yangtze did not have a considerable Han Chinese population until the beginning of the current era (Ramsey 1987, Norman 1988). In early times the scattered Chinese communities of the region must have been on a numerical and cultural par with the coterritorial non-Chinese populations, with borrowing of material culture and vocabulary proceeding in all directions (Benedict 1975; Mei and Norman 1976; Sagart 1990). As late as the end of the first millennium A.D., non-Chinese states flourished on the periphery of the Middle Kingdom (Nanchao and Bai in Yunnan, Xixia in the Gansu/Qinghai/Tibet border regions, Lolo (Yi) chieftaincies in Sichuan. The Mongol Yuan dynasty finally consolidated Chinese power south of the Yangtze in the 13th century. Tibet also fell under Mongol influence then, but did not come under complete Chinese control until the 18th century.

Whatever their genetic affiliations, the languages of the East and SE Asian area have undergone massive convergence in all areas of their structure  phonological, grammatical, and semantic.\footnote{An excellent recent study of such phenomena is Thomason and Kaufman 1988.}  Hundreds of words have crossed over genetic boundaries in the course of millennia of intense language contact, so that it is often exceedingly difficult to distinguish ancient loans from genuine cognates.

Teleo- and meso-reconstructionsthe current state of comparative/historical TB research is quite uneven. While some branches of the family are relatively well studied, to the point where ``mesolanguages'' have been reconstructed at the subgroup level,\footnote{See, e.g. Proto-Karen (Haudricourt 1942-5, 1975; Jones 1961; Burling 1969; Solnit, in prep.); Proto-Bodo (Burling 1959); Proto-Lolo-Burmese (Burling 1968, JAM 1969, 1972a; Bradley 1978); Proto-Tamang-Gurung-Thakali (Mazaudon 1978); Proto-Kiranti (Michailovsky 1991); Proto-N.-Naga (W. French 1983); Proto-Tani [Mirish] (J.T. Sun 1993).} large gaps remain  we have nothing approaching well-worked out reconstructions for such key subgroups as Qiangic, Baic, Luish, and Nungish. Still unclear is the exact genetic position of many transitional languages like Chepang, Kham, Lepcha, Newar (all lumped currently with ``Himalayish''), or Meithei, Mikir, Mru (close to the Kuki-Chin-Naga branch), or Naxi/Moso and Jinuo (close to Lolo-Burmese), or the mysterious Tujia of Hunan/Hubei. The position of the crucially important Jingpho language is undergoing reevaluation, with current opinion returning to the notion of a special relationship with the Bodo-Garo-Konyak group (Burling 1971, Weidert 1987).\footnote{Cf. the volume of Grierson and Konow (1903-28) called ``Bodo-Naga-Kachin.'' Elsewhere (JAM 1974, 1991c) I have discussed the pros and cons of lumping Jingpho and Lolo-Burmese together into a supergroup facetiously called ``Jiburish'' (Jingpho-burmish-Loloish).} It remains to be seen whether the large ``Kamarupan'' (NE India) and ``Himalayish'' groups are anything more than purely geographic divisions of the family, and if so what the internal relationships among their many parts might be.

Although it remains true that ``supergroups within TB cannot safely be set up at the present level of investigation'' (STC, p. 11), the same can be said of Indo-European (IE) after nearly 200 years of scholarly investigation. Thus while it is obvious that the closely related Baltic and Slavic languages constitute a valid IE supergroup, ``Balto-Slavic'' (just as, e.g. the Loloish and Burmish languages clearly group together as ``Lolo-Burmese''), higher order IE lumpings (e.g. ``Italo-Celtic'', ``Italo-Germanic'', ``Italo-Greek'') remain highly controversial, since patterns of shared innovations, or overlapping features of special resemblance, may be found between virtually any two major subgroups of the family.\footnote{See the discussion in JAM (VSTB) 1978a:3-12.}Meso-level reconstruction per se is not one of the goals of the STEDT project; nor does the project's reconstruction of PTB depend strictly on the direct comparison of meso-level reconstructions. However, such reconstructions are used when available in reconstructing roots at the Proto-Tibeto-Burman level. We therefore treat meso-level proto-forms as lexical data records, just like attested forms in individual languages. I follow Benedict in caring little for a chimerical methodological purity in this respect, and generally endorse his philosophy of ``teleoreconstruction'', by which salient characteristics of the proto-language may be deduced by inspection of attested forms in well-chosen languages from different subgroups, thereby ``leap-frogging'' the need for step-wise reconstruction.\footnote{This method must of course be applied with due caution, and I feel that Benedict applied it too loosely with respect to the vexed question of the existence of a reconstructible tonal system at the PTB level. See e.g. Benedict 1973 (``Tibeto-Burman tones, with a note on teleo-reconstruction'').} This in fact has been the only practical methodology for reconstructing TB given the uneven state of our present knowledge. It goes without saying that one's teleo-hypotheses are subject to constant revision in the light of new data at the level of individual languages or subgroups. As in all scientific inquiry, the process of formulating falsifiable hypotheses lies at the heart of the reconstructive enterprise. I feel that it is perfectly justifiable to ``take a peek'' outside a given subgroup in order to help one choose between alternative reconstructions that might be equally plausible on the basis of intra-group evidence alone.\footnote{Many of the features of W. French's excellent reconstruction of Proto-N.-Naga (1983) were motivated by extra-Naga evidence.} It is for this reason that TB evidence will prove to be so crucial in evaluating the multitude of competing reconstructions of Old Chinese.

[[INSERT JAM PROSE]]

\subsection{Language names}


Tibeto-Burman languages are notorious for the multiplicity of names by which
they are referred to. These may include the name they use for themselves
(autonym), as opposed to the name(s) other groups use for them (exonyms). 
Languages are frequently referred to by the principal town in which they are
spoken (loconyms).  Some exonyms are now felt to be pejorative, and have been
abandoned, thus acquiring the status of “paleonyms” for which “neonyms” have
been substituted.\footnote{The terminology for the various types of TB language
names was developed in Matisoff 1986a: “The languages and dialects of
Tibeto-Burman: an alphabetic/genetic listing, with some prefatory remarks on
ethnonymic and glossonymic complications.”  In John McCoy and Timothy Light,
eds., \textit{Contributions to Sino-Tibetan Studies},  pp.~1-75.  This article was later
(1996) expanded into a STEDT Monograph, with the assistance of J.B. Lowe and
S.\ P.\ Baron.}  A certain Angamoid Naga group call themselves and their language
\textit{Memi} (autonym), and their chief village they call \textit{Sopvoma};
but other groups use
\textit{Mao} for this village or its people (exonym), and either \textit{Mao}
or \textit{Sopvoma} (exonymic
loconym) for their language.  There is an older term \textit{Imemai} (probably an
autonymic paleonym) which refers to the same language and people.


Some names are used in both a broader and a narrower sense, both for a
specific language and for a group of languages that share a close contact
relationship.  The Maru, Atsi, and Lashi\footnote{Referred to as Langsu, Zaiwa,
and Leqi in Chinese sources.} (who speak Burmish languages) consider themselves
to be “Kachin” in the broad sense, and in this the Jingpho themselves seem to
agree, even though the Jingpho language belongs to a different TB subgroup.


In recent years cultural sensitivities have forced the abandonment of many
language names that had been well established in the academic literature.  The
important Central Chin language that used to be called \textit{Lushai} (a name which is
said to mean “long-headed”) should now properly be called \textit{Mizo}.
A Karenic group
that used to be known by the Burmese exonym \textit{Taungthu} (lit.\ “mountain folk”) now
prefers their autonym \textit{Pa-o}.  The Southern Loloish people
formerly known by the Tai exonym \textit{Phunoi} (lit.\ “little people”) should now be
called by their autonym \textit{Coong}.  Speakers of several TB languages of Nepal now
object to the Indianized versions of their names with the Indo-Aryan \textbf{-i} suffix
(e.g.\ \textit{Newari, Magari, Sunwari}), and prefer to omit the suffix, even though this
can lead to ambiguity between the names of the people and their languages
(\textit{Newar, Magar, Sunwar}).  The psychological dimensions of these issues are often
as fascinating as they are paradoxical.  Chinese linguists now feel that the
term \textit{Lolo(ish)}, widely used outside of China,
is offensive, and insist that the
proper respectful term is \textit{Yi}, written with the character \TC{彝} ‘type of
sacrificial wine vessel’.  Yet this is only a recent substitution for the
homophonous character \TC{夷} ‘barbarian; savage group on the fringes of the Chinese
empire’.

[[LUISH > ASAKIAN?]]


Naturally enough, what is true for the names of individual languages is also
true for the names of subgroups.  Some of this nomenclatural variation goes back
to differences between Benedict and his former collaborator and supervisor
Robert Shafer,\footnote{Shafer and Benedict collaborated on the Depression-era
\textit{Sino-Tibetan Linguistics} project at Berkeley (1939-40), which aimed to assemble
all data then available on TB languages.  The direct fruits of this project were
Shafer’s \textit{Introduction to Sino-Tibetan} (1967-73), 5 vols. (Wiesbaden: Otto
Harrassowitz) and the MS of Benedict’s \textit{STC}.  Benedict produced (1975) an
entertaining account of this seminal project in LTBA 2.1:81-92: “Where it all
began: memories of Robert Shafer and the \textit{Sino-Tibetan Linguistics} project,
Berkeley (1939-40).”} e.g.\ Shafer’s \textit{Barish} and \textit{Mirish} are the same as Benedict’s
\textit{Bodo-Garo} and \textit{Abor-Miri-Dafla}, respectively.  An important group of at least a
dozen TB languages spoken in East Nepal is known either as \textit{Kiranti} or
\textit{Rai}.\footnote{According to K.~P.~Malla (p.c.~2008), “\textit{Kirãt} is a loose label in Old Indo-Aryan for the cave-dweller, attested in late Vedic texts as well as in the \textit{Mahābhārata}.” Rai is “a Nepali word, linked to IA \textit{raaya} ‘lord’, given to the Khambu chiefs by the Gorkhali rulers in the late 18th century.”}
  An extreme example of proliferation is furnished by the well-established and
non-controversial group I call Lolo-Burmese, which has also been referred to as
Burmese-Lolo, Yi-Burmese, Burmese-Yi, Burmese-Yipho, Yipho-Burmese, Yi-Myanmar,
Myanmar-Yipho, etc.—and even Myanmar-Ngwi!


Bearing all these complicating factors in mind, an attempt has been made in
this volume to use maximally clear and consistent designations for the TB
languages and subgroups.


\section{Meso-Reconstructions}

Stepwise vs. teleo-reconstruction
Both techniques are indispensable. Benedict (1972) made brilliant use of “teleoreconstructive” methods by relying mainly on 5 criterial, phonologically conservative languages: Written Tibetan, Written Burmese, Jingpho, Lushai (Mizo), and Bodo. Teleoreconstruction sometimes involves peeking outside a subgroup in order to come up with a meso-reconstruction; i.e. making educated guesses. Thus final stops may sometimes be reconstructed for Proto-Karenic by looking elsewhere in TB. W.T. French (1983) used similar techniques in his reconstruction of some aspects of his Proto-Northern Naga. The main thing is to form precise and correctible hypotheses, i.e. hypotheses that can be tested.

In favorable circumstances (i.e. when the data are sufficient and the sound-laws are well enough known) we can reconstruct a given etymon at several subgroup levels. A few examples from our database: 

% \begin{tabular}{l l}
PIG & \\
\#1006: PTB *p?ak PIG & \\
PKC *wok PIG	PKC	Proto-Kuki-Chin \\
PNN *wak PIG 	PNN	Proto-Northern-Naga\\
TGTM *??wa PIG 	TGTM	Tamang/Gurung/Thakali/Manang\\
PLB *wak? PIG 	PLB	Proto-Lolo-Burmese\\
PKar *th???/ tháu?  PIG	PKar	Proto-Karenic\\
PBai *te PIG 	PBai	Proto-Baic\\
PTk *hwok PIG	  PTk	Proto-Tangkhulic\\

FOUR
\#2409: PTB *b-l?y FOUR
Proto-Tani *pri FOUR
PKC *lii FOUR
PNN *b? l?y FOUR
TGTM *?bli FOUR
PLB *b/?-l?y? FOUR
PKar *lwi-t FOUR

STONE
\#1269: PTB *r-lu? ? k-luk STONE
PTani *l?? STONE
PKC *lu? STONE / ROCK
PLB *k-lok ? k-lo? STONE
PTk *lu? STONE

TREE/WOOD
\#2658: PTB *si? ? sik TREE, WOOD
PTani *s?? WOOD / TREE
PKC *thi? TREE / WOOD
PTk *t?i? WOOD

EIGHT
\#2259: PTB *b-r-gyat ? (b-)g-ryat EIGHT
PKC *riat EIGHT
PLB *?-rit? EIGHT
PTk *??t EIGHT

ASHES
\#3514: PTB *hot ASHES
PKC *wut ASHES / DUST
Proto-Asakian *k-but ASHES	
PTk *hwot ASHES

[[SHOULD THIS GO HERE?]]

\subsection{Language contact: loanwords vs.\ cognates}

We include reconstructions of a number of Indo-Aryan roots, if they've been borrowed into a large number of TB languages (esp.\ in the Himalayas and NE India).
Even when the donor language is not related to the receptor TB language, there may well be an accidental phonological resemblance between the form in the donor language and a similar genuine TB root. To facilitate recognition of loanwords, we have adopted the practice of including e.g.\ Nepali roots (identified as such) in our list of etyma if they have been borrowed into several TB languages.

\subsection{Special challenges for Sino-Tibeto-Burman etymologization}

·Relative lack of anciently attested languages: only Chinese, Tibetan, Tangut (Xixia), Burmese, Meithei, Newar, and a few others.

·Monosyllabicity	—as opposed, e.g., to Austronesian, with juicy disyllabic roots (trisyllables if one follows Benedict’s “Austro-Tai” (1975).

·Unevenness of our knowledge with respect to the various subgroups of Tibeto-Burman
Subgrouping issues: 

·Phonologically eroded forms in many subgroups
This presents problems common to reconstruction in other mono- or sesqui-syllabic languages of mainland Southeast Asia (Hmong-Mien, Tai-Kradai, Mon-Khmer).
	
There are great differences among the subgroups with respect to degree of preservation of proto-entities. Contrast, e.g. Written Tibetan and Kuki-Chin on the one hand, to Loloish, Asakian, and Qiangic on the other. So it’s good to find languages which erode differently (e.g., Hmong preserves initials better, while Mien preserves finals better; Kadu preserves finals better while Sak preserves initials better). This differential preservation makes it exciting when you can show that WT brgyad and Lahu hí are perfect cognates for EIGHT.

Phonological erosion makes it tough to distinguish between phonologically similar etyma in the same semantic field:
		LOUSE *sar and *s-r(y)ik. What do you do with a form like ?? ?
		MIND/BREATH *sem and *sak
		MOUTH *mu?r and *muk
		FACE *s-ma?y and *s-mel
As an extreme example, we have reconstructed no fewer than 11 etyma meaning NECK/THROAT with initial velars or velar clusters:\footnote{This is perhaps further confirmation of the frequently noted tendency of words for these parts of the body to have “guttural” initials.} 1 #389 *s-gwa-n ? *r-gwa-n NECK/NAPE (probably related to #495 *kwa THROAT/NECK); #481 *s-ke-k ? *m-ke-k NECK; #486 *l-kok THROAT/NECK; #488 *k-ro THROAT; #489 *k/s-rok ? *k-ro? THROAT; #491 *gre-k THROAT; #493 *kak THROAT; #494 *ka THROAT/NECK; #3361 *go? NECK/THROAT; #5651 *ku NECK/THROAT.
	Some of these putative etyma may ultimately be combinable (e.g. #389 and #495, #481 and #491).\footnote{ This classic problem of distinguishing co-allofams from reflexes of separate roots has bedeviled Tibeto-Burmanists from the beginning. Benedict (1972) hesitated before assigning *g-yak and *s-rak ASHAMED to two separate roots (#452 and #431); similarly with *m-da and *b-la ARROW (pp. 111-112 and #449). I now believe that two separate roots I set up in GSTC (JAM 1985) actually reflect the same etymon, so I would now combine CATTLE/DOMESTIC ANIMAL *dzay (#129) with ELEPHANT/CATTLE *tsa?y (#143). In 1988 I tried to show that two homophonous roots set up in Benedict 1972 (*dyam STRAIGHT and *dyam FULL/FILL are actually one and the same root. For many similar examples, see JAM 2013.} It is necessary to constantly rectify one’s reconstructions in the light of new data (see below XYZ), and equally imperative to try to establish “sound laws” by finding parallel examples (see below XYZ).
	
	
	
\section{Sources and Source Abbreviations}

[[INSERT JAM TEXT]]

\subsection{Source abbreviations}

[[``true, in this volume, but not generally; even when identical]]

Each supporting form is ascribed to a particular source.  Many forms are
cited in more than one source in our database.  If the form is not identical in
different sources, we include them all.  This is especially useful in cases
where one or more of the sources might not be totally accurate phonemically, or
where subphonemic phonetic detail is provided. When the forms in different
sources are identical, the form only appears once, but there are multiple source
abbreviations, separated by commas.  Forms from well-studied languages
(e.g.\ Written Tibetan, Written Burmese, Jingpho) are likely to appear in several
sources used by STEDT.


The STEDT database contains forms from sources of many different kinds,
including:
\begin{itemize}
\item printed books, monographs, articles, especially dictionaries and grammars of
individual languages;
\item synonym lists (i.e.\ groups of forms from different languages with the same
meaning, but with no reconstructions provided), e.g.\ Luce 1986 (PPPB); Sun
Hongkai et al.\ 1991 (ZMYYC); Dai Qingxia, Huang Bufan et al.\ 1992 (TBL);
\item semantically based questionnaires solicited by STEDT from fieldworkers working
on particular languages;
\item monographs and treatises of an etymological nature, including works which
provide reconstructions at the subgroup level, e.g.:

\begin{verse}
Proto-Bodo: Burling 1959\\
Proto-Karenic: Haudricourt 1942-45/1975, Jones 1961, Burling 1969, Benedict~1972 (\textit{STC}), Solnit, in prep.\\
Proto-Kiranti: Michailovsky 1991\\
Proto-Kuki-Chin: VanBik 2003\\
Proto-Lolo-Burmese: Burling 1968, Matisoff 1969/1972, Bradley 1979\\
Proto-Northern-Naga: French 1983\\
Proto-Tamangic: Mazaudon 1978\\
Proto-Tani: Sun Tianshin 1993\\
\end{verse}
\end{itemize}


The abbreviations used in these source attributions are in general quite
transparent,\footnote{For a complete list of the source abbreviations that
appear in this volume, see the \textit{Appendix}.} e.g.:

\begin{quote}
\begin{tabular}{ll}
CK-YiQ	&Chen Kang, “Yi Questionnaire”\\
JZ-Zaiwa	&Xu Xijian, \textit{Outline Grammar (Jiǎnzhì) of Zaiwa}\\
AW-TBT	&A.~Weidert, \textit{Tibeto-Burman Tonology}\\
GHL-PPB	&G.~H.~Luce, \textit{Phases of Pre-Pagán Burma}\\
JAM:MLBM	&J.~A.~Matisoff, “Mpi and Lolo-Burmese microlinguistics”\\
EJAH:BKD	&E.~J.~A.~Henderson, \textit{Bwo Karen Dictionary}\\
\end{tabular}
\end{quote}

The abbreviation “JAM-Ety” refers to my own etymological notes compiled in the
pre-STEDT era, derived especially from older, classic sources.  These specific
sources can easily be tracked down from the \textit{Bibliography}.

\subsection{Supporting forms in the individual languages}

[[margin notes: ``paper is not a problem electronically''; maryama fix for absolutely identical ones]]

The forms which support the reconstructions are cited according to the
notation of the particular source.  Although this policy of “following copy”
often leads to redundancy (see 2.7 below), since one and the same form in a
given language may be transcribed in a variety of different ways,\footnote{Cf.\ the
multiple transcriptions of the Written Burmese form for BREAST/MILK under
\textbf{*s-nəw}, \#53 below:  \textbf{no¹}
(ZMYYC:281.39); \textbf{nuí} (AW-TBT:327; \textit{STC}:419); \textbf{núi}
(WSC-SH:48); \textbf{nuiʼ} (JAM-Ety; GEM-CNL; PKB-WBRD);
\textbf{nui.} (GEM-CNL); \textbf{nuiwʼ} (GHL-PPB).
 For these source abbreviations, see the \textit{Appendix}. Similarly, cf.\ the many slightly different forms from the Bodic and Tamangic groups that reflect the etymon \textbf{*tsaŋ} NEST/WOMB/PLACENTA (\#103 below).} it seems preferable to a
policy of “normalization”, which might have the effect of losing some phonetic
detail that is captured in one source but not in another.

\subsection{Glosses of the supporting forms}


In almost all cases, the gloss given in each particular source is preserved,
unless it is so awkward or misleading as to require emendation.  Even if the
glosses in consecutive records are identical, the gloss is repeated for each
individual record, instead of using a symbol like the “ditto-mark”.  


If a gloss is too long to fit onto a single line, it is “wrapped” so that
the additional lines are indented under the first one.


\section{Organization of the etymologies}

[[INSERT JAM PROSE]]

\subsection{Supporting forms in the individual languages}


The forms which support the reconstructions are cited according to the
notation of the particular source.  Although this policy of “following copy”
often leads to redundancy (see 2.7 below), since one and the same form in a
given language may be transcribed in a variety of different ways,\footnote{Cf.\ the
multiple transcriptions of the Written Burmese form for BREAST/MILK under
\textbf{*s-nəw}, \#53 below:  \textbf{no¹}
(ZMYYC:281.39); \textbf{nuí} (AW-TBT:327; \textit{STC}:419); \textbf{núi}
(WSC-SH:48); \textbf{nuiʼ} (JAM-Ety; GEM-CNL; PKB-WBRD);
\textbf{nui.} (GEM-CNL); \textbf{nuiwʼ} (GHL-PPB).
 For these source abbreviations, see the \textit{Appendix}. Similarly, cf.\ the many slightly different forms from the Bodic and Tamangic groups that reflect the etymon \textbf{*tsaŋ} NEST/WOMB/PLACENTA (\#103 below).} it seems preferable to a
policy of “normalization”, which might have the effect of losing some phonetic
detail that is captured in one source but not in another.

\subsection{Glosses of the supporting forms}


In almost all cases, the gloss given in each particular source is preserved,
unless it is so awkward or misleading as to require emendation.  Even if the
glosses in consecutive records are identical, the gloss is repeated for each
individual record, instead of using a symbol like the “ditto-mark”.  


If a gloss is too long to fit onto a single line, it is “wrapped” so that
the additional lines are indented under the first one.

\subsection{Source abbreviations}


Each supporting form is ascribed to a particular source.  Many forms are
cited in more than one source in our database.  If the form is not identical in
different sources, we include them all.  This is especially useful in cases
where one or more of the sources might not be totally accurate phonemically, or
where subphonemic phonetic detail is provided. When the forms in different
sources are identical, the form only appears once, but there are multiple source
abbreviations, separated by commas.  Forms from well-studied languages
(e.g.\ Written Tibetan, Written Burmese, Jingpho) are likely to appear in several
sources used by STEDT.


The STEDT database contains forms from sources of many different kinds,
including:
\begin{itemize}
\item printed books, monographs, articles, especially dictionaries and grammars of
individual languages;
\item synonym lists (i.e.\ groups of forms from different languages with the same
meaning, but with no reconstructions provided), e.g.\ Luce 1986 (PPPB); Sun
Hongkai et al.\ 1991 (ZMYYC); Dai Qingxia, Huang Bufan et al.\ 1992 (TBL);
\item semantically based questionnaires solicited by STEDT from fieldworkers working
on particular languages;
\item monographs and treatises of an etymological nature, including works which
provide reconstructions at the subgroup level, e.g.:

\begin{verse}
Proto-Bodo: Burling 1959\\
Proto-Karenic: Haudricourt 1942-45/1975, Jones 1961, Burling 1969, Benedict~1972 (\textit{STC}), Solnit, in prep.\\
Proto-Kiranti: Michailovsky 1991\\
Proto-Kuki-Chin: VanBik 2003\\
Proto-Lolo-Burmese: Burling 1968, Matisoff 1969/1972, Bradley 1979\\
Proto-Northern-Naga: French 1983\\
Proto-Tamangic: Mazaudon 1978\\
Proto-Tani: Sun Tianshin 1993\\
\end{verse}
\end{itemize}


The abbreviations used in these source attributions are in general quite
transparent,\footnote{For a complete list of the source abbreviations that
appear in this volume, see the \textit{Appendix}.} e.g.:

\begin{quote}
\begin{tabular}{ll}
CK-YiQ	&Chen Kang, “Yi Questionnaire”\\
JZ-Zaiwa	&Xu Xijian, \textit{Outline Grammar (Jiǎnzhì) of Zaiwa}\\
AW-TBT	&A.~Weidert, \textit{Tibeto-Burman Tonology}\\
GHL-PPB	&G.~H.~Luce, \textit{Phases of Pre-Pagán Burma}\\
JAM:MLBM	&J.~A.~Matisoff, “Mpi and Lolo-Burmese microlinguistics”\\
EJAH:BKD	&E.~J.~A.~Henderson, \textit{Bwo Karen Dictionary}\\
\end{tabular}
\end{quote}

The abbreviation “JAM-Ety” refers to my own etymological notes compiled in the
pre-STEDT era, derived especially from older, classic sources.  These specific
sources can easily be tracked down from the \textit{Bibliography}.

\subsection{Chinese comparanda}

After the evidence for a TB etymon is presented, one or more Chinese
comparanda are often suggested in the interests of pushing the reconstruction
further back to the Proto-Sino-Tibetan stage.  For all of these comparanda
Zev J.\ Handel has kindly provided comparisons of the Old Chinese reconstructions cited
in Karlgren’s (1957) system with those of Li Fang-kuei (1971, 1976, 1980) and
William Baxter (1992),\footnote{Handel also contributed a detailed comparison
of these systems in his \textit{A Concise Introduction to Old Chinese Phonology},
which appeared as Appendix A to \textit{HPTB}, pp.~543-74.}
evaluating the plausibility of the
putative TB/Chinese comparison according to each of these systems.\footnote{Handel  also frequently refers to several other reconstructive systems for OC that are to be found
in the literature, e.g.\ those of W. South Coblin (1986), Axel Schuessler (1987),
Laurent Sagart (1999), Gong Hwang-cherng (1990, 1994, 1995, 1997, 2000), and Pan
Wuyun (2000).}   Handel’s invaluable contributions are marked with his initials “ZJH”. 
Comparisons between TB and OC etyma that are not explicitly ascribed to a
particular scholar are original with me, as far as I know.

\subsection{Notes}

Footnotes may appear at virtually any point in the text. They may refer to an entire chapter, to a semantic diagram, to an etymon as a whole, to a specific supporting form, or to a Chinese comparandum.


