\vspace{0.25em}

\chapter*{Acknowledgments}
\addcontentsline{toc}{chapter}{Acknowledgments}

\renewcommand{\thefootnote}{\arabic{footnote}}
\setcounter{footnote}{0}

\section{NSF and NEH Support}

I am deeply grateful to the National Science Foundation and the National Endowment for the Humanities for their unswerving support of the {\it Sino-Tibetan Etymological Dictionary and Thesaurus} (STEDT) Project since 1987, even through times of budgetary stringency. I would especially like to thank Dr.\ Paul G.\ Chapin and Dr.\ Joan Maling, of the Language, Cognition, and Social Behavior division of NSF; and Dr.\ Guinevere Greist, Dr.\ Helen Ag\"uera, Dr.\ Martha Bohachevsky-Chomiak and Dr.\ Nadine Gardner of the Research Tools Division of NEH. I can only hope that the fruits of this project will repay their confidence and patience.

Grants to the \textit{Sino-Tibetan Etymological Dictionary and Thesaurus} project from:

\begin{itemize}
\item[*] The National Science Foundation (NSF), Division of
  Behavioral \& Cognitive Sciences, Grant Nos.\ BNS-86-17726,
  BNS-90-11918, DBS-92-09481, FD-95-11034, SBR-9808952,
  BCS-9904950, BCS-0345929,  BCS-0712570, and BCS-1028192.
\item[*] The National Endowment for the Humanities (NEH),
  Preservation and Access, Grant Nos.\ RT-20789-87, RT-21203-90,
  RT-21420-92, PA-22843-96, PA-23353-99, PA-24168-02,
  PA-50709-04,  PM-50072-07, and PW-50122-08.
\end{itemize}

\section{Administrators}

Several Organized Research Units and academic departments of the Berkeley campus have given their moral or practical support to the STEDT project, including the Center for Southeast Asia Studies, the Center for Chinese Studies, the Department of Linguistics, the Department of South and Southeast Asian Studies, the Department of East Asian Languages and Cultures, and especially the Institute of International and Area Studies, to whose administrative staff I am deeply obliged: Karin Beros, Management Services Officer and all-around trouble-shooter, who was instrumental in solving the practical problems of getting the project started back in 1987; Jerilyn C.\ Foush\'ee, who has handled our budget and helped with our grant proposals and reports since 1987; and Nell Haskell (1987–95) and Kerttu K.\ McCray (1995–2002), who have kept track of personnel matters.  Since 2009, the STEDT grants have been administered by the Linguistics Department, under the able stewardship of Paula Floro and Bel\'en Flores.

\section{Contributors}

I would like to thank Anthony Meadow, founder of the Bear River Associates software development firm, now in Oakland, who generously gave many hours of his time during 1986–87 in {\it pro bono} consultations about how to formulate the computer needs of the project in our original grant proposals to NSF and NEH.

The importance of data contributors cannot be overstated, and STEDT would like to express the deepest gratitude to the many linguists, native speakers, and other individuals who advanced the project by contributing lexical and/or etymological data to the database. Below is a list of some of these contributors in alphabetical order of surname/family name. Some contributors have no doubt been inadvertently missed, and we apologize for any omissions. (Asterisks indicate contributions that have not (yet) been imported.)


\begin{itemize}
\item \textit{Name} | \textit{Contribution(s)}
\item N.\ Achumi | Sumi
\item M.\ Balawan (via Karl-Heinz Grüssner) | Tiwa
\item Peri Bhaskararao | Mizo, Tiddim Chin
\item Balthasar Bickel | Belhare*, Puma*
\item Naomi Bishop | Sherpa Helambu
\item Sri Chhimed Bodh | Spiti Tibetan
\item David Bradley | Ugong, Bisu
\item Philip \& Cecilia Brassett | Tujia
\item Seino van Breugel | Atong
\item Daniel Bruhn | Central Naga languages
\item Robbins Burling | Garo
\item Robbins Buring \& U.V. Joseph | Bodo-Garo languages*, Proto-Bodo-Garo*
\item Christopher Button | Northern Chin languages*, Proto-Northern Chin*
\item Ross Caughley | Chepang
\item \textsc{Chen} Kang | Tujia
\item Katia Chirkova | Uppper Xumi*, Lizu*
\item Terry Cooke | Manang*
\item Alec Coupe | Mongsen Ao
\item \textsc{Dai} Qingxia | various \& sundry languages (including Deng, Himalayish, Tibetan, Qiangic, rGyalrongic, Nungic, Burmish, \& Loloish languages)
\item George van Driem | Dumi, Limbu
\item Karen Ebert | Chamling*
\item Jonathan Evans | Southern Qiang languages
\item Jonathan Evans, John B. Lowe, \& Jackson Tianshin Sun | Mandarin Chinese
\item Michael Ferlus | Phunoi
\item Carol Genetti | Newar
\item Karen Grunow-Hårsta | Magar*
\item Thien Haokip | Thado
\item \textsc{He} Zhiwu | Naxi*
\item \textsc{Huang} Chenglong | Yadu Qiang
\item \textsc{Huziwara} Keisuke | Sak, Marma, Ganan*, Kadu*, Usoi Tripura*, Proto-Luish*
\item François Jacquesson | Deori
\item Tej Ratna Kansakar | Baram
\item John King | Dhimal
\item Tamara Kohn | Yakha
\item Donald Kosha | Moyon
\item Shree Krishan | Chaudangsi, Darma, Raji
\item Randy LaPolla | Rawang, Dulong
\item Paul Lewis | Akha
\item \textsc{Li} Fanwen (via Richard Cook \& Andrew West) | Tangut
\item \textsc{Li} Shaoni | Jianchuan Bai*
\item Liberty Lidz | Yongning Na
\item Theraphan Luangthongkum | Karenic languages, Proto-Karen
\item \textsc{Luo} Meizhen | Ahi
\item Anton Lustig | Zaiwa*
\item \textsc{Ma} Erji | Daofu
\item \textsc{Ma} Xueliang | Sani
\item Kamal Malla | Newar
\item Ken Manson | Kayan
\item LaRaw Maran et al.\ | Jingpho*
\item James Matisoff | various \& sundry languages
\item Martine Mazaudon | Nepali*, Tamangic languages
\item Boyd Michailovsky | Chepang, Kham, Kiranti languages, Proto-Kiranti, Magar, Nepali, Newar, Written Tibetan
\item Alexis Michaud | Yonging Na, Laze
\item Dinkapar Moral | Boro*, Garo*, Koch*, Tintekiya Koch*
\item David Mortensen | Sorbung, Tankghulic languages, Proto-Tangkhulic
\item \textsc{Nagano} Yasuhiko et al.\ | rGyalrongic languages
\item Vikuosa Nienu | Angami, Chokri, Lotha
\item Michael Noonan et al.\ | Chantyal
\item Jamin Pelkey | Phula languages
\item Audra Phillips | Moulmein Pwo
\item Heleen Plaisier | Lepcha
\item Mark Post | Apatani*
\item Mark Post et al. | Galo
\item Krishna Prasad Rai, Anna Holzhausen, \& Andreas Holzhausen (via Boyd Michailovsky) | Kulung
\item Roland Rutgers | Yamphu
\item Daya Ratna Shakya \& David Hargreaves | Newar
\item Suhnu Ram Sharma | Bunan, Byangsi, Manchati, Rongpo
\item \textsc{Shintani} Tadahiko | Pyen
\item Ch.\ Yashwanta Singh | Meithei
\item Helga So-Hartmann | Southern Chin languages
\item David Solnit | Eastern Kayah Li*, Pa-O Karen
\item \textsc{Sun} Hongkai | various \& sundry languages (including Qiangic, Bodic, rGyalrongic, \& Loloish languages)
\item Jackson Tianshin Sun | Caodeng, Mandarin Chinese, Mawo Qiang, rGBenzhen, Shili, Amdo Tibetan, Written Tibetan, ``North Assam" languages, Tani languages, Proto-Tani
\item Tabu Taid | Mising*
\item T.\ Temsunungsang | Chungli Ao, Mongsen Ao
\item \textsc{Tian} Desheng | Tujia*
\item \textsc{Toba} Sueyoshi \& Allen Kom | Kom Rem
\item Prashanta Tripuri \& Dan Jurafsky | Kokborok
\item Kenneth VanBik | Kuki-Chin languages, Prto-Kuki-Chin
\item Nienu Vikuosa | Angami, Chokri, Lotha
\item \textsc{Wang} Ersong | Haoni*
\item David \& Nancy Watters | Kham
\item Julian Wheatley | Luquan*
\item \textsc{Xu} Xijian | Langsu, Lashi, Zaiwa*
\item Shiro Yabu | Khezha
\item \textsc{Zhang} Fengyu | Batang Tibetan
\item \textsc{Zhang} Liansheng | Written Tibetan
\item \textsc{Zhao} Yansun (via Grace Wiersma) | Bai
\end{itemize}


\section{STEDTniks}

Out of the approximately 70 individuals (undergraduates, graduate students, post-docs) who have worked at STEDT since 1987, I am here singling out 14 for special mention because of the significance of their contributions to the project. They are cited approximately in the order in which they have received their highest degrees from U.C.\ Berkeley or elsewhere. It should be understood that these brief mentions in no way do justice to the brilliance of their accomplishments, or to the love I feel for them all.

\textbf{Randy J.\ LaPolla}. (STEDT, 1987–90); Ph.D.\ Berkeley, 1990.) Has since taught at City University of Hong Kong, La Trobe University (Melbourne), and now at Nanyang Technological University (Singapore). Specialties include Nungish, Qiangic, comparative TB grammar. Edited early STEDT monographs and questionnaires.

\textbf{Jackson (Tianshin) Sun}. (STEDT 1990–1992; Ph.D.\ Berkeley, 1993.) Now teaching, conducting research, and administrating at Academia Sinica, Taipei. Specialties include Qiangic, rGyalrongic, and the NE Indian branch of Tibeto-Burman to which he assigned the now generally accepted name “Tani”.

\textbf{John B.\ Lowe} (STEDT 1987–2000; 2010–2014; Ph.D.\ Berkeley, 1995.) The only original project member still on the STEDT staff, and the designer of our original computer environment. Library scientist and programmer extraordinaire, he has participated in virtually all sub-projects at STEDT since the beginning.

\textbf{Zev J.\ Handel}. (STEDT 1991–1998; Ph.D.\ Berkeley, 1998.) Now Associate Professor of Chinese Language and Linguistics at University of Washington. Specialties include Chinese historical phonology, Sino-Xenic writing systems. Has contributed extensive notes on the Chinese comparanda cited in connection with TB etymologies in several STEDT publications.

\textbf{Ju Namkung}. (STEDT 1993–1997; M.A.\ Berkeley, 1995.) Currently a web and software developer in the Seattle area. She edited the widely used STEDT Monograph \#3, Phonological Inventories of Tibeto-Burman Languages (1996). Editorial assistant (1996–1997) for the journal then published at STEDT, Linguistics of the Tibeto-Burman Area (LTBA).

\textbf{Jonathan P.\ Evans}. (STEDT 1990–2005; Ph.D.\ Berkeley, 1999.) Now Associate Research Fellow at Academia Sinica, Taipei. Specialties include Qiangic grammar and historical phonology (especially the development of rudimentary tonal systems), and language contact between Chinese and coterritorial TB languages.

\textbf{Richard S.\ Cook}. (STEDT 1998–2011; Ph.D.\ Berkeley, 2003.) Independent researcher, specializing in the history of Chinese characters and the Chinese lexicographical tradition. Played a central role in the production of the Handbook of Proto-Tibeto-Burman. Created electronic versions of several key reference works, which have become “ancillary STEDT databases”.

\textbf{Kenneth VanBik}. (STEDT 1997–2006; Ph.D.\ Berkeley, 2006.) Now teaching at San Jose State University. A native speaker of Lai Chin and Burmese, he has contributed many PTB etymologies based on his newly discovered Chin/Burmese cognates. His dissertation was a full reconstruction of Proto-Kuki-Chin, with over 1350 cognate sets.

\textbf{David Mortensen}. (STEDT 2003–2006; Ph.D.\ Berkeley, 2006.) Now teaching at University of Pittsburgh. Originally with a background in the Hmong-Mien family, he now also specializes in the Tangkhulic branch of the Naga languages, as well as in theoretical phonology. Played a major role in the production of the English-Lahu Lexicon.

\textbf{Nina J.\ Keefer}. (STEDT 2000–2008; M.A./M.J.\ Berkeley [Group in Asian Studies, Graduate School of Journalism].) Editorial Assistant for LTBA (2001–2004) and my chief assistant in the production of the English-Lahu Lexicon. Background in Burmese culture and religion. Currently working as a forensic mental health nurse in Canada.

\textbf{Liberty A.\ Lidz}. (STEDT 1998, 2009-2014). Ph.D.\ University of Texas at Austin (2010). Specialist in languages of the Na(xi) group, closely related to Lolo-Burmese, and author of a dissertation grammar on Yongning Na. She did extensive data analysis for the STEDT project, having tagged roughly 43% of the forms tagged in the database at the time of publication, as well as contributing a number of new etymologies and working closely with the P.I. to rectify existing PTB roots.

\textbf{Dominic Yu}. (STEDT 2004–2012; Ph.D.\ Berkeley, 2012.) Now employed as an International UI Software Engineer at Apple. Has specialized in Qiangic languages, especially those in the Ersuic group. Formatted and produced The Tibeto-Burman Reproductive System (2008). He continues to make key contributions to the structure of the STEDT database and website.

\textbf{Daniel W.\ Bruhn}. (STEDT 2009–2014; Ph.D.\ Berkeley, 2014.) Currently on the STEDT staff. Has specialized in the languages of the Central Naga group. A Jack-of-all-trades, he is contributing significantly to all ongoing STEDT projects, including the incorporation of new sources into our database, modifications to our underlying code, improvements in our website, and formatting our publications.

\textbf{Chundra A.\ Cathcart}. (STEDT 2011–2014; Ph.D.\ Berkeley, expected 2014–5.) Currently on the STEDT staff. Specializes in the comparative/historical study of the Indo-Aryan family, and frequently identifies loans from IA languages into Tibeto-Burman. Is assisting the P.I.\ in the preparation for publication of a large corpus of Lahu texts.

\section{Visiting Scholars}

The STEDT project has been greatly enriched by the specialized expertise, unpublished data, and intellectual stimulation provided by a succession of visiting scholars, who have spent anywhere from a few weeks to more than two years at the project headquarters. Two especially deserve special mention:

\textbf{Martine Mazaudon}. (STEDT 1987–89, 1990–91; Doctorat 3e Cycle, Paris V, 1971; Doctorat d’Etat, Paris III, 1994.) Now a Researcher Emerita at the Centre National de la Recherche Scientifique (Paris), specializing in the Tibeto-Burman languages of the Tamangic group of Central Nepal. Spent two full years at STEDT.

\textbf{Boyd Michailovsky}. (STEDT 1987–89, 1990–91; Ph.D.\ Berkeley, 1981; Doctorat 3e Cycle, Paris III, 1981; Doctorat d’Etat, Paris III, 2004.) Now a Researcher Emeritus at the Centre National de la Recherche Scientifique (Paris), specializing in the Tibeto-Burman languages of the Kiranti (Rai) group of Eastern Nepal. Spent two full years at STEDT.

{\sc Dai} Qingxia and {\sc Xu} Xijian (Oct.–Nov.\ 1989) Nationalities University (Beijing), TB languages of China; {\sc Zhang} Jichuan (Nov.\ 1990) Chinese Academy of Social Sciences (Beijing), Tibetan dialects; the late Rev.\ George Kraft (1990–99), Khams Tibetan; Nicolas Tournadre (Feb.\ 1991) University of Paris III, Tibetan; {\sc Sun} Hongkai and {\sc Liu} Guangkun (April–May, 1991) Chinese Academy of Social Sciences (Beijing), TB languages of China; {\sc Yabu} Shiro (April–Aug.\ 1994) Osaka Foreign Languages University, Burmish languages and Xixia; William H.\ Baxter III (May, 1995) University of Michigan, Old Chinese; Balthasar Bickel (Sept.–Oct.\ 1996; Feb.–Mar.\ 1997) University of Z\"urich and Johann Gutenberg University (Mainz), Kiranti languages; {\sc Lin} Ying-chin (1997–98) Academia Sinica, Taipei (Xixia, Muya); David B.\ Solnit (1998–) STEDT, Karenic; {\sc Wu} Sheng-hsiung (Spring, 2002) Taiwan Normal University, Chinese phonology; {\sc Ikeda} Takumi (2002–03) Kyoto University, Qiangic languages; Yashawanta Singh  Manipur University (Imphal) Meithei; Maung Maung (Aaron Tun) (2011) Payap University (Chiangmai), Lahu; {\sc Li} Yan (2013–14) Shaanxi Normal University (Xi’an) Historical linguistics and dialectology.

\section{Complete List of STEDTniks}

Most of all, I am indebted to the phalanges of talented students, past and present, who have been working at STEDT anywhere from five to 20 or 30 hours per week, performing a host of vital tasks such as the inputting and proofreading of hundreds of thousands of lexical records, the development of special fonts and relational database software, computer maintenance and troubleshooting, formatting articles for our journal {\it Linguistics of the Tibeto-Burman Area}, and editing the publications in the STEDT Monograph Series. 70 researchers have contributed to the STEDT project since 1987, mostly graduate students working as research assistants in the Berkeley Linguistics and East Asian Languages and Cultures Departments, but also including several undergraduate volunteers and non-enrolled or former students. Here they are, in an alphabetical honor roll:

\begin{multicols}{3}
\begin{itemize}
\item Madeleine Adkins
\item Jocelyn Ahlers
\item Shelley Axmaker
\item Kenny Baclawski
\item Stephen P.\ Baron
\item Leela Bilmes (Goldstein)
\item Michael Brodhead
\item Daniel Bruhn
\item Chundra Cathcart
\item Jeff Chan
\item Patrick Chew
\item Melissa Chin
\item Isara Choosri
\item Richard Cook
\item Jeff Dale
\item Amy Dolcourt
\item Julia Elliot
\item Jonathan P.\ Evans
\item Allegra Giovine
\item Cynthia Gould
\item Daniel Granville
\item Joshua Guenter
\item Kira Hall
\item Zev J.\ Handel
\item Takumi Ikeda
\item Alexander Jacobson
\item Annie Jaisser
\item Matthew Juge
\item Daniel Jurafsky
\item David Kamholz
\item Nina Keefer
\item Jean Kim
\item Kyung-Ah Kim
\item Heidi Kong
\item Aim\'ee Lahaussois (Bartosik)
\item Randy J.\ LaPolla
\item Jennifer Leehey
\item Anita Liang
\item Liberty Lidz
\item John B.\ Lowe
\item Jean McAneny
\item Pamela Morgan
\item David Mortensen
\item Karin Myrhe
\item Ju Namkung
\item Toshio Ohori
\item Weera Ostapirat
\item Jeong-Woon Park
\item Jason Patent
\item Chris Redfearn
\item S.\ Ruffin
\item Keith Sanders
\item Marina Shawver
\item Elizabeth Shriberg
\item Helen Singmaster
\item Tanya Smith
\item Gabriela Solomon
\item Silvia Sotomayor
\item Jackson Tianshin Sun
\item Laurel Sutton
\item Prashanta Tripura
\item Margaret Urban
\item Nancy Urban
\item Kenneth VanBik
\item Blong Xiong
\item Dominic Yu
\item Liansheng Zhang
\end{itemize}
\end{multicols}
