{\large \parindent=-1em \textbf{Acknowledgements}}
\addcontentsline{toc}{chapter}{Acknowledgements}
\vspace{0.25em}

\section{NSF and NEH Support}

I am deeply grateful to the National Science Foundation and the National Endowment for the Humanities for their unswerving support of the {\it Sino-Tibetan Etymological Dictionary and Thesaurus} (STEDT) Project since 1987, even through times of budgetary stringency. I would especially like to thank Dr. Paul G. Chapin, of the Language, Cognition, and Social Behavior division of NSF; and Dr. Guinevere Greist, Dr. Helen Ag\"uera and Dr. Martha Bohachevsky-Chomiak of the Research Tools Division of NEH. I can only hope that the fruits of this project will repay their confidence and patience.

Grants to the \textit{Sino-Tibetan Etymological Dictionary and Thesaurus} project from:

\begin{itemize}
\item[*] The National Science Foundation (NSF), Division of
  Behavioral \& Cognitive Sciences, Grant Nos.\ BNS-86-17726,
  BNS-90-11918, DBS-92-09481, FD-95-11034, SBR-9808952,
  BCS-9904950, BCS-0345929, and BCS-0712570.
\item[*] The National Endowment for the Humanities (NEH),
  Preservation and Access, Grant Nos.\ RT-20789-87, RT-21203-90,
  RT-21420-92, PA-22843-96, PA-23353-99, PA-24168-02,
  PA-50709-04, and PM-50072-07.
\end{itemize}

\section{Administrators}

Several Organized Research Units and academic departments of the Berkeley campus have given their moral or practical support to the STEDT project, including the Center for Southeast Asia Studies, the Center for Chinese Studies, the Department of Linguistics, the Department of South and Southeast Asian Studies, the Department of East Asian Languages and Cultures, and especially the Institute of International and Area Studies, to whose administrative staff I am deeply obliged: Karin Beros, Management Services Officer and all-around trouble-shooter, who was instrumental in solving the practical problems of getting the project started back in 1987; Jerilyn C. Foush\'ee, who has handled our budget and helped with our grant proposals and reports since 1987; and Nell Haskell (1987-95) and Kerttu K. McCray (1995-2002), who have kept track of personnel matters.  Since 200x, the STEDT grants have been administered by the Linguistics Department, under the able stewardship of Paula Floro.

\section{Contributors}

I would like to thank Anthony Meadow, founder of the Bear River Associates software development firm, now in Oakland, who generously gave many hours of his time during 1986-87 in {\it pro bono} consultations about how to formulate the computer needs of the project in our original grant proposals to NSF and NEH. 

It is a pleasure to single out several ``Stedtniki''  whose contributions to this project and the present volume have been particularly outstanding, and all of whose computorial expertise infinitely outstrips my own:
\begin{itemize}
\item John B. (``J.B.'' ) Lowe, the only researcher who has been continuously working at STEDT since its inception in 1987, designed our initial computer environment and has been fine-tuning it ever since, creating original database software adapted to the highly specialized needs of the project and breaking new conceptual ground in the use of the computer for etymological research.\footnote{J.B.'s work at STEDT has already spun off into several other etymological projects on which he has consulted here and abroad: M. Mazaudon and Boyd Michailovsky's {\it Reconstruction Engine} (Paris) for testing putative cognate sets in Himalayan languages; L. M. Hyman's {\it Comparative Bantu On-line Dictionary} (CBOLD, Berkeley); Sjors van Driem and K.B. Kepping's {\it Tangut Dictionary Project} (Leiden), and Sharon Inkelas'  {\it Turkish Electronic Living Lexicon}.}
\item Randy J. LaPolla, now teaching at the City University of Hongkong, has also been affiliated with STEDT since the beginning. Until receiving his doctorate in 1990, he played a vital part in our activities, including the preparation of STEDT Monographs and the processing of fieldworkers' questionnaires. His superb knowledge of Chinese has been a prime asset to the project.
\item Zev J. Handel (``Z as in {\it zebra}, V as in {\it violin}'', as he explains over the telephone), is a specialist in Chinese historical phonology, now teaching at the University of Washington. He was active at STEDT in the 1990's, and had a major role in the formatting of our prototype ``fascicle''  on the {\it Reproductive System} for our projected {\item Bodyparts} volume, adding bells and whistles like the program to insert notes at various points in the etymologies, and transforming my hand-scrawled semantic diagrams into elegant computer graphics. I am especially grateful to him for producing the concise comparison of three of the most influential systems for reconstructing Old Chinese that appears as an Appendix to this {\it Handbook}.
\end{itemize}

When I went off on sabbatical to Taiwan during 1995-96, I left the day-to-day running of STEDT in the capable hands of J.B. and Zev. One day I e-mailed them from Taipei, referring to them as the ``duumvirate'' . Back came an aggrieved message from J.B., protesting that they really would rather be called the ``smart-virate'' . No argument there.

\begin{itemize}
\item Kenneth VanBik is a native speaker of Lai Chin and a graduate of Rangoon University. Possessing an intimate knowledge of languages from two branches of Tibeto- Burman, he was able to identify a number of new Burmese/Chin cognates that are thus reconstructible at the PTB level. His etymologies are included in this volume, marked ``KVB''.
\item Richard S. Cook, currently producing a mammoth dissertation on the Eastern Han ``Grammaticon''  {\it Shu\=o W\'en Ji\v{e} Z\`i} , has been the chief architect of the formatting of this Handbook  during 2002-3. It was his idea to transfer the whole MS from Microsoft Word 5.1a to Adobe FrameMaker\textsuperscript{TM}, an arduous process that has paid off in the end, as the attractive appearance of the book testifies. Richard wrote Appendix B  (in consultation with Zev Handel), and extracted the etymologies from the electronic  Dictionary of Lahu  files to supplement the Index of Proto-Forms . He wrote the computer programs to format the Index of Proto-Forms  and to generate and format the indexes of {\it Proto-Glosses}, {\it Proto-Root-Syllables}, {\it Proper Names}, and {\it Chinese Character}s. He produced the kerned version of the STEDT PostScript font family, as well as the font for the rare Chinese characters found in this book.
\item David Mortensen, a linguistics graduate student specializing in Hmong-Mien, has contributed equally to the production of this {\it Handbook}. An accomplished computorial troubleshooter, he did much formatting work, and has carried out such vital tasks as assuring the integrity of the {\it Handbook's} innumerable internal cross-references.
\end{itemize}

During this period J.~B.\ Lowe devised a pioneering program called "The Tagger's Assistant", that enabled me to etymologize tens of thousands of syllables in our database by labelling them with numerical "tags" that could then be used to assemble them into cognate sets.  (That is, each syllable deemed to be a reflex of a particular etymon would be tagged with the same number.)  With an eye to the eventual publication of our results, J.~B. also solved such essential formatting problems as how to insert footnotes at any point in a printed etymological text, whether on a semantic diagram, an etymon as a whole, or a particular supporting form.\footnote{See Section 2.9 of the \textit{Introduction}, below.}

So I decided to let the thesauric side of STEDT slide for awhile, and to switch the emphasis of the project to \textit{phonologically} presented etyma (the "D" or "dictionary" part of "STEDT"), an effort which culminated in the publication of the \textit{Handbook of Proto-Tibeto-Burman} (2003).

It gives me special pleasure to thank Professor Zev J.\ Handel of the University of Washington for his work in producing the Chinese comparanda in this work.\footnote{See the \textit{Introduction}, Section 2.8.}   Zev had originally contributed such comments to the preliminary version of the manuscript some ten years ago, evaluating my suggested Proto-Tibeto-Burman/Old Chinese comparisons in terms of the competing reconstructive systems of leading Sinologists, past and present.  These updated comments, presented in a neutral, non-judgmental tone, constitute a precious guide through the minefield of Chinese historical phonology!

The structure and essential layout of the Dictionary-Thesaurus have been set for some time -- indeed, several intermediate iterations of the work have been produced, starting with the 1995 draft of the Body Parts volume produced in RTF by J.~B.  The "fascicle producing software has now been re-written three times, each iteration corresponding to a major revision in the underlying technology supporting the database. \mbox{Dominic} Yu did the initial programming of the latest and final "publication system", a set of PERL programs which extracted material from the STEDT database and created LaTex files from which the publication-ready documents were produced.  These have been substanially revised in the last two years by J.~B., Daniel, and Chundra.  

On the back end, his efforts involved porting the entire database to a web-accessible engine using MySQL, accomplished in conjunction with David R.\ Mortensen and J.~B.\ Lowe, and simultaneously converting our in-house legacy STEDT Font encoding to Unicode. The final print volume is typeset in \XeLaTeX\ using Charis SIL as the main font.

The STEDT project has been sponsored from the beginning by the National Endowment for the Humanities and the National Science Foundation.  To both agencies I express again my enduring gratitude.\footnote{See \textit{Grant Support}, p.~i above.}

The STEDT logo was designed by Nadja R. Matisoff.

Finally I would like to thank my wife Susan for her constant support, and for having taught me so much about the reproductive system over the past 46 years.

\section{Visiting Scholars}

The STEDT project has been greatly enriched by the specialized expertise, unpublished data, and intellectual stimulation provided by a succession of visiting scholars, who have spent anywhere from a few weeks to more than two years at the project headquarters: 

Martine Mazaudon and Boyd M. Michailovsky (1987-89, 1990-91) Centre National de la Recherche Scientifique (Paris), Himalayan languages ; {\sc Dai} Qingxia and {\sc Xu} Xijian (Oct.-Nov. 1989) Nationalities University (Beijing), TB languages of China ; {\sc Zhang} Jichuan (Nov. 1990) Chinese Academy of Social Sciences (Beijing), Tibetan dialects ; the late Rev. George Kraft (1990-99), Khams Tibetan ; Nicolas Tournadre (Feb. 1991) University of Paris III, Tibetan ; {\sc Sun} Hongkai and {\sc Liu} Guangkun (April-May, 1991) Chinese Academy of Social Sciences (Beijing), TB languages of China ; {\sc Yabu} Shiro (April-Aug. 1994) Osaka Foreign Languages University, Burmish languages and Xixia ; William H. Baxter III (May, 1995) University of Michigan, Old Chinese ; Balthasar Bickel (Sept.-Oct. 1996; Feb.-Mar. 1997) University of Z\"urich and Johann Gutenberg University (Mainz), Kiranti languages ; {\sc Lin} Ying-chin (1997-98) Academia Sinica, Taipei (Xixia, Muya); David B. Solnit (1998-) STEDT, Karenic ; {\sc Wu} Sheng-hsiung (spring, 2002) Taiwan Normal University, Chinese phonology ; {\sc Ikeda} Takumi (2002-03) Kyoto University, Qiangic languages 

\section{STEDTniks}

Most of all, I am indebted to the phalanges of talented students, past and present, who have been working at STEDT anywhere from five to 20 or 30 hours per week, performing a host of vital tasks such as the inputting and proofreading of hundreds of thousands of lexical records, the development of special fonts and relational database software, computer maintenance and troubleshooting, formatting articles for our journal {\it Linguistics of the Tibeto-Burman Area}, and editing the publications in the STEDT Monograph Series. 63 researchers have contributed to the STEDT project since 1987, mostly graduate students working as research assistants in the Berkeley Linguistics and East Asian Languages and Cultures Departments, but also including several undergraduate volunteers and non-enrolled or former students. Here they are, in an alphabetical honor roll:

\begin{itemize}
\item Madeleine Adkins
\item Jocelyn Ahlers
\item Shelley Axmaker
\item Stephen P. Baron
\item Leela Bilmes (Goldstein)
\item Michael Brodhead
\item Daniel Bruhn
\item Chundra Cathcart
\item Jeff Chan
\item Patrick Chew
\item Melissa Chin
\item Isara Choosri
\item Richard Cook
\item Jeff Dale
\item Amy Dolcourt
\item Julia Elliot
\item Jonathan P. Evans
\item Allegra Giovine
\item Cynthia Gould
\item Daniel Granville
\item Joshua Guenter
\item Kira Hall
\item Zev J. Handel
\item Takumi Ikeda
\item Annie Jaisser
\item Matthew Juge
\item Daniel Jurafsky
\item David Kamholz
\item Nina Keefer
\item Jean Kim
\item Kyung-Ah Kim
\item Heidi Kong
\item Aim\'ee Lahaussois (Bartosik)
\item Randy J. LaPolla
\item Jennifer Leehey
\item Anita Liang
\item Liberty Lidz
\item John B. Lowe
\item Jean McAneny
\item Pamela Morgan
\item David Mortensen
\item Karin Myrhe
\item Ju Namkung
\item Toshio Ohori
\item Weera Ostapirat
\item Jeong-Woon Park
\item Jason Patent
\item Chris Redfearn
\item S. Ruffin
\item Keith Sanders
\item Marina Shawver
\item Elizabeth Shriberg
\item Helen Singmaster
\item Tanya Smith
\item Gabriela Solomon
\item Silvia Sotomayor
\item Jackson Tianshin Sun
\item Laurel Sutton
\item Prashanta Tripura
\item Nancy Urban
\item Kenneth VanBik
\item Blong Xiong
\item Dominic Yu
\item Liansheng Zhang
\end{itemize}