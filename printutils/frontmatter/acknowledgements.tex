\vspace{0.25em}

\chapter*{Acknowledgments}

\renewcommand{\thefootnote}{\arabic{footnote}}
\setcounter{footnote}{0}

\section{NSF and NEH Support}

I am deeply grateful to the National Science Foundation and the National Endowment for the Humanities for their unswerving support of the {\it Sino-Tibetan Etymological Dictionary and Thesaurus} (STEDT) Project since 1987, even through times of budgetary stringency. I would especially like to thank Dr. Paul G. Chapin and Dr. Joan Maling, of the Language, Cognition, and Social Behavior division of NSF; and Dr. Guinevere Greist, Dr. Helen Ag\"uera, Dr. Martha Bohachevsky-Chomiak and Dr. Nadine Gardner of the Research Tools Division of NEH. I can only hope that the fruits of this project will repay their confidence and patience.

Grants to the \textit{Sino-Tibetan Etymological Dictionary and Thesaurus} project from:

\begin{itemize}
\item[*] The National Science Foundation (NSF), Division of
  Behavioral \& Cognitive Sciences, Grant Nos.\ BNS-86-17726,
  BNS-90-11918, DBS-92-09481, FD-95-11034, SBR-9808952,
  BCS-9904950, BCS-0345929,  BCS-0712570, and BCS-1028192.
\item[*] The National Endowment for the Humanities (NEH),
  Preservation and Access, Grant Nos.\ RT-20789-87, RT-21203-90,
  RT-21420-92, PA-22843-96, PA-23353-99, PA-24168-02,
  PA-50709-04,  PM-50072-07, and PW-50122-08.
\end{itemize}

\section{Administrators}

Several Organized Research Units and academic departments of the Berkeley campus have given their moral or practical support to the STEDT project, including the Center for Southeast Asia Studies, the Center for Chinese Studies, the Department of Linguistics, the Department of South and Southeast Asian Studies, the Department of East Asian Languages and Cultures, and especially the Institute of International and Area Studies, to whose administrative staff I am deeply obliged: Karin Beros, Management Services Officer and all-around trouble-shooter, who was instrumental in solving the practical problems of getting the project started back in 1987; Jerilyn C. Foush\'ee, who has handled our budget and helped with our grant proposals and reports since 1987; and Nell Haskell (1987-95) and Kerttu K. McCray (1995-2002), who have kept track of personnel matters.  Since 2009, the STEDT grants have been administered by the Linguistics Department, under the able stewardship of Paula Floro and Bel\'een Flores.

\section{Contributors}

I would like to thank Anthony Meadow, founder of the Bear River Associates software development firm, now in Oakland, who generously gave many hours of his time during 1986-87 in {\it pro bono} consultations about how to formulate the computer needs of the project in our original grant proposals to NSF and NEH. 

[need to decide whether to enumerate all data contributors. And if so, how]

\section{STEDTniks}

Out of the approximately 70 individuals (undergraduates, graduate students, post-docs) who have worked at STEDT since 1987, I am here singling out 14 for special mention because of the significance of their contributions to the project. They are cited approximately in the order in which they have received their highest degrees from U.C. Berkeley or elsewhere. It should be understood that these brief mentions in no way do justice to the brilliance of their accomplishments, or to the love I feel for them all.

\textbf{Randy J. LaPolla}. (STEDT, 1987-90); Ph.D. Berkeley, 1990.) Has since taught at City University of Hong Kong, La Trobe University (Melbourne), and now at National Technological University (Singapore). Specialties include Nungish, Qiangic, comparative TB grammar. Edited early STEDT monographs and questionnaires.

\textbf{Jackson (Tianshin) Sun}. (STEDT xyz; Ph.D. Berkeley, 1993.) Now teaching, conducting research, and administrating at Academia Sinica, Taipei. Specialties include Qiangic, rGyalrongic, and the NE Indian branch of Tibeto-Burman to which he assigned the now generally accepted name “Tani”.

\textbf{John B. Lowe (“J.B.”)} (STEDT 1987-2003; 2010-2014; Ph.D. Berkeley, 1995.) The only original project member still on the STEDT staff, and the designer of our original computer environment. Library scientist and programmer extraordinaire, he has participated in virtually all sub-projects at STEDT since the beginning.

\textbf{Zev J. Handel}. (STEDT xyz; Ph.D. Berkeley, 1998.) Now professor at University of Washington. Specialties include Chinese historical phonology, Sino-Xenic writing systems. Has contributed extensive notes on the Chinese comparanda cited in connection with TB etymologies in several STEDT publications.

\textbf{Ju Namkung}. (STEDT xyz; M.A. degree < UCB? When?) Now [title?] at Amazon.com. She edited the widely used STEDT Monograph \#3, Phonological Inventories of Tibeto-Burman Languages (1996). Editorial assistant [give years] for the journal then published at STEDT, Linguistics of the Tibeto-Burman Area (LTBA).

\textbf{Jonathan P. Evans}. (STEDT xyz; Ph.D. Berkeley, 1999.) Now researcher [exact title?] at Academic Sinica, Taipei. Specialties include Qiangic grammar and historical phonology (especially the development of rudimentary tonal systems), and language contact between Chinese and coterritorial TB languages.

\textbf{Richard S. Cook}. (STEDT xyz; Ph.D. Berkeley, 2003.) Independent researcher, specializing in the history of Chinese characters and the Chinese lexicographical tradition. Played a central role in the production of the Handbook of Proto-Tibeto-Burman. Created electronic versions of several key reference works, which have become “ancillary STEDT databases”.

\textbf{Kenneth VanBik}. (STEDT xyz; Ph.D. Berkeley, 2006.) Now teaching at San Jose State University. A native speaker of Lai Chin and Burmese, he has contributed many PTB etymologies based on his newly discovered Chin/Burmese cognates. His dissertation was a full reconstruction of Proto-Kuki-Chin, with over 1350 cognate sets.

\textbf{David Mortensen}. (STEDT xyz; Ph.D. Berkeley, 2006.) Now teaching at University of Pittsburgh. Originally with a background in the Hmong-Mien family, he now also specializes in the Tangkhulic branch of the Naga languages, as well as in theoretical phonology. Played a major role in the production of the English-Lahu Lexicon.

\textbf{Nina J. Keefer}. (STEDT xyz; M.A. Berkeley (Group in Asian Studies, Dept. of SSEAL.) Background in journalism, and in Burmese culture and history. Editorial assistant for LTBA [dates], and my chief assistant in the production of the English-Lahu Lexicon. Recently received a nursing degree in Canada, perhaps with a view to joining an international NGO in Burma.

\textbf{Liberty A. Lidz}. (STEDT 1998, 2009-2014; Ph.D. University of Texas (Austin), 2010.) Currently on the STEDT staff. She has specialized in the languages of the Na(xi) group, closely related to Lolo-Burmese. Now working intensively with the P.I. on the “rectification” of our etymologies before the project draws to a close. Has contributed a number of new ones as well.

\textbf{Dominic Yu}. (STEDT 2002-2014; Ph.D. Berkeley, 2012.) Now employed as a Language Technologies Engineer at Apple. Has specialized in Qiangic languages, especially those in the Ersuic group. Formatted and produced The Tibeto-Burman Reproductive System (2008). He continues to make key contributions to the structure of the STEDT database and website.

\textbf{Daniel W. Bruhn}. (STEDT 2009-2014; Ph.D. Berkeley, 2014.) Currently on the STEDT staff. Has specialized in the languages of the Central Naga group. An expert programmer, he is contributing significantly to all ongoing STEDT projects, including the incorporation of new sources into our database, improvements in our website, and formatting our publications.

\textbf{Chundra A. Cathcart}. (STEDT 2011-2014; Ph.D. Berkeley, expected 2014-5.) Currently on the STEDT staff. Specializes in the comparative/historical study of the Indo-Aryan family, and frequently identifies loans from IA languages into Tibeto-Burman. Is assisting the P.I. in the preparation for publication of a large corpus of Lahu texts.

\section{Visiting Scholars}

The STEDT project has been greatly enriched by the specialized expertise, unpublished data, and intellectual stimulation provided by a succession of visiting scholars, who have spent anywhere from a few weeks to more than two years at the project headquarters: 

Martine Mazaudon and Boyd M. Michailovsky (1987-89, 1990-91) Centre National de la Recherche Scientifique (Paris), Himalayan languages; {\sc Dai} Qingxia and {\sc Xu} Xijian (Oct.-Nov. 1989) Nationalities University (Beijing), TB languages of China; {\sc Zhang} Jichuan (Nov. 1990) Chinese Academy of Social Sciences (Beijing), Tibetan dialects; the late Rev. George Kraft (1990-99), Khams Tibetan; Nicolas Tournadre (Feb. 1991) University of Paris III, Tibetan; {\sc Sun} Hongkai and {\sc Liu} Guangkun (April-May, 1991) Chinese Academy of Social Sciences (Beijing), TB languages of China; {\sc Yabu} Shiro (April-Aug. 1994) Osaka Foreign Languages University, Burmish languages and Xixia; William H. Baxter III (May, 1995) University of Michigan, Old Chinese; Balthasar Bickel (Sept.-Oct. 1996; Feb.-Mar. 1997) University of Z\"urich and Johann Gutenberg University (Mainz), Kiranti languages; {\sc Lin} Ying-chin (1997-98) Academia Sinica, Taipei (Xixia, Muya); David B. Solnit (1998-) STEDT, Karenic; {\sc Wu} Sheng-hsiung (spring, 2002) Taiwan Normal University, Chinese phonology; {\sc Ikeda} Takumi (2002-03) Kyoto University, Qiangic languages. Yashawanta Singh  Manipur University (Imphal) Meithei;  Maung Maung (Aaron Tun) (2011) Payap University (Chiangmai), Lahu; Lin Jinrong (Cathaw) ???, Li Yan (2013-14) Xi'an Normal University (Xi'an) Chinese Tones. 

\section{Complete List of STEDTniks}

Most of all, I am indebted to the phalanges of talented students, past and present, who have been working at STEDT anywhere from five to 20 or 30 hours per week, performing a host of vital tasks such as the inputting and proofreading of hundreds of thousands of lexical records, the development of special fonts and relational database software, computer maintenance and troubleshooting, formatting articles for our journal {\it Linguistics of the Tibeto-Burman Area}, and editing the publications in the STEDT Monograph Series. 70 researchers have contributed to the STEDT project since 1987, mostly graduate students working as research assistants in the Berkeley Linguistics and East Asian Languages and Cultures Departments, but also including several undergraduate volunteers and non-enrolled or former students. Here they are, in an alphabetical honor roll:

\begin{multicols}{3}
\begin{itemize}
\item Madeleine Adkins
\item Jocelyn Ahlers
\item Shelley Axmaker
\item Stephen P. Baron
\item Leela Bilmes (Goldstein)
\item Michael Brodhead
\item Daniel Bruhn
\item Chundra Cathcart
\item Jeff Chan
\item Patrick Chew
\item Melissa Chin
\item Isara Choosri
\item Richard Cook
\item Jeff Dale
\item Amy Dolcourt
\item Julia Elliot
\item Jonathan P. Evans
\item Allegra Giovine
\item Cynthia Gould
\item Daniel Granville
\item Joshua Guenter
\item Kira Hall
\item Zev J. Handel
\item Takumi Ikeda
\item Alexander Jacobson
\item Annie Jaisser
\item Matthew Juge
\item Daniel Jurafsky
\item David Kamholz
\item Nina Keefer
\item Jean Kim
\item Kyung-Ah Kim
\item Heidi Kong
\item Aim\'ee Lahaussois (Bartosik)
\item Randy J. LaPolla
\item Jennifer Leehey
\item Anita Liang
\item Liberty Lidz
\item John B. Lowe
\item Jean McAneny
\item Pamela Morgan
\item David Mortensen
\item Karin Myrhe
\item Ju Namkung
\item Toshio Ohori
\item Weera Ostapirat
\item Jeong-Woon Park
\item Jason Patent
\item Chris Redfearn
\item S. Ruffin
\item Keith Sanders
\item Marina Shawver
\item Elizabeth Shriberg
\item Helen Singmaster
\item Tanya Smith
\item Gabriela Solomon
\item Silvia Sotomayor
\item Jackson Tianshin Sun
\item Laurel Sutton
\item Prashanta Tripura
\item Margaret Urban
\item Nancy Urban
\item Kenneth VanBik
\item Blong Xiong
\item Dominic Yu
\item Liansheng Zhang
\end{itemize}
\end{multicols}