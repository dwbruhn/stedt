Acknowledgments

1. Grant support 
	List all grant numbers

2. Administrators

3. Contributors
	Tony Meadow
	Questionnaire respondents
	Data contributors

4. Visiting scholars
	HPTB, p. ii. Add Yashawanta Singh, Maung Maung (Aaron Tun), Lin Jinrong 	(Cathaw), Li Yan

5. Stedtniks

	Out of the approximately 64 individuals (undergraduates, graduate students, post-docs) who have worked at STEDT since 1987, I am here singling out 14 for special mention because of the significance of their contributions to the project. They are cited approximately in the order in which they have received their highest degrees from U.C. Berkeley or elsewhere. It should be understood that these brief mentions in no way do justice to the brilliance of their accomplishments, or to the love I feel for them all.

\textbf{Randy J. LaPolla}. (STEDT, 1987-90); Ph.D. Berkeley, 1990.) Has since taught at City University of Hong Kong, La Trobe University (Melbourne), and now at National Technological University (Singapore). Specialties include Nungish, Qiangic, comparative TB grammar. Edited early STEDT monographs and questionnaires.

\textbf{Jackson (Tianshin) Sun}. (STEDT xyz; Ph.D. Berkeley, 1993.) Now teaching, conducting research, and administrating at Academia Sinica, Taipei. Specialties include Qiangic, rGyalrongic, and the NE Indian branch of Tibeto-Burman to which he assigned the now generally accepted name “Tani”.

\textbf{John B. Lowe (“J.B.”)} (STEDT 1987-xyz; 20xyz-2014; Ph.D. Berkeley, 1995.) The only original project member still on the STEDT staff, and the designer of our original computer environment. Library scientist and programmer extraordinaire, he has participated in virtually all sub-projects at STEDT since the beginning.

\textbf{Zev J. Handel}. (STEDT xyz; Ph.D. Berkeley, 1998.) Now professor at University of Washington. Specialties include Chinese historical phonology, Sino-Xenic writing systems. Has contributed extensive notes on the Chinese comparanda cited in connection with TB etymologies in several STEDT publications.

\textbf{Ju Namkung}. (STEDT xyz; M.A. degree < UCB? When?) Now [title?] at Amazon.com. She edited the widely used STEDT Monograph #3, Phonological Inventories of Tibeto-Burman Languages (1996). Editorial assistant [give years] for the journal then published at STEDT, Linguistics of the Tibeto-Burman Area (LTBA).

\textbf{Jonathan P. Evans}. (STEDT xyz; Ph.D. Berkeley, 1999.) Now researcher [exact title?] at Academic Sinica, Taipei. Specialties include Qiangic grammar and historical phonology (especially the development of rudimentary tonal systems), and language contact between Chinese and coterritorial TB languages.

\textbf{Richard S. Cook}. (STEDT xyz; Ph.D. Berkeley, 2003.) Independent researcher, specializing in the history of Chinese characters and the Chinese lexicographical tradition. Played a central role in the production of the Handbook of Proto-Tibeto-Burman. Created electronic versions of several key reference works, which have become “ancillary STEDT databases”.

\textbf{Kenneth VanBik}. (STEDT xyz; Ph.D. Berkeley, 2006.) Now teaching at San Jose State University. A native speaker of Lai Chin and Burmese, he has contributed many PTB etymologies based on his newly discovered Chin/Burmese cognates. His dissertation was a full reconstruction of Proto-Kuki-Chin, with over 1350 cognate sets.

\textbf{David Mortensen}. (STEDT xyz; Ph.D. Berkeley, 2006.) Now teaching at University of Pittsburgh. Originally with a background in the Hmong-Mien family, he now also specializes in the Tangkhulic branch of the Naga languages, as well as in theoretical phonology. Played a major role in the production of the English-Lahu Lexicon.

\textbf{Nina J. Keefer}. (STEDT xyz; M.A. Berkeley (Group in Asian Studies? Dept. of SSEALL?), [year?].) Background in journalism, and in Burmese culture and history. Editorial assistant for LTBA [dates], and my chief assistant in the production of the English-Lahu Lexicon. Recently received a nursing degree in Canada, perhaps with a view to joining an international NGO in Burma.

\textbf{Liberty A. Lidz}. (STEDT xyz-2014; Ph.D. University of Texas (Austin), 2010.) Currently on the STEDT staff. She has specialized in the languages of the Na(xi) group, closely related to Lolo-Burmese. Now working intensively with the P.I. on the “rectification” of our etymologies before the project draws to a close. Has contributed a number of new ones as well.

\textbf{Dominic Yu}. (STEDT xyz; Ph.D. Berkeley, 2012.) Now employed as a Language Technologies Engineer at Apple. Has specialized in Qiangic languages, especially those in the Ersuic group. Formatted and produced The Tibeto-Burman Reproductive System (2008). He continues to make key contributions to the structure of the STEDT database and website.

\textbf{Daniel W. Bruhn}. (STEDT xyz; Ph.D. Berkeley, 2014.) Currently on the STEDT staff. Has specialized in the languages of the Central Naga group. An expert programmer, he is contributing significantly to all ongoing STEDT projects, including the incorporation of new sources into our database, improvements in our website, and formatting our publications.

\textbf{Chundra A. Cathcart}. (STEDT xyz; Ph.D. Berkeley, expected 2014-5.) Currently on the STEDT staff. Specializes in the comparative/historical study of the Indo-Aryan family, and frequently identifies loans from IA languages into Tibeto-Burman. Is assisting the P.I. in the preparation for publication of a large corpus of Lahu texts.
