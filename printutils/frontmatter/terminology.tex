{\large \parindent=-1em \textbf{Terminology}}
\addcontentsline{toc}{chapter}{}
\vspace{0.25em}

\chapter{Symbols and Abbreviations}\markright{Symbols and Abbreviations}

\renewcommand\thefootnote{*}
\section*{Books, Monographs, Monograph Series\footnotemark}
\footnotetext{Here listed only by author and date.  For full citations see the References, pp.\ \pageref{pg:start-refs}-\pageref{pg:end-refs}.}
\renewcommand\thefootnote{\arabic{footnote}} % set it back
\setcounter{footnote}{0}

\begin{multicols}{2}
\begin{description}
\item[AHD]	\textit{American Heritage Dictionary}
\item[CISTL]	Kitamura, Nishida, and Nagano, eds. (1994)
\item[CSDPN]	Hale (1973)
\item[CTT]	Hyman, ed. (1973)
\item[GL]	Matisoff (1973b/1982)
\item[GSR]	Karlgren (1957)
\item[GSTC]	Matisoff (1985a)
\item[HCT]	Li (1977)
\item[HPTB]	Matisoff (2003)
\item[HRAF]	\textit{Human Relations Area Files} (New Haven)
\item[NHTBM]	Nishi, Matisoff, and Nagano, eds. (1995)
\item[OED]	\textit{Oxford English Dictionary}
\item[OPWSTBL]	\textit{Occasional Papers of the Wolfenden Society on Tibeto-Burman Linguistics}
\item[PPPB]	Luce (1986)
\item[STC]	Benedict (1972)
\item[TBL]	Dai et al., eds. (1992)
\item[TBT]	Weidert (1987)
\item[TSR]	Matisoff (1972a)
\item[UCPL]	\textit{University of California Publications in Linguistics} (Berkeley, Los Angeles, London)
\item[SELAF]	\textit{Société d'Etudes Linguistiques et Anthropologiques de France} (Paris)
\item[VSTB]	Matisoff (1978a)
\item[ZMYYC]	Sun et al., eds. (1991)
\end{description}
\end{multicols}

\section*{Journals}
\begin{multicols}{2}
\begin{description}
\item[AM]	\textit{Asia Major} (Leipzig; London; Taipei)
\item[AO]	\textit{Acta Orientalia} (Copenhagen)
\item[BIHP]	\textit{Bulletin of the Institute of History and Philology} (Taipei)
\item[BMFEA]	\textit{Bulletin of the Museum of Far Eastern Antiquities} (Stockholm)
\item[BSLP]	\textit{Bulletin de la Société de Linguistique de Paris} (Paris)
\item[BSOAS]	\textit{Bulletin of the School of Oriental and African Studies} (London)
\item[GK]	\textit{Gengo Kenky≈´} (Tokyo)
\end{description}
\end{multicols}

\pagebreak

\begin{multicols}{2}
\begin{description}
\item[HJAS]	\textit{Harvard Journal of Asiatic Studies} (Cambridge, MA)
\item[IJAL]	\textit{International Journal of American Linguistics} (Chicago)
\item[JAOS]	\textit{Journal of the American Oriental Society} (New Haven)
\item[JASB]	\textit{Journal of the Asiatic Society of Bengal} (Calcutta)
\item[LTBA]	\textit{Linguistics of the Tibeto-Burman Area} (Berkeley; Chico, CA; Melbourne)
\columnbreak
\item[MSOS]	\textit{Mitteilungen des Seminars für orientalische Sprachen an der königlichen Friedrich-Wilhelms-Universität zu Berlin} (Berlin)
\item[TAK]	\textit{Tōnan Azia Kenkyū} (Southeast Asian Studies) (Kyoto)
\item[ZDMG]	\textit{Zeitschrift der deutschen morgenländischen Gesellschaft} (Wiesbaden)
\end{description}
\end{multicols}

\section*{Conferences}
\begin{multicols}{2}
\begin{description}
\item[ICSTLL]	International Conferences on Sino-Tibetan Languages and Linguistics
\item[SEALS]	Southeast Asia Linguistics Society
\end{description}
\end{multicols}

\section*{Research Units}
\begin{multicols}{2}
\begin{description}
\item[AS]	Academia Sinica (Taipei)
\item[CIIL]	Central Institute of Indian Languages (Mysore)
\item[EFEO]	Ecole Française d'Extrême Orient (Hanoi/Paris)
\item[ILCAA]	Institute for the Study of Cultures of Asia and Africa (Tokyo)
\columnbreak
\item[POLA]	Project on Linguistic Analysis (Berkeley)
\item[SIL]	Summer Institute of Linguistics (\mbox{Dallas})
\item[STEDT]	Sino-Tibetan Etymological Dictionary and Thesaurus (Berkeley)
\end{description}
\end{multicols}

\section*{Languages}
\begin{multicols}{2}
\begin{description}
\item[HM]	Hmong-Mien (= Miao-Yao)
\item[IA]	Indo-Aryan
\item[IE]	Indo-European
\item[Jg.]	Jingpho (= Kachin)
\item[KC]	Kuki-Chin
\item[LB]	Lolo-Burmese
\item[Lh.]	Lahu
\item[MC]	Middle Chinese
\item[OC]	Old Chinese
\item[PIE]	Proto-Indo-European
\item[PLB]	Proto-Lolo-Burmese
\item[PNN]	Proto-Northern Naga
\item[PST]	Proto-Sino-Tibetan
\item[PTB]	Proto-Tibeto-Burman
\item[ST]	Sino-Tibetan
\item[TB]	Tibeto-Burman
\item[TK]	Tai-Kadai
\item[WB]	Written Burmese
\item[WT]	Written Tibetan
\end{description}
\end{multicols}

\section*{Miscellaneous}
\begin{multicols}{2}
\begin{description}
\item[semcat]
\item[pan-allofamic formula ("PAF")]
\item[extra-fascicular etyma]
\item[mesoroots]
\item[supporting forms]
\item["add-sourcing"]
\item["Mariama Fix"]
\item[metastatic flowcharts]
\item[weakly-attested root]
\item[>]	goes to; becomes
\item[<]	comes from; is derivable from
\item[A \STEDTU{⪤} B]		A and B are co-allofams; A and B are members of the same word-family
\item[Clf.]	classifier
\item[JAM]	James A.\ Matisoff
\item[lit.]	literally
\item[OICC]	Obscure internal channels and connections” (see Ch.~III)
\item[ult.]	ultimately
\item[WHB]	William H.\ Baxter
\item[ZJH]	Zev J.\ Handel
\end{description}
\end{multicols}
