\vspace{0.25em}

\renewcommand{\thefootnote}{\arabic{footnote}}
\setcounter{footnote}{0}

\chapter*{Symbols and Abbreviations}\markright{Symbols and Abbreviations}
\addcontentsline{toc}{chapter}{Symbols and Abbreviations}

\renewcommand\thefootnote{*}

\section*{Languages}
\begin{multicols}{2}
\begin{description}
\item[HM]	Hmong-Mien (= Miao-Yao)
\item[IA]	Indo-Aryan
\item[IE]	Indo-European
\item[Jg.]	Jingpho (= Kachin)
\item[KC]	Kuki-Chin
\item[LB]	Lolo-Burmese
\item[Lh.]	Lahu
\item[MC]	Middle Chinese
\item[OC]	Old Chinese
\item[PIE]	Proto-Indo-European
\item[PLB]	Proto-Lolo-Burmese
\item[PNN]	Proto-Northern Naga
\item[PST]	Proto-Sino-Tibetan
\item[PTB]	Proto-Tibeto-Burman
\item[ST]	Sino-Tibetan
\item[TB]	Tibeto-Burman
\item[TK]	Tai-Kadai
\item[WB]	Written Burmese
\item[WT]	Written Tibetan
\end{description}
\end{multicols}

\section*{Miscellaneous}
\begin{multicols}{2}
\begin{description}
% \item[“add-sourcing”]	the often laborious process by which a new source is added to the STEDT database
\item[allofam]	a variant of an etymon
\item[allofam box]	a computer display where all the allofams of an etymon are grouped together within a box on the screen
% \item[ancillary databases]	searchable databases on particular languages or sources, accompaniments to the main STEDT databases (lexical and etymological)
\item[chapter]	the smallest section of a fascicle, containing all the etyma reconstructed with a given gloss, e.g. KIDNEY
% \item[etyma table]	the database table where all STEDT etymologies are stored
\item[extra-fascicular etyma]	etyma that are cited outside of their normal fascicle to illustrate a particular point
\item[fascicle]	a major section of a STEDT volume, comprising many chapters with closely related glosses, e.g. INTERNAL ORGANS
% \item[lexicon table]	the database table where the forms from individual languages are stored
\item[metastatic flowchart]	a diagram illustrating the patterns of semantic association displayed by an etymon
\item[meso-reconstruction]	reconstruction of an etymon at the subgroup level, deemed to be descended from that same etymon at a higher taxonomic level
\item[meso-root]	an etymon reconstructed at the subgroup level
\item[pan-allofamic formula (PAF)]	an abstract reconstruction intended to display simultaneously all the patterns of variation attested for a given etymon
\item[rectification]	the process by which etymologies are subjected to reanalysis, to determine whether they should be accepted as is, accepted with modifications, or rejected
\item[root canal]	the process through which a STEDT etymology is made public, so that it can be commented upon 
\item[semantic flowchart]	SEE metastatic flowchart
\item[semcat]	a semantic category; each STEDT etymology is assigned to one or more of them
\item[supporting forms]	forms from individual languages that support an etymology
\item[volume]	one of the ten major divisions of STEDT, e.g. BODY-PARTS
\item[weakly attested root]	a promising etymology, but with insufficient support
\item[>]	goes to; becomes
\item[<]	comes from; is derivable from
\item[A \STEDTU{⪤} B]	A and B are co-allofams; A and B are members of the same word-family
\item[A \STEDTU{↭} B]	Are A and B co-allofams? Do A and B belong to the same word family?
\item[A \STEDTU{↮} B]	A and B are not co-allofams; A and B do not belong to the same word family
\item[Clf.]	classifier
\item[JAM]	James A.\ Matisoff
\item[lit.]	literally
% \item[OICC]	Obscure internal channels and connections‚ (see Ch.~III)
\item[ult.]	ultimately
\item[WHB]	William H.\ Baxter
% \item[ZJH]	Zev J.\ Handel
\end{description}
\end{multicols}
