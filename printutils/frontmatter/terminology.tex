\vspace{0.25em}

\renewcommand{\thefootnote}{\arabic{footnote}}
\setcounter{footnote}{0}

\chapter*{Symbols and Abbreviations}\markright{Symbols and Abbreviations}

\renewcommand\thefootnote{*}

\section*{Languages}
\begin{multicols}{2}
\begin{description}
\item[HM]	Hmong-Mien (= Miao-Yao)
\item[IA]	Indo-Aryan
\item[IE]	Indo-European
\item[Jg.]	Jingpho (= Kachin)
\item[KC]	Kuki-Chin
\item[LB]	Lolo-Burmese
\item[Lh.]	Lahu
\item[MC]	Middle Chinese
\item[OC]	Old Chinese
\item[PIE]	Proto-Indo-European
\item[PLB]	Proto-Lolo-Burmese
\item[PNN]	Proto-Northern Naga
\item[PST]	Proto-Sino-Tibetan
\item[PTB]	Proto-Tibeto-Burman
\item[ST]	Sino-Tibetan
\item[TB]	Tibeto-Burman
\item[TK]	Tai-Kadai
\item[WB]	Written Burmese
\item[WT]	Written Tibetan
\end{description}
\end{multicols}

\section*{Miscellaneous}
\begin{multicols}{2}
\begin{description}
\item[semcat]
\item[pan-allofamic formula (“PAF”)]
\item[extra-fascicular etyma]
\item[mesoroots]
\item[supporting forms]
\item[“add-sourcing”]
\item[“Mariama Fix”]
\item[metastatic flowcharts]
\item[weakly-attested root]
\item[>]	goes to; becomes
\item[<]	comes from; is derivable from
\item[A \STEDTU{⪤} B]		A and B are co-allofams; A and B are members of the same word-family
\item[Clf.]	classifier
\item[JAM]	James A.\ Matisoff
\item[lit.]	literally
\item[OICC]	Obscure internal channels and connections‚ (see Ch.~III)
\item[ult.]	ultimately
\item[WHB]	William H.\ Baxter
\item[ZJH]	Zev J.\ Handel
\end{description}
\end{multicols}
