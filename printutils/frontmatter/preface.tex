\vspace{0.25em}

\renewcommand{\thefootnote}{\arabic{footnote}}
\setcounter{footnote}{0}

\chapter*{Preface}

My involvement in Tibeto-Burman (TB) and Sino-Tibetan (ST) comparative reconstruction dates from my first fieldwork on Jingpho, Burmese, and Lahu in the 1960's, and especially from my intense contact with Paul K. Benedict when I was teaching at Columbia University (1966-69). The manuscript version of Benedict's Sino-Tibetan: a Conspectus (STC) had been lying around unpublished since its composition around 1940; it was exciting for me to contribute to its eventual publication in 1972. With its nearly 700 TB cognate sets, and over 300 TB/Chinese comparisons, the Conspectus ushered in the current renaissance of TB and ST comparative linguistics. Its rigor and precision, as well as the breadth of its vision, have made it the indispensable point of departure for subsequent work in the field. While there is certainly room for tinkering with a few details of Benedict's reconstructive scheme for Proto-Tibeto-Burman (PTB), the major features of the system itself remain basically unassailable. The real progress that has been made in the past 30 years lies elsewhere. An avalanche of new data from recent fieldwork has strengthened the support for previously reconstructed etyma and has permitted the reconstruction of hundreds of new roots at all taxonomic levels of TB, though many more undoubtedly remain to be discovered. The harnessing of the computer for etymological research has speeded up the identification of new cognates and provided a powerful tool for testing the validity of proposed reconstructions. A better understanding of the variational processes at work in TB and ST word-families has enabled us to decide more accurately whether sets of forms that bear partial phonosemantic resemblances to each other are really variants of the same etymon or etymologically independent. On the Chinese side, the successors to Karlgren have made profound changes in the reconstructive scheme for Old Chinese, and it is no exaggeration to say that the field of historical Sinology is now going through a period of ferment. Still, almost all of STC's suggested Chinese comparanda for PTB etyma have gone unchallenged.\footnote{The over 300 TB/OC comparisons made in STC are conveniently indexed in the excellent review by Chou Fa-kao (1972).}

{\tt Despite its brilliance, the Conspectus is notoriously difficult to use, largely due to its complex apparatus of footnotes, which often (especially in the Chinese section) occupy more of the page than the text itself. These notes include Benedict's original ones from the 1940's, as well as those he and I added before publication in 1972. Some 200 valid etymologies are squirreled away in these convoluted notes, but they also contain a number of errors, unsubstantiated speculations, and over-complications.

Benedict himself realized the limitations of the data he had to work with, and never intended STC to be more than an overview or ``conspectus'' of its vast subject. Neither did he structure it as a practical handbook which systematically tabulated the sound cor- respondences among the major languages of the family at all canonical points of the syllable. (Such information is certainly extractable from the terse but labyrinthine pages of STC, but at the cost of considerable labor.) Towards the end of his life Benedict does seem to have felt the need to embark on such a systematic project, although it never actually got off the ground.}

Some reconstructions are given at the subgroup level, when they are available, and a number of roots are marked as being confined to certain subgroups (e.g. Himalayan, Kiranti, Kamarupan, Lolo-Burmese, Karenic). It is precisely these roots of limited distribution, or ``cognate isoglosses'', that will prove to be important for a finer subgrouping of the TB family. However, new data frequently forces us to revise our judgments of etyma distribution: many roots considered to be confined to a single subgroup in STC must now be set up for TB as a whole. These are usually noted in the text.
As emphasized in the Conclusion (Ch. XIII), the approach of this Handbook is definitely conservative, in that speculative etymologies are almost always avoided, or at any rate suitably hedged. Variational phenomena are handled with care; phonosemantically non-identical roots are not claimed to be co-allofams unless the morphophonemic relationship between them is paralleled in other word families. Semantic leaps are kept to a minimum, and detailed justification is provided when the meanings of putative cognates diverge significantly. Many solid Chinese comparanda to TB etyma are offered, but no attempt is made to choose among the often contradictory reconstructive schemes for Old Chinese;\footnote{See ``A Concise Introduction to Old Chinese Phonology'' by Zev Handel (below, Appendix A), which treats the major differences in the reconstructive systems of Karlgren, Li Fang-Kuei, and W.H. Baxter.} for now I just use the classic reconstructions of Karlgren (with some modifications\footnote{One minor change is that we write the velar nasal as ``{\ng}'' instead of ``ng''.}), a policy which STC also followed.\footnote{Despite of the fact that Karlgren's system has been superseded and simplified in some respects by subsequent scholars, GSR remains the best-known, most copious, and most convenient reference for OC. I conventionally do not precede OC reconstructions with an asterisk. Asterisks do, however, appear before the OC forms cited in Appendix A.} I usually have not tried to set up PST forms, as STC sporadically tries to do. I just give the best comparanda. That is why this is basically a Tibeto-Burman handbook, even though its system and methodology apply to all of Sino-Tibetan (hence the subtitle).
The primary organization of this Handbook is by rhyme, since this is the most stable part of the syllable.\footnote{Hence the great utility of rhyming dictionaries for TB languages; Benedict put several such to good use during the compilation of the Conspectus.} In sharp contrast to Indo-European, the manner of initial consonants (voicing and aspiration) in TB/ST is highly variable, due to the pervasive phenomenon of prefixation (see Ch. IV). Chinese comparanda (I usually avoid the term ``cognate'') are given mostly under the proto-rhyme of their TB counterparts. Most correspondence charts of reflexes also appear under the rhymes. Still there is a certain unavoidable repetitiveness, in that the same root might be discussed in different contexts, e.g. with respect to its initial, its rhyme, and/or its variational pattern. The Indexes will facilitate finding all references to a given etymon.
A few words about nomenclatural and transcriptional matters:\footnote{For more details about the transcriptional systems used for key languages, see Citational and Transcriptional Conventions, below.}
\begin{itemize}

\item Names for TB languages have undergone frequent changes, as exonyms are replaced by autonyms, and as names felt to be pejorative become politically incorrect.\footnote{For a discussion of the issues surrounding the proliferation of language names in TB, see JAM 1986a.} However, certain older language names have been retained, just because they are more widely used in the literature: thus I use ``Lushai'' instead of the now-preferred self-designation ``Mizo''.
\item Subgroup names can be particularly confusing. Occasionally I use equivalent names for the same subgroup, e.g. ``Himalayish'' or ``Himalayan'', ``Bodo-Garo'' or ``Barish'', ``Kuki-Naga'' or ``Kuki-Chin-Naga''. My use of ``Kamarupan'' as a geographical cover term for the subgroups of Northeast India (including Abor-Miri-Dafla, Bodo-Garo, and Kuki-Chin-Naga) remains controversial, although it is certainly useful.\footnote{See JAM 1999c (``In defense of `Kamarupan' '').}
\item Tones are not marked for every language that has them, especially not for those where no good tonological description is available. Tones are consistently marked for Lolo-Burmese languages and for Jingpho, as well as for the tonal languages cited in Sun et al., 1991 (ZMYYC) and Dai et al., 1992 (TBL); but they are only sporadically provided for such languages as Lushai and Lai Chin.
\end{itemize}

Great care has been taken to ascribe etymologies to their original source. Any TB etymology or part thereof not specifically ascribed to a prior source is original with me, as far as I know. In any case, the responsibility for the TB reconstructions is mine alone.

It is hoped that this Handbook will prove useful to specialists and general linguists alike, and that it will help to demystify the most important understudied language family in the world.