\vspace{0.25em}

\renewcommand{\thefootnote}{\arabic{footnote}}
\setcounter{footnote}{0}

\chapter*{Preface}

My involvement in Tibeto-Burman (TB) and Sino-Tibetan (ST) comparative reconstruction dates from my first fieldwork on Jingpho, Burmese, and Lahu in the 1960's, and especially from my intense contact with Paul K. Benedict when I was teaching at Columbia University (1966-69). The manuscript version of Benedict's Sino-Tibetan: a Conspectus (STC) had been lying around unpublished since its composition around 1940; it was exciting for me to contribute to its eventual publication in 1972. With its nearly 700 TB cognate sets, and over 300 TB/Chinese comparisons, the Conspectus ushered in the current renaissance of TB and ST comparative linguistics. Its rigor and precision, as well as the breadth of its vision, have made it the indispensable point of departure for subsequent work in the field. While there is certainly room for tinkering with a few details of Benedict's reconstructive scheme for Proto-Tibeto-Burman (PTB), the major features of the system itself remain basically unassailable. The real progress that has been made in the past 30 years lies elsewhere. An avalanche of new data from recent fieldwork has strengthened the support for previously reconstructed etyma and has permitted the reconstruction of hundreds of new roots at all taxonomic levels of TB, though many more undoubtedly remain to be discovered. The harnessing of the computer for etymological research has speeded up the identification of new cognates and provided a powerful tool for testing the validity of proposed reconstructions. A better understanding of the variational processes at work in TB and ST word-families has enabled us to decide more accurately whether sets of forms that bear partial phonosemantic resemblances to each other are really variants of the same etymon or etymologically independent. On the Chinese side, the successors to Karlgren have made profound changes in the reconstructive scheme for Old Chinese, and it is no exaggeration to say that the field of historical Sinology is now going through a period of ferment. Still, almost all of STC's suggested Chinese comparanda for PTB etyma have gone unchallenged.\footnote{The over 300 TB/OC comparisons made in STC are conveniently indexed in the excellent review by Chou Fa-kao (1972).}

{\tt Despite its brilliance, the Conspectus is notoriously difficult to use, largely due to its complex apparatus of footnotes, which often (especially in the Chinese section) occupy more of the page than the text itself. These notes include Benedict's original ones from the 1940's, as well as those he and I added before publication in 1972. Some 200 valid etymologies are squirreled away in these convoluted notes, but they also contain a number of errors, unsubstantiated speculations, and over-complications.

Benedict himself realized the limitations of the data he had to work with, and never intended STC to be more than an overview or ``conspectus'' of its vast subject. Neither did he structure it as a practical handbook which systematically tabulated the sound cor- respondences among the major languages of the family at all canonical points of the syllable. (Such information is certainly extractable from the terse but labyrinthine pages of STC, but at the cost of considerable labor.) Towards the end of his life Benedict does seem to have felt the need to embark on such a systematic project, although it never actually got off the ground.}

The present work may be viewed largely as an updating, clarification, and expansion of STC. It aims to build on the valid etymologies already proposed, but also to present new ones that conform to established sound correspondences. When necessary, previously proposed etymologies are modified in order to accommodate new data.
In this Handbook, I have organized the discussion according to the inventory of proto-entities at the various points of the syllable: initial consonants; medial glides; prefixes; simple and diphthongal vocalic nuclei; closed syllable rhymes (with final nasals, stops, liquids, and -s); and suffixes.\footnote{Similarly organized examples of the Handbook genre in Southeast Asia include Li Fang-Kuei's {\it A Handbook of Comparative Tai} (1977) and Wang Fushi's {\it Mi\'aoy\v{u} fangy\'an sh\=engy\`unm\v{u} b\v{i} ji\`ao} (Comparison of the Initials and Rhymes of the Miao Dialects; 1979).} Wherever possible, the regular reflexes in major languages of these syllabic elements are displayed in tabular form. The best etymologies illustrating each sound-correspondence are presented, and exceptional or problematic cases are discussed, with alternative analyses suggested.

That is the ``systematic'' part. The ``philosophical'' aspects of this book are more elusive, but implicit throughout. First of all, I have striven for clarity and simplicity of presentation, for ``user-friendliness''. Being understandable rather than obscurantist poses certain risks, in that one's opinions are clear and therefore falsifiable in the light of new data, but it has the advantage of encouraging feedback from others.\footnote{The difficulty of STC can be used as an excuse for not studying it thoroughly. It would be tragic if its fundamental insights were to be forgotten.} Secondly, I operate under a theoretical framework according to which the proto-lexicon is not conceived of in terms of monolithic, phonosemantically invariant etyma, but rather as a collection of word families that may each exhibit some internal variation on both the phonological and semantic planes, but according to certain reasonable principles. Distinguishing between such valid variational phenomena and wild speculative leaps is not always easy.

* * *

{\tt After the publication of the Conspectus, further progress in intra-TB and TB/Chinese comparison seemed to depend on multiplying the number of reliably reconstructed etyma, as well as systematizing and refining the methodological underpinnings of the reconstructions. In the mid-1970's, when I was attempting to apply the principles of glottochronology in order to subgroup the TB family, the very first item of ``basic vocabulary'' that I looked at happened to be `belly / stomach'. Much to my initial dismay, I quickly found that it was futile to use a simple wordlist to try to subgroup a family as complex and ramified as TB. In fact it was impossible even to deal in isolation with a single point in semantic space; etyma with the meaning `belly' or `stomach' spilled over into concepts like `cave / hole', `swelling', `calf of leg', `liver', `guts', etc. I became preoccupied with notions of semantic variability, semantic fields, and the field of bodypart nomenclature in particular. At the same time I could not help noticing the morphophonemic variations displayed by almost every etymon previously or newly reconstructed. Instead of guiltily sweeping these variational phenomena under the rug, I began to revel in them. In Variational Semantics in Tibeto-Burman (1978) I set out to establish an explicit methodology for handling phonosemantic variation in word families, introducing the notion of allofams and a notation for diagramming patterns of semantic association (``metastatic flowcharts'').
In those pre-computer days, I naturally had to assemble my data by hand, copying out bodypart words from dictionaries and sorting them into synonym sets on filecards, then grouping them into putative cognate sets. The older sources used by Shafer and Benedict were supplemented by an ever-increasing volume of new material in the 1970's and 1980's, much of it from post-Cultural Revolution China, but also from India, Nepal, and Thailand. It eventually became apparent that the job of digesting these massive amounts of new and old data would be vastly facilitated by the use of computers.
The hitch was my own ignorance of computer technology beyond the level of simple word-processing. Fortunately I somehow got the idea of applying to federal granting agencies for a longterm project to create a computerized etymological dictionary of Tibeto-Burman / Sino-Tibetan based on semantic principles, i.
e. an etymological thesaurus.\footnote{The shining example of an etymological thesaurus in the field of Indo-European is Carl Darling Buck's {\it A Dictionary of Selected Synonyms in the Principal Indo-European Languages} (1949).} In 1987, the Sino-Tibetan Etymological Dictionary and Thesaurus Project (STEDT) got under way, funded jointly by the National Science Foundation and the National Endowment for the Humanities.
Thanks to the efforts of a succession of computer-savvy graduate students (see the Acknowledgments), a massive lexical database of forms from over 250 TB languages and dialects has been created, mostly of bodypart terminology at first, but rapidly extending to other areas of the lexicon. It has been a race between the vertiginous progress of computer technology (when we started in the Pleistocene, 1987, we were using Mac Pluses!) and our ever-expanding needs for disk capacity, memory, and operating speed. The hardwon experience gained at the STEDT project has inspired similar lexical database projects in the U.S. and abroad.

It was originally planned to publish the Sino-Tibetan Etymological Dictionary and Thesaurus as a series of printed volumes, each containing full details on all the etymologies in a given semantic area, starting with bodyparts and then proceeding to animal names, natural objects, verbs of motion, and all the rest of the lexicon. The sheer amount of the etymologizable data soon made it clear that this was unrealistic, and that each projected volume of STEDT would have to be split up into smaller units or ``fascicles'', e.g. in the case of bodyparts into ten subdivisions including head, limbs, internal organs, diffuse organs, reproductive system, etc., each to be published separately. I decided to start with the reproductive system, not only because of its prurient interest but also because it seemed like the point of departure for all things. Accordingly a printed manuscript of some 480 pages was produced in 1997-98, called Sino-Tibetan Etymological Dictionary and Thesaurus, Volume I: Bodyparts, Fascicle 1: The Reproductive System, containing 286 pages of forms assembled into 174 cognate sets, divided into nine chapters: (1) Egg, (2) Birth, (3) Navel, (4) Breast, (5) Vagina, (6) Womb, (7) Penis, (8) Copulate, (9) Body Fluids. As part of the front matter, I put together a 60 page essay on the initial consonants and consonant clusters of Proto-Tibeto-Burman.}

As it turned out, perhaps fortunately, that introductory essay soon took on a teratoid life of its own, and became an example of what one might call in Proto-Tibeto-Burman 
\begin{quote}
\begin{center}
\begin{tabular}{c c c c c}
*k$^w${\textschwa}y & l{\textschwa}tak & r{\textschwa}may & g{\textschwa}ya{:}p & way\\
dog & ACC & tail & wag & COP/NOM
\end{tabular}
\end{center}
\end{quote}
or ``the tail wagging the dog''.\footnote{The presence of the accusative particle l{\textschwa}tak is motivated by the semantic anomalousness of this phrase, which has also caused the fronting of the object *k$^w${\textschwa}y `dog' to initial position.} Was I not responsible for dealing with the whole proto-syllable, not just the initial consonants? I delayed publication of the ``Reproductive Fascicle'' until I could get the whole job done. The ``introductory essay'', then entitled System and Philosophy of Tibeto-Burman Reconstruction, eventually grew to its present length of some 600 pages. It gradually dawned on me that it would be preferable to publish it as a stand-alone book, indeed a Handbook.
This decision has much to recommend it. In its present form, the phonological approach of this Handbook is complementary to the main thrust of the STEDT project, which is semantically organized. Both prongs of attack are certainly necessary. Henceforth each set of etymologies in the various semantic areas of the lexicon can be put up on the worldwide web as soon as they are deemed ready to go, rather than waiting until they can appear in print form. Many trees will be spared as reams of paper are saved. As each series of etymologies is released, it will be possible to solicit comments and criticisms from colleagues all over the world, and it will be simplicity itself to incorporate any addenda or corrigenda. It is extremely wasteful of space to print out computer records from a database -- who wants to see the gloss `egg' printed out hundreds of times? Since STEDT has had a policy of ``following copy'', the same form from a given language (especially well documented ones like Written Burmese or Written Tibetan) is likely to appear several times in slightly different transcriptions used in the various sources. Instead of trying to ``normalize'' these, or indeed to delete totally identical records from different sources, we can just include them all, thereby saving much drudgery, since space will not be an issue.
Perhaps the greatest advantage of having this Handbook appear before the semantically organized etymologies are promulgated is that it can serve as a standard or ``template'' against which each newly proposed etymon can be tested. Let us say, e.g., that a hypothetical new PTB root *b-zer-s has been reconstructed with the meaning `tonsil'. The supporting forms for this etymology can then be compared for consistency with other data that motivate the reconstructions of the same proto-elements, i.
e. other etyma with prefixal *b- (\S 4.
4.
3), with initial *z- (\S 3.
3), with the liquid-final rhyme *-er (\S 9.
2.
3), and with suffixal *-s (\S 11.
4). Before long the Handbook itself can be put up on the web, so that these new etymologies may be plugged directly into it.
Much obviously remains to be done. The data are still uneven in the various branches of the family, ranging from the overwhelmingly copious to the tantalizingly sparse. Most strikingly perhaps, this Handbook makes no attempt to reconstruct tones at the PTB level, although this can already be done at the level of certain individual subgroups (e.g. Lolo-Burmese, Tamangic, Karenic).
Some reconstructions are given at the subgroup level, when they are available, and a number of roots are marked as being confined to certain subgroups (e.g. Himalayan, Kiranti, Kamarupan, Lolo-Burmese, Karenic). It is precisely these roots of limited distribution, or ``cognate isoglosses'', that will prove to be important for a finer subgrouping of the TB family. However, new data frequently forces us to revise our judgments of etyma distribution: many roots considered to be confined to a single subgroup in STC must now be set up for TB as a whole. These are usually noted in the text.
As emphasized in the Conclusion (Ch. XIII), the approach of this Handbook is definitely conservative, in that speculative etymologies are almost always avoided, or at any rate suitably hedged. Variational phenomena are handled with care; phonosemantically non-identical roots are not claimed to be co-allofams unless the morphophonemic relationship between them is paralleled in other word families. Semantic leaps are kept to a minimum, and detailed justification is provided when the meanings of putative cognates diverge significantly. Many solid Chinese comparanda to TB etyma are offered, but no attempt is made to choose among the often contradictory reconstructive schemes for Old Chinese;\footnote{See ``A Concise Introduction to Old Chinese Phonology'' by Zev Handel (below, Appendix A), which treats the major differences in the reconstructive systems of Karlgren, Li Fang-Kuei, and W.H. Baxter.} for now I just use the classic reconstructions of Karlgren (with some modifications\footnote{One minor change is that we write the velar nasal as ``{\ng}'' instead of ``ng''.}), a policy which STC also followed.\footnote{Despite of the fact that Karlgren's system has been superseded and simplified in some respects by subsequent scholars, GSR remains the best-known, most copious, and most convenient reference for OC. I conventionally do not precede OC reconstructions with an asterisk. Asterisks do, however, appear before the OC forms cited in Appendix A.} I usually have not tried to set up PST forms, as STC sporadically tries to do. I just give the best comparanda. That is why this is basically a Tibeto-Burman handbook, even though its system and methodology apply to all of Sino-Tibetan (hence the subtitle).
The primary organization of this Handbook is by rhyme, since this is the most stable part of the syllable.\footnote{Hence the great utility of rhyming dictionaries for TB languages; Benedict put several such to good use during the compilation of the Conspectus.} In sharp contrast to Indo-European, the manner of initial consonants (voicing and aspiration) in TB/ST is highly variable, due to the pervasive phenomenon of prefixation (see Ch. IV). Chinese comparanda (I usually avoid the term ``cognate'') are given mostly under the proto-rhyme of their TB counterparts. Most correspondence charts of reflexes also appear under the rhymes. Still there is a certain unavoidable repetitiveness, in that the same root might be discussed in different contexts, e.g. with respect to its initial, its rhyme, and/or its variational pattern. The Indexes will facilitate finding all references to a given etymon.
A few words about nomenclatural and transcriptional matters:\footnote{For more details about the transcriptional systems used for key languages, see Citational and Transcriptional Conventions, below.}
\begin{itemize}

\item Names for TB languages have undergone frequent changes, as exonyms are replaced by autonyms, and as names felt to be pejorative become politically incorrect.\footnote{For a discussion of the issues surrounding the proliferation of language names in TB, see JAM 1986a.} However, certain older language names have been retained, just because they are more widely used in the literature: thus I use ``Lushai'' instead of the now-preferred self-designation ``Mizo''.
\item Subgroup names can be particularly confusing. Occasionally I use equivalent names for the same subgroup, e.g. ``Himalayish'' or ``Himalayan'', ``Bodo-Garo'' or ``Barish'', ``Kuki-Naga'' or ``Kuki-Chin-Naga''. My use of ``Kamarupan'' as a geographical cover term for the subgroups of Northeast India (including Abor-Miri-Dafla, Bodo-Garo, and Kuki-Chin-Naga) remains controversial, although it is certainly useful.\footnote{See JAM 1999c (``In defense of `Kamarupan' '').}
\item Tones are not marked for every language that has them, especially not for those where no good tonological description is available. Tones are consistently marked for Lolo-Burmese languages and for Jingpho, as well as for the tonal languages cited in Sun et al., 1991 (ZMYYC) and Dai et al., 1992 (TBL); but they are only sporadically provided for such languages as Lushai and Lai Chin.
\end{itemize}

Great care has been taken to ascribe etymologies to their original source. Any TB etymology or part thereof not specifically ascribed to a prior source is original with me, as far as I know. In any case, the responsibility for the TB reconstructions is mine alone.

It is hoped that this Handbook will prove useful to specialists and general linguists alike, and that it will help to demystify the most important understudied language family in the world.