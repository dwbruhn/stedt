\documentclass[10pt]{article}
\usepackage[polutonikogreek,latin,english]{babel}
\usepackage{fancyhdr}
\pagestyle{plain}
\usepackage{lineno}
\usepackage{float}
\usepackage{acronym}
\usepackage{microtype}
\usepackage{amsmath,amssymb}
\exhyphenpenalty=25000
\usepackage{fullpage}
\usepackage{titlesec}
\usepackage{hyperref}
\usepackage{tikz}
%\usepackage{pslatex}
\usepackage{color,colortbl}
\usetikzlibrary{decorations.pathreplacing,calc}
\usepackage{tikz-qtree}
\newcommand{\tikzmark}[1]{\tikz[overlay,remember picture] \node (#1) {};}
\usepackage[T1]{fontenc}
\usepackage{graphicx}
\usepackage{covington}
\def\citeapos#1{\citeauthor{#1}'s (\citeyear{#1})}
\usepackage{natbib}
\Roman{section}
\bibpunct[: ]{(}{)}{;}{a}{}{,}
\usepackage{tipa}
\begin{document}
\section{Conclusion}
One cannot very well end a book with the word ``horse-leech'', and so a few concluding philosophical remarks seem appropriate. Perhaps the best way to organize this discussion is in terms of a set of adjectives with the -ive suffix.
\subsection{Cumulative}
In linguistics as in other disciplines, it is a constant temptation to try to overthrow the work of one's predecessors, so that one's own research will appear to be the {\it fons et origo} of the truth.\footnote{I have referred to this phenomenon in Freudian terms as ``patricidal'' (keynote address at the Summer Institute of the Linguistic Society of America, Ann Arbor 1973).}
This tendency has been especially characteristic of generative grammar, where ``theories of language'' have a built-in planned obsolescence, with each new theory claiming to invalidate all previous ones.

Historical linguists are hardly exempt from this primal urge for revolutionary novelty  the desire to be different just for the sake of being different. This can take many forms, some of them trivial and innocuous, like replacing a phonetic symbol in a previous reconstruction by a new but equivalent one; or changing the name of a subgroup of a language family.\footnote{Hence the proliferation of alternate names even for well-established subgroups like Lolo-Burmese (Burmese-Lolo, Yi-Burmese, Burmese-Yi, Burmese-Yipho, Yi-Myanmar, Myanmar-Yipho, etc.). T. Shintani has recently (2002) proposed the euphonious neologism ``Brakaloungic'' for Karenic.}
More serious is the itch to carve out totally novel subgroupings,\footnote{I have called this ``neosubgroupitis'' (JAM 2000b ``On Sino-Bodic''). Trying to establish higher-order combinations of TB subgroups is premature at best. Even Indo-Europeanists are still unable to do so unequivocally for their much better documented family. See the discussion in JAM 1978a (VSTB):1-12.} a process rather similar to the decennial gerrymandering of congressional districts in the House of Representatives. At an extreme level we find ``megalocomparative'' proposals of genetic relationship that turn received notions upside down (e.g. Sino-Mayan, Sino-Caucasian, Sino-Austronesian, Japanese-Dravidian), and which can lead the unwary down fruitless paths, obscuring the differences among cognates, borrowings, and chance resemblances.\footnote{See JAM 1990a (\xd2 On megalocomparison\xd3 ). Megalocomparison has the apparent advantage of non-falsi\xde ability, since, as Haudricourt has observed, one can never prove that any two languages are not related. But non-falsi\xde able hypotheses are not scienti\xde c. When presented with alternative non-falsi\xde able proposals it is impossible to choose among them.}

Perpetual revolution gets to be fatiguing after a while.\footnote{As Leon Trotsky found to his cost in 1940.} Surely it is preferable to build on the past rather than to repudiate it. Historical linguistics is a cumulative enterprise. Thanks to the foundation laid by pioneering scholars, especially Paul K. Benedict, a solid body of TB/ST etymological knowledge has been accumulated, in terms of which new etymologies must be evaluated. No longer can one get away with reconstructing whatever one pleases, no matter how typologically bizarre or adhoc or mechanistic the reconstruction might be.\footnote{Those who lack what I have called \xd2 Proto-Sprachgef\x9f hl\xd3  (JAM 1982a) can produce reconstructions bristling with strange symbols but devoid of any phonetic or typological plausibility; see e.g. Sedl|a\xc7 cek 1970; Weidert 1975, 1979, 1981, 1987; Peiros & Starostin 1996.}

There is a dialectical relationship between synchronic data and sound laws. The ``laws'' are derived by inference from the data in the first place, but once proto-forms are reconstructed, they can be used to guide us in our hunt for cognates in languages not yet examined (even if they have undergone semantic change). Almost every TB/ST etymology so far proposed presents problems and complications  irregularities  in some language or other, which is par for the course even in the much better known Indo-European family. Part of our task is to indicate where the exceptions, problems, and irregularities lie, in the hope that they can ultimately be explained.

The concept of ``regularity'' itself is by no means simple, nor does it mean the same thing to different scholars.\footnote{See JAM 1992 (\xd2 Following the marrow\xd3 ) and 1994a (\xd2 Regularity and variation\xd3 ).} The Nostraticist or Sino-Mayanist can convince himself that his fantastical comparisons are ``perfectly regular''. Given sufficient semantic latitude and proto-forms that are complex enough, one can formulate ``sound laws'' in such a way that they appear exceptionless. Paradigmatically one can multiply the number of proto-phonemes. If you reconstruct 35 proto-vowels, any anomalous vowel correspondence can be regarded as ``regularly reflecting'' a separate proto-vowel. Syntagmatically, if you reconstruct etyma like *mrgsla, and the monstrous proto-cluster *mrgsl- occurs only in a single etymon, any set of reflexes in the daughter languages can be said to be ``regular''.\footnote{This is actually the proto-form offered in Weidert 1981:25 for an etymon meaning \xd4 spirit, ghost, shadow\xd5  (reconstructed as *m-hla in STC #475). As I have observed, \xd2 It is always possible and sometimes necessary to invent an ad hoc explanation for an anomalous case. It is even true that some such ad hoc \xd4 solutions\xd5  are more plausible than others. The only harm is in deluding oneself that an explanation which covers only a single case establishes a \xd4 regularity\xd5 .\xd3  (JAM 1982a:22).}

As the alternative to such ``proto-form stuffing'', one must have recourse to proto-variation, though in a controlled and constrained way. Not everything may be said to vary with everything else.\footnote{This issue is the major theme of JAM 1978a (VSTB).} This Handbook places special emphasis on variational patterns in TB/ST, and attempts to classify them according to how well attested they are.\footnote{Note the large percentage of PTB roots for which proto-variation is posited in the Index of Proto-Forms, below.} As I put it 30 years ago, ``We must steer an Aristotelian middle path between a dangerous speculativism and a stodgy insensitivity to the workings of variational phenomena in language history.''\footnote{JAM 1972b (\xd2 Tangkhul Naga and comparative TB\xd3 ):282.}

The time-depth of PST is perhaps 6000 years B.P., about at the limits of the comparative method. We can hardly afford to insist on ``perfect regularity'', though we must never settle for a roseate Greenbergian haze.\footnote{See Greenberg 1987, and my review of it (JAM 1990a).}

\subsection{Self-corrective}
A few of our etyma are only set up provisionally, and some individual forms are assigned only tentatively to a certain etymon. It must be admitted that a lot of guesswork is involved in etymologizing material from hundreds of languages and dialects at once, without having established the ``soundlaws'' in advance. The problems are especially acute when comparing phonologically depleted languages with those having richer syllable canons. When there is a partial phonological similarity between distinct etyma with the same meaning (e.g. *sem and *sak `mind / breath'; *mu\xf2 r and *muk `mouth'; *s-ma\xf2 y and *s-mel `face'; *s-r(y)ik and *s(y)ar `louse'), it is not easy to decide by simple inspection to which etymon we should assign a phonologically slight form in a daughter language (e.g. s\x81  `mind', m\xbf  `mouth', hm\xe4  `face').\footnote{See JAM 1994a (\xd2 Regularity and variation\xd3 ):54-55.}

There are all too many ways in which one can make etymological mistakes. A rough taxonomy of errors would have to include the following:Treating a loanword as nativeI was at first delighted when I ran across the Jingpho form w|e\xd6 -wu `screw', since its first syllable looked like an excellent match with Lahu \\\xbf -v\\\xe4 \xd6  `id.', for which I then had no etymology. Could this be a precious example of the rare PTB rhyme *-ek ?\footnote{See above, 8.4(1).}
But the screw is hardly an artifact of any great antiquity, and it would be prima facie implausible that a root with such a meaning would have existed in PTB. The truth quickly became apparent. The modern Burmese form for `screw', w|\xe4 \xd6 -\xd6 u (WB wak-\xd6 u), the obvious source from which both Jingpho and Lahu borrowed these words, means literally ``pig-intestine''. The semantic association is the squiggly corkscrew-like appearance of a pig's small intestine.\footnote{See the photographs of a pig being butchered in a Lahu village in JAM 1978a, between pp. 168 and 169. This same semantic association is to be found with the root *ri\xf2 l, above 9.3.2(3).} This etymology is also interesting from the viewpoint of distinguishing native vs. borrowed co-allofams. The usual, native words for `pig' in Jingpho and Lahu are w\\a\xd6  and v\\a\xd6 , respectively; but the doublets borrowed from Burmese have front vowels, as in spoken Burmese. Unless a native speaker of Jingpho knows Burmese, s/he is unlikely to realize that the first syllable of w|e\xd6 -wu means `pig', especially since this syllable is in the high-stopped tone, while `pig' is low-stopped. The native Lahu speaker is even less likely to recognize the source of \\\xbf -v\\\xe4 \xd6 , since the morpheme for `intestine' has been completely dropped from the original Burmese compound,\footnote{The Lahu cognate to WB \xd6 u \xd4 intestine\xd5  is \xa9 \\u (usually in the compound \\\xbf -\xa9 \\u-t\xc8 \xe4 \xd6 ).} rather like the way our word camera (< Lat. `room; chamber; vaulted enclosure') is a shortening of the old compound camera obscura (``dark chamber'').\footnote{There is a difference in detail between the two cases, however: the deleted \xd4 intestine\xd5  is the head of the compound \xd2 pig-intestine\xd3 , but the deleted obscura is the modi\xde er in the collocation \xd2 dark-chamber\xd3 .}

\subsection{Combining reflexes of unrelated roots}
When two forms bearing a semantic resemblance in a phonologically depleted language differ only in tone, it is tempting to try to relate them. I once entertained the possibility that such pairs of Lahu forms as phu `silver, money' / ph\xc8 u `price, cost' and mu `high, tall' / m\xc8 u `sky' were co-allofams, though they can easily be shown to descend from quite separate etyma: phu < PTB *plu (STC p. 89) / ph\xc8 u < PTB *p\xfa w (STC #41); mu < PTB *mra\xb3  (STC p. 43) / m\xc8 u < PTB *r-m\xfa w (STC #488).\footnote{See JAM 1973b (GL:29); such speculations were debunked in the 2nd Printing (1982) of GL, p. 675.}

\subsubsection{Failure to recognize that separately reconstructed etyma are really co-allofams}
An opposite type of error is to overlook the etymological identity between sets of forms, assigning them to separate etyma when they are really co-allofams. Thus STC sets up two independent PTB roots, both with the shape *dyam, one meaning `full; fill' (STC #226) and the other glossed as `straight' (STC #227). Yet it can be shown that the latter root also means `flat', and that all reflexes of #226 and #227 may be subsumed under a single etymon, with the underlying idea being ``perfection in a certain dimension''.\footnote{See JAM 1988a:4-9, and above 3.4.2(c), 7.5(6).} Similarly, I was slow to recognize that two roots I had set up separately, PLB *dzay\xaa  `cattle; domestic animal' (GSTC #129) and Kamarupan *tsa\xf2 y `elephant; cattle' (GSTC #143) are really one and the same.\footnote{I have argued that a third root set up in GSTC (#106), *(t)sa\xf2 y & *(d)za\xf2 y \xd4 temperament / aptitude / talent\xd5 , is also related, the common notion being \xd4 property (either material or intellectual)\xd5 . See JAM 1985a:44-45; 1988a:10-13; and above 5.5.2(1b), 5.5.2(2).}

\subsubsection{Double-dipping}
This embarrassing situation occurs when an author inadvertently assigns the same form in a daughter language to two different etyma, perhaps within the pages of the same book, but more likely in separate articles.\footnote{The computer can be very useful in deciding between alternative etymologies. Once \xd2 sound-laws\xd3  have been formulated, computer checking can test whether a particular reconstruction follows the laws, identifying inconsistencies in the re\xdf exes of the same proto-element in a given language. Such a methodology has been applied to the Tamangic languages, using the \xd2 reconstruction engine\xd3  developed by J.B. Lowe at STEDT in collaboration with Martine Mazaudon and Boyd Michailovsky during their sojourns at Berkeley as visiting scholars (1987-89, 1990-91).} At different times I have compared Chinese \xae B `lip' (OC \x84 ``iw\xfa n) to both PTB *dyal (above 9.2.1) and *m-ts(y)ul, finally deciding in favor of the latter (above 9.22(4), 9.2.4). It is perfectly legitimate to change one's mind, as long as one explains why. The best course is to present the alternative etymologies together, inviting the reader to choose between them.

\subsubsection{Misanalyses of compounds}
A vast number of words in TB languages are di- or tri-syllabic compounds, a fact which greatly complicates the task of etymologization.\footnote{See JAM 1978a (VSTB):58-72.} Many traps lie in wait for the analyst, leading to potential errors of several kinds, all of which I have been guilty of at one time or another:

\paragraph{Wrong segmentation}
This can happen when a form in an inadequately transcribed source is not syllabified. The Pochury and Sangtam forms for `star', transcribed as awutsi and chinghi, respectively, in the little glossaries compiled by the Nagaland Bhasha Parishad,\footnote{Hindi Pochury English Dictionary (1972); Hindi Sangtam English Dictionary (1973). Kohima: Linguistic Circle of Nagaland.} should be segmented as a-wu-tsi and ching-hi, and not as a-wut-si and chi-nghi, as I imprudently did in JAM 1980:21.

\paragraph{Misunderstanding the meaning of a constituent}
A special case of this problem is mistaking an affix for a root, especially likely to occur when no grammatical description exists for a language. Several Naga languages have dissyllabic forms for `moon' with similar final syllables, e.g. Chang litnyu, Konyak linnyu, Phom linny\xd9 u, Sangtam chonu, Liangmai chahiu. Yet these final elements do not constitute a new root meaning `moon', as I had originally guessed; rather they represent an abstract formative, ultimately grammaticalized from a root *n(y)u `mother', that occurs in nouns from all sorts of semantic fields (e.g. Chang chinyu `center', henyu `ladder', lamnyu `road', pinyu `snake').\footnote{See JAM (\xd2 Stars, moon, spirits\xd3 )1980:35; for the suf\xde xal use of morphemes meaning \xd4 mother\xd5 , see JAM 1991e.}

\paragraph{Choosing the wrong syllable of a compound for an etymology}
This can happen when two different syllables of a compound are phonologically similar, especially if one is dealing with a poorly known language with depleted final consonants, e.g. Guiqiong Ganzi t\xc6 h\xfa \xb0 \xb0  s~\x81 \xb0 \xb0  and Ersu \xa8 \xbd \xb0 \xb0  ji\xb0 \xb0  `otter'. Which syllables are to be ascribed to PTB *sram ?\footnote{See above 7.1(1).}

\paragraph{Semantic leaps}
Deciding how much semantic latitude to allow among putative cognates is definitely an art rather than a science. Here as elsewhere a middle-of-the-road approach is necessary, neither overly conservative nor too wildly speculative. As a positive example of a promising new etymology involving a semantic leap, we may offer *m-t(s)i `salt / yeast'.\footnote{Above 3.3.1.} Phonologically the reflexes correspond perfectly well. On the other hand, the semantic association between `salt' and `yeast' has yet to be attested elsewhere, even though it has great initial plausibility. Both are efficacious substances that have dramatic effects on the taste of food or drink; their lack renders the food or drink insipid.\footnote{Yeast is used for brewing liquor rather than for baking bread in East and SE Asia.}

\section*{}
Although I feel that we are entering a new era of etymological responsibility in TB/ST studies  the bar has been raised, as it were  I am not suggesting that we turn our field into a ``tough neighborhood'' like that of the Indo-Europeanists. In particular I hope we can avoid the ``Gotcha!'' attitude,\footnote{Non-American readers might need a word of explanation here: \xd2 Gotcha\xd3  is an attempt to render the colloquial pronunciation of \xd2 (I\xd5 ve) got you (now)!\xd3 , a triumphant phrase used by someone who feels he has won an argument.} whereby if a single error, real or fancied, is found in an article or book, the whole work is impugned. This attitude is encapsulated in the dreadful maxim {\it Falsum in uno, falsum in omnibus}.\footnote{ \xd2 If one thing is wrong, it\xd5 s all wrong.\xd3  This was the approach of Miller\xd5 s (1974) bitter review of STC, in which he tried to kill the Conspectus just as it was born. In my \xd2 rejection\xd3  of his \xd2 Conspectus inspection\xd3  I characterized his strategy as follows: (a) make some criticism of a particular point, no matter how trivial, irrelevant or obfuscatory; (b) claim that tout se tient, and that the entire work stands or falls on the point in question; (c) beat the point elaborately to death; (d) avoid any substantive comments by pleading lack of space (JAM 1975a:157).}

Historical linguists cannot afford to be too thin-skinned, as long as criticism is fair, constructive, and proportionate. As I have said in print, ``I ask nothing better than to be corrected.''\footnote{JAM 1985b (\xd2 Out on a limb\xd3 ):422.} Or again, ``We can take comfort from our mistakes. Reconstruction of a proto-lexicon is a piecemeal process. It is hardly surprising that we stumble along from one half-truth to another, as we try to trace the [phonological and] semantic interconnections among our reconstructed etyma. We should not be discouraged if we barge off down blind alleys occasionally, or if the solution to one problem raises as many questions as it answers.''\footnote{JAM 1988a:13.}

After all, a computerized etymological enterprise by its very nature is eminently revisable.

\subsection{Desiderative}
I am acutely aware of the incompleteness of this Handbook. As noted in the Preface, we still have a long way to go before comparative/historical TB/ST studies are as advanced as they deserve to be. Despite the quickening pace of research, our knowledge of the various branches of this multifarious family remains highly uneven. With a few important exceptions,\footnote{These bright spots include Proto-Karen (Haudricourt, Jones, Solnit), PNN (French), Proto-Tani (J. Sun), Proto-Tamangic (Mazaudon), Proto-Kiranti (Michailovsky), Proto-Central Chin (VanBik), Proto-Lolo-Burmese (Burling, Matisoff, Bradley). See the References.} reliable reconstructions at the subgroup level are not yet available, so that ``teleo-reconstruction'' has to be resorted to.\footnote{See Benedict 1973.} Many more roots remain to be reconstructed at all taxonomic levels of the family.

Much remains to be done on the Chinese side as well, and we seem destined for a period of flux until the dust settles and competing reconstructions have sorted themselves out.

A large lacuna in this Handbook is the lack of a systematic treatment of tone. This is not because the topic does not interest me, but rather because it merits a book-length treatment by itself. We are only just coming to appreciate the richness and variety of TB tone systems as more and more data become available.\footnote{For a rough typology of TB tone systems, see JAM 1999a. Weidert (1987) is an attempt to reduce all TB tone systems to a single proto-phonational contrast among clear, breathy, and creaky voice qualities, but is marred by an over-formalistic and disorganized presentation which renders it virtually incomprehensible. See the review by JAM (1994c).} The big questions are still open, especially the key issue of monogenesis vs. polygenesis: Can we reconstruct a single tonal system at the PTB level? At the PST level? If so, was this original system primarily phonational or melodic? Or are tonogenesis and tonoexodus cyclical processes, with tones having arisen repeatedly and independently in the various branches of TB, so that even if there was an ``original'' system, it can no longer be recovered?

All in all, it is hard not to be optimistic about the future of TB/ST linguistics, as fieldwork opportunities increase and new generations of talented researchers enter the discipline. Eventually it seems inevitable that scholars throughout the world will share their information more and more, granting mutual access to their databases for the common good. On the other hand, too many TB languages are endangered, and may well disappear before they have been adequately recorded.

In any case, ``the reconstruction of PTB is a noble enterprise, where a spirit of competitive territoriality is out of place. We should pool our knowledge and encourage each other to venture outside of our specialized niches, so that we begin to appreciate the full range of TB languages...''\footnote{JAM 1982a:41.}

\end{document}

